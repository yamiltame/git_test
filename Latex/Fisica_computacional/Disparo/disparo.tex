\documentclass[10pt]{article}
\usepackage{graphicx}
\usepackage[spanish]{babel}
\usepackage{amsmath}
\usepackage{amssymb}

\begin{document}
\title{M\'etodo del disparo para resolver una ecuaci\'on diferencial de 
segundo orden con condiciones de frontera}
\author{Mauricio Yamil Tame Soria}
\maketitle

\begin{abstract}
Resolvemos num\'ericamente una ecuaci\'on utilizando el m\'etodo del 
disparo con la aproximaci\'on de Newton
\end{abstract}

\emph{Palabras clave: Disparo, condiciones de frontera, edo segundo orden}

\section{Introducci\'on}
Tenemos una ecuaci\'on $y''=f(x,y,y')$ con condiciones de frontera $y(a)=\alpha$ y $y(b)=\beta$, 
utilizando el m\'etodo del disparo obtendremos una 
soluci\'on num\'erica, la cual graficaremos. Tambi\'en compararemos nuestras soluciones con la soluci\'on exacta y lo haremos con distintas 
resoluciones para el RungeKutta y distintas tolerancias para la condici\'on de frontera. Nuestra ecuaci\'on es la siguiente.
 
\begin{equation}
	y''=2y^3-6y-2x^3 \hspace{0.5cm} 1 \leq x \leq 2 \hspace{0.5cm} y(1)=2 \hspace{0.25cm} y(2)=5/2
\label{sistema}
\end{equation}

\section{Planteamiento del Disparo}
Primero necesitamos pasar nuestro problema de valores de frontera a un problema de valor inicial. Para esto proponemos el siguiente sistema

\begin{equation}
	y'=z \hspace{0.5cm} z'=2y^3-6y-2x^3	\hspace{0.5cm} 1 \leq x \leq 2 \hspace{0.5cm} y(1)=2 \hspace{0.25cm} y'(1)=t
\label{sistema1}
\end{equation}

el sistema representa un problema de valores iniciales. Resolviendo el sistema obtenemos una soluci\'on que tiene un cierto valor en la frontera, es 
decir $y(b)$ tiene un valor que depende del par\'ametro $t$ entonces realmente nuestra soluci\'on es una funci\'on $y(x,t)$ y necesitamos un $t$ tal 
que $y(b,t)=5/2$ que es nuestra condici\'on de frontera, esta condici\'on nos da otra ecuaci\'on 

\begin{equation}
	y(b,t)-5/2=0
\label{newton}
\end{equation}

la cual resolveremos usando el m\'etodo para encontrar ra\'ices de Newton, entonces necesitamos resolver por $\frac{\partial y(b,t)}{\partial t}$ en la 
frontera, para esto planteamos el otro sistema definiendo 
$u=\frac{\partial y}{\partial t}$ y derivamos entonces $\frac{\partial y''}{\partial t}=\frac{\partial f}{\partial t}$ utilizando la regla de la cadena
 y sabiendo que $x$ no depende de $t$, adem\'as suponemos que el orden de 
derivaci\'on de $x$ y $t$ puede cambiarse entonces la 
ecuaci\'on (\ref{sistema}) derivada respecto a $t$ queda

\begin{equation} 
	u''=(6y^2-6)u \hspace{0.25cm} 1\leq x\leq 2 \hspace{0.25cm} u(a,t)=0 \hspace{0.25cm} u'(a,t)=1
\label{sistemita}
\end{equation}

de esta manera para Newton debemos resolver dos problemas de valor inicial, el problema (\ref{sistema}) y (\ref{sistemita}) por lo tanto tenemos un 
sistema de 4 ecuaciones de primer orden de la siguiente manera

\begin{eqnarray} 
	y'=z \\
	z'=2y^3-6y-2x^3 \\
	u'=p \\
	p'=(6y^2-6)u
\end{eqnarray}

este sistema se resuelve usando el RungeKutta de orden 4 programado en el c\'odigo Disparo.cpp. Una vez resuelto el sistema tenemos $\frac{\partial 
y}{\partial t}$ y tenemos $y(b,t)$ entonces hacemos Newton para obtener una aproximaci\'on de $t$ para la ecuaci\'on (\ref{newton}) de la siguiente 
manera 
\begin{equation}
	t_{k+1}=t_k - \frac{y(b,t_k)-5/2}{u(b,t_k)}
\end{equation}

as\'i obtenemos un nuevo $t_{k+1}$ que nos sirve para un nuevo diparo m\'as aproximado. Iterando de esta manera aproximamos la soluci\'on. 

\section{Resultados}
El c\'odigo se ejecut\'o con 2 valores para $\delta x$ y dos valores para la tolerancia: ${0.02,0.01}$ y ${1.0e-6,1.0e-8}$ respectivamente. Se 
compararon los resultados obtenidos num\'ericamente con los obtenidos por la funci\'on que representa la soluci\'on exacta y se grafic\'o la diferencia 
de los dos valores respecto a las $x$ de la siguiente manera

\begin{equation}
	error(x)=|y_{num}(x)-y_{exc}(x)|
\end{equation}
arrojando lo siguiente

\begin{figure}
	\centering
	\input{error_tol1e-06_h0.01.tex}
	\caption{error con tolerancia $10^{-6}$ y $\delta x=0.01$}
\end{figure}

\begin{figure}
	\centering
	\input{error_tol1e-06_h0.02.tex}
	\caption{error con tolerancia $10^{-6}$ y $\delta x=0.02$}
\end{figure}	
\begin{figure}
	\centering
	% GNUPLOT: LaTeX picture
\setlength{\unitlength}{0.240900pt}
\ifx\plotpoint\undefined\newsavebox{\plotpoint}\fi
\begin{picture}(1500,900)(0,0)
\sbox{\plotpoint}{\rule[-0.200pt]{0.400pt}{0.400pt}}%
\put(191.0,131.0){\rule[-0.200pt]{4.818pt}{0.400pt}}
\put(171,131){\makebox(0,0)[r]{$0$}}
\put(1419.0,131.0){\rule[-0.200pt]{4.818pt}{0.400pt}}
\put(191.0,313.0){\rule[-0.200pt]{4.818pt}{0.400pt}}
\put(171,313){\makebox(0,0)[r]{$0.005$}}
\put(1419.0,313.0){\rule[-0.200pt]{4.818pt}{0.400pt}}
\put(191.0,495.0){\rule[-0.200pt]{4.818pt}{0.400pt}}
\put(171,495){\makebox(0,0)[r]{$0.01$}}
\put(1419.0,495.0){\rule[-0.200pt]{4.818pt}{0.400pt}}
\put(191.0,677.0){\rule[-0.200pt]{4.818pt}{0.400pt}}
\put(171,677){\makebox(0,0)[r]{$0.015$}}
\put(1419.0,677.0){\rule[-0.200pt]{4.818pt}{0.400pt}}
\put(191.0,859.0){\rule[-0.200pt]{4.818pt}{0.400pt}}
\put(171,859){\makebox(0,0)[r]{$0.02$}}
\put(1419.0,859.0){\rule[-0.200pt]{4.818pt}{0.400pt}}
\put(191.0,131.0){\rule[-0.200pt]{0.400pt}{4.818pt}}
\put(191,90){\makebox(0,0){$1$}}
\put(191.0,839.0){\rule[-0.200pt]{0.400pt}{4.818pt}}
\put(441.0,131.0){\rule[-0.200pt]{0.400pt}{4.818pt}}
\put(441,90){\makebox(0,0){$1.2$}}
\put(441.0,839.0){\rule[-0.200pt]{0.400pt}{4.818pt}}
\put(690.0,131.0){\rule[-0.200pt]{0.400pt}{4.818pt}}
\put(690,90){\makebox(0,0){$1.4$}}
\put(690.0,839.0){\rule[-0.200pt]{0.400pt}{4.818pt}}
\put(940.0,131.0){\rule[-0.200pt]{0.400pt}{4.818pt}}
\put(940,90){\makebox(0,0){$1.6$}}
\put(940.0,839.0){\rule[-0.200pt]{0.400pt}{4.818pt}}
\put(1189.0,131.0){\rule[-0.200pt]{0.400pt}{4.818pt}}
\put(1189,90){\makebox(0,0){$1.8$}}
\put(1189.0,839.0){\rule[-0.200pt]{0.400pt}{4.818pt}}
\put(1439.0,131.0){\rule[-0.200pt]{0.400pt}{4.818pt}}
\put(1439,90){\makebox(0,0){$2$}}
\put(1439.0,839.0){\rule[-0.200pt]{0.400pt}{4.818pt}}
\put(191.0,131.0){\rule[-0.200pt]{0.400pt}{175.375pt}}
\put(191.0,131.0){\rule[-0.200pt]{300.643pt}{0.400pt}}
\put(1439.0,131.0){\rule[-0.200pt]{0.400pt}{175.375pt}}
\put(191.0,859.0){\rule[-0.200pt]{300.643pt}{0.400pt}}
\put(30,495){\makebox(0,0){error}}
\put(815,29){\makebox(0,0){$x$}}
\put(191,131){\makebox(0,0){$\bullet$}}
\put(203,131){\makebox(0,0){$\bullet$}}
\put(216,131){\makebox(0,0){$\bullet$}}
\put(228,132){\makebox(0,0){$\bullet$}}
\put(241,132){\makebox(0,0){$\bullet$}}
\put(253,132){\makebox(0,0){$\bullet$}}
\put(266,132){\makebox(0,0){$\bullet$}}
\put(278,132){\makebox(0,0){$\bullet$}}
\put(291,133){\makebox(0,0){$\bullet$}}
\put(303,133){\makebox(0,0){$\bullet$}}
\put(316,133){\makebox(0,0){$\bullet$}}
\put(328,133){\makebox(0,0){$\bullet$}}
\put(341,134){\makebox(0,0){$\bullet$}}
\put(353,134){\makebox(0,0){$\bullet$}}
\put(366,134){\makebox(0,0){$\bullet$}}
\put(378,134){\makebox(0,0){$\bullet$}}
\put(391,135){\makebox(0,0){$\bullet$}}
\put(403,135){\makebox(0,0){$\bullet$}}
\put(416,135){\makebox(0,0){$\bullet$}}
\put(428,135){\makebox(0,0){$\bullet$}}
\put(441,136){\makebox(0,0){$\bullet$}}
\put(453,136){\makebox(0,0){$\bullet$}}
\put(466,136){\makebox(0,0){$\bullet$}}
\put(478,137){\makebox(0,0){$\bullet$}}
\put(491,137){\makebox(0,0){$\bullet$}}
\put(503,137){\makebox(0,0){$\bullet$}}
\put(515,138){\makebox(0,0){$\bullet$}}
\put(528,138){\makebox(0,0){$\bullet$}}
\put(540,138){\makebox(0,0){$\bullet$}}
\put(553,139){\makebox(0,0){$\bullet$}}
\put(565,139){\makebox(0,0){$\bullet$}}
\put(578,140){\makebox(0,0){$\bullet$}}
\put(590,140){\makebox(0,0){$\bullet$}}
\put(603,141){\makebox(0,0){$\bullet$}}
\put(615,141){\makebox(0,0){$\bullet$}}
\put(628,142){\makebox(0,0){$\bullet$}}
\put(640,142){\makebox(0,0){$\bullet$}}
\put(653,143){\makebox(0,0){$\bullet$}}
\put(665,143){\makebox(0,0){$\bullet$}}
\put(678,144){\makebox(0,0){$\bullet$}}
\put(690,144){\makebox(0,0){$\bullet$}}
\put(703,145){\makebox(0,0){$\bullet$}}
\put(715,146){\makebox(0,0){$\bullet$}}
\put(728,146){\makebox(0,0){$\bullet$}}
\put(740,147){\makebox(0,0){$\bullet$}}
\put(753,148){\makebox(0,0){$\bullet$}}
\put(765,149){\makebox(0,0){$\bullet$}}
\put(778,150){\makebox(0,0){$\bullet$}}
\put(790,150){\makebox(0,0){$\bullet$}}
\put(803,151){\makebox(0,0){$\bullet$}}
\put(815,152){\makebox(0,0){$\bullet$}}
\put(827,153){\makebox(0,0){$\bullet$}}
\put(840,154){\makebox(0,0){$\bullet$}}
\put(852,156){\makebox(0,0){$\bullet$}}
\put(865,157){\makebox(0,0){$\bullet$}}
\put(877,158){\makebox(0,0){$\bullet$}}
\put(890,159){\makebox(0,0){$\bullet$}}
\put(902,161){\makebox(0,0){$\bullet$}}
\put(915,162){\makebox(0,0){$\bullet$}}
\put(927,164){\makebox(0,0){$\bullet$}}
\put(940,165){\makebox(0,0){$\bullet$}}
\put(952,167){\makebox(0,0){$\bullet$}}
\put(965,169){\makebox(0,0){$\bullet$}}
\put(977,170){\makebox(0,0){$\bullet$}}
\put(990,172){\makebox(0,0){$\bullet$}}
\put(1002,174){\makebox(0,0){$\bullet$}}
\put(1015,177){\makebox(0,0){$\bullet$}}
\put(1027,179){\makebox(0,0){$\bullet$}}
\put(1040,181){\makebox(0,0){$\bullet$}}
\put(1052,184){\makebox(0,0){$\bullet$}}
\put(1065,186){\makebox(0,0){$\bullet$}}
\put(1077,189){\makebox(0,0){$\bullet$}}
\put(1090,192){\makebox(0,0){$\bullet$}}
\put(1102,195){\makebox(0,0){$\bullet$}}
\put(1115,198){\makebox(0,0){$\bullet$}}
\put(1127,202){\makebox(0,0){$\bullet$}}
\put(1139,205){\makebox(0,0){$\bullet$}}
\put(1152,209){\makebox(0,0){$\bullet$}}
\put(1164,213){\makebox(0,0){$\bullet$}}
\put(1177,217){\makebox(0,0){$\bullet$}}
\put(1189,222){\makebox(0,0){$\bullet$}}
\put(1202,227){\makebox(0,0){$\bullet$}}
\put(1214,232){\makebox(0,0){$\bullet$}}
\put(1227,237){\makebox(0,0){$\bullet$}}
\put(1239,242){\makebox(0,0){$\bullet$}}
\put(1252,248){\makebox(0,0){$\bullet$}}
\put(1264,254){\makebox(0,0){$\bullet$}}
\put(1277,261){\makebox(0,0){$\bullet$}}
\put(1289,268){\makebox(0,0){$\bullet$}}
\put(1302,275){\makebox(0,0){$\bullet$}}
\put(1314,283){\makebox(0,0){$\bullet$}}
\put(1327,291){\makebox(0,0){$\bullet$}}
\put(1339,300){\makebox(0,0){$\bullet$}}
\put(1352,309){\makebox(0,0){$\bullet$}}
\put(1364,319){\makebox(0,0){$\bullet$}}
\put(1377,329){\makebox(0,0){$\bullet$}}
\put(1389,340){\makebox(0,0){$\bullet$}}
\put(1402,351){\makebox(0,0){$\bullet$}}
\put(1414,363){\makebox(0,0){$\bullet$}}
\put(1427,376){\makebox(0,0){$\bullet$}}
\put(1439,390){\makebox(0,0){$\bullet$}}
\put(191.0,131.0){\rule[-0.200pt]{0.400pt}{175.375pt}}
\put(191.0,131.0){\rule[-0.200pt]{300.643pt}{0.400pt}}
\put(1439.0,131.0){\rule[-0.200pt]{0.400pt}{175.375pt}}
\put(191.0,859.0){\rule[-0.200pt]{300.643pt}{0.400pt}}
\end{picture}

	\caption{error con tolerancia $10^{-8}$ y $\delta x=0.01$}
\end{figure}	
\begin{figure}
	\centering
	% GNUPLOT: LaTeX picture
\setlength{\unitlength}{0.240900pt}
\ifx\plotpoint\undefined\newsavebox{\plotpoint}\fi
\begin{picture}(1500,900)(0,0)
\sbox{\plotpoint}{\rule[-0.200pt]{0.400pt}{0.400pt}}%
\put(191.0,131.0){\rule[-0.200pt]{4.818pt}{0.400pt}}
\put(171,131){\makebox(0,0)[r]{$0$}}
\put(1419.0,131.0){\rule[-0.200pt]{4.818pt}{0.400pt}}
\put(191.0,313.0){\rule[-0.200pt]{4.818pt}{0.400pt}}
\put(171,313){\makebox(0,0)[r]{$0.005$}}
\put(1419.0,313.0){\rule[-0.200pt]{4.818pt}{0.400pt}}
\put(191.0,495.0){\rule[-0.200pt]{4.818pt}{0.400pt}}
\put(171,495){\makebox(0,0)[r]{$0.01$}}
\put(1419.0,495.0){\rule[-0.200pt]{4.818pt}{0.400pt}}
\put(191.0,677.0){\rule[-0.200pt]{4.818pt}{0.400pt}}
\put(171,677){\makebox(0,0)[r]{$0.015$}}
\put(1419.0,677.0){\rule[-0.200pt]{4.818pt}{0.400pt}}
\put(191.0,859.0){\rule[-0.200pt]{4.818pt}{0.400pt}}
\put(171,859){\makebox(0,0)[r]{$0.02$}}
\put(1419.0,859.0){\rule[-0.200pt]{4.818pt}{0.400pt}}
\put(191.0,131.0){\rule[-0.200pt]{0.400pt}{4.818pt}}
\put(191,90){\makebox(0,0){$1$}}
\put(191.0,839.0){\rule[-0.200pt]{0.400pt}{4.818pt}}
\put(441.0,131.0){\rule[-0.200pt]{0.400pt}{4.818pt}}
\put(441,90){\makebox(0,0){$1.2$}}
\put(441.0,839.0){\rule[-0.200pt]{0.400pt}{4.818pt}}
\put(690.0,131.0){\rule[-0.200pt]{0.400pt}{4.818pt}}
\put(690,90){\makebox(0,0){$1.4$}}
\put(690.0,839.0){\rule[-0.200pt]{0.400pt}{4.818pt}}
\put(940.0,131.0){\rule[-0.200pt]{0.400pt}{4.818pt}}
\put(940,90){\makebox(0,0){$1.6$}}
\put(940.0,839.0){\rule[-0.200pt]{0.400pt}{4.818pt}}
\put(1189.0,131.0){\rule[-0.200pt]{0.400pt}{4.818pt}}
\put(1189,90){\makebox(0,0){$1.8$}}
\put(1189.0,839.0){\rule[-0.200pt]{0.400pt}{4.818pt}}
\put(1439.0,131.0){\rule[-0.200pt]{0.400pt}{4.818pt}}
\put(1439,90){\makebox(0,0){$2$}}
\put(1439.0,839.0){\rule[-0.200pt]{0.400pt}{4.818pt}}
\put(191.0,131.0){\rule[-0.200pt]{0.400pt}{175.375pt}}
\put(191.0,131.0){\rule[-0.200pt]{300.643pt}{0.400pt}}
\put(1439.0,131.0){\rule[-0.200pt]{0.400pt}{175.375pt}}
\put(191.0,859.0){\rule[-0.200pt]{300.643pt}{0.400pt}}
\put(30,495){\makebox(0,0){error}}
\put(815,29){\makebox(0,0){$x$}}
\put(191,131){\makebox(0,0){$\bullet$}}
\put(216,132){\makebox(0,0){$\bullet$}}
\put(241,133){\makebox(0,0){$\bullet$}}
\put(266,133){\makebox(0,0){$\bullet$}}
\put(291,134){\makebox(0,0){$\bullet$}}
\put(316,135){\makebox(0,0){$\bullet$}}
\put(341,136){\makebox(0,0){$\bullet$}}
\put(366,137){\makebox(0,0){$\bullet$}}
\put(391,138){\makebox(0,0){$\bullet$}}
\put(416,139){\makebox(0,0){$\bullet$}}
\put(441,140){\makebox(0,0){$\bullet$}}
\put(466,141){\makebox(0,0){$\bullet$}}
\put(491,142){\makebox(0,0){$\bullet$}}
\put(515,144){\makebox(0,0){$\bullet$}}
\put(540,145){\makebox(0,0){$\bullet$}}
\put(565,147){\makebox(0,0){$\bullet$}}
\put(590,148){\makebox(0,0){$\bullet$}}
\put(615,150){\makebox(0,0){$\bullet$}}
\put(640,152){\makebox(0,0){$\bullet$}}
\put(665,154){\makebox(0,0){$\bullet$}}
\put(690,156){\makebox(0,0){$\bullet$}}
\put(715,159){\makebox(0,0){$\bullet$}}
\put(740,162){\makebox(0,0){$\bullet$}}
\put(765,165){\makebox(0,0){$\bullet$}}
\put(790,168){\makebox(0,0){$\bullet$}}
\put(815,172){\makebox(0,0){$\bullet$}}
\put(840,176){\makebox(0,0){$\bullet$}}
\put(865,180){\makebox(0,0){$\bullet$}}
\put(890,185){\makebox(0,0){$\bullet$}}
\put(915,190){\makebox(0,0){$\bullet$}}
\put(940,196){\makebox(0,0){$\bullet$}}
\put(965,202){\makebox(0,0){$\bullet$}}
\put(990,210){\makebox(0,0){$\bullet$}}
\put(1015,218){\makebox(0,0){$\bullet$}}
\put(1040,226){\makebox(0,0){$\bullet$}}
\put(1065,236){\makebox(0,0){$\bullet$}}
\put(1090,247){\makebox(0,0){$\bullet$}}
\put(1115,259){\makebox(0,0){$\bullet$}}
\put(1139,272){\makebox(0,0){$\bullet$}}
\put(1164,287){\makebox(0,0){$\bullet$}}
\put(1189,304){\makebox(0,0){$\bullet$}}
\put(1214,322){\makebox(0,0){$\bullet$}}
\put(1239,343){\makebox(0,0){$\bullet$}}
\put(1264,365){\makebox(0,0){$\bullet$}}
\put(1289,391){\makebox(0,0){$\bullet$}}
\put(1314,420){\makebox(0,0){$\bullet$}}
\put(1339,451){\makebox(0,0){$\bullet$}}
\put(1364,487){\makebox(0,0){$\bullet$}}
\put(1389,527){\makebox(0,0){$\bullet$}}
\put(1414,572){\makebox(0,0){$\bullet$}}
\put(1439,622){\makebox(0,0){$\bullet$}}
\put(191.0,131.0){\rule[-0.200pt]{0.400pt}{175.375pt}}
\put(191.0,131.0){\rule[-0.200pt]{300.643pt}{0.400pt}}
\put(1439.0,131.0){\rule[-0.200pt]{0.400pt}{175.375pt}}
\put(191.0,859.0){\rule[-0.200pt]{300.643pt}{0.400pt}}
\end{picture}

	\caption{error con tolerancia $10^{-8}$ y $\delta x=0.02$}
\end{figure}	

Para la soluci\'on num\'ericagraficaremos la de menor error que es 

\begin{figure}
	\input{solucion_tol1e-08_h0.01.tex}
	\caption{solucion con la menor toleracnica y $\delta x$}
\end{figure}

\section{Conclusiones}
En mi opini\'on la dificultad del m\'etodo corresponde a la implementaci\'on del RungeKutta. Mientras 
mayor resoluci\'on la soluci\'on tendr\'a una mayor precisi\'on. Uno debe ser cuidadoso al escoger el $t$ 
inicial, en este sistema utilizar $\frac{\beta - \alpha}{b - a}$ provoca que el m\'etodo de newton raphson 
no tenga soluci\'on ya que crece mucho y en unas cuantas iteraciones con Newton la variable $t$ se vuelve 
nan (not a number); observando la soluci\'on exacta podemos estimar una $t$ inicial cercana a cero pero no 
siempre nos dar\'an la soluci\'on exacta entonces se gener\'o una iteraci\'on sobre el \'angulo inicial, 
se escogi\'o un rango de $\pi /4$ y $-\pi /4$ en 10 \'angulos distintos, de esas iteraciones se nota que 
si $t$ inicial es mayor que 0.02 el valor de $t$ no deja de crecer. La iteraci\'on sobre $t_{init}$ 
permiti\'o encontrar un $t$ favorable para el m\'etodo de Newton.
 
\begin{thebibliography}{9}
\bibitem{tesis}
Luis Jos\'e Berbes\'i� M\'arquez 
	\emph{Soluci´on Num´erica de Problemas de Valor de Frontera
para Ecuaciones Diferenciales Ordinarias}
Universidad de Los Andes
Facultad de Ciencias
Departamento de Matem\'aticas
Grupo de Ciencias de la Computaci\'on
\end{thebibliography}

\end{document}
