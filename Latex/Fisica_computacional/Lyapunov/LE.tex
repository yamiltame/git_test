\documentclass[10pt]{article}
\usepackage{graphicx}
\usepackage[spanish]{babel}
\usepackage{amsmath}
\usepackage{amssymb}
\usepackage{float}

\begin{document}
\title{Espectro de Lyapunov para el modelo hyper-ca\'otico de 
Rossler}
\author{Mauricio Yamil Tame Soria}
\maketitle

\begin{abstract}
Obtendremos el espectro de Lyapunov para el modelo de Rossler en 
lugares cercanos a sus puntos fijos utilizando el m\'etodo estandar 
descrito en el art\'iculo que nos envi\'o.
\end{abstract}

\emph{Palabras clave: Exponentes Lyapunov, Rossler, hypercaos}

\section{Introducci\'on}
Los exponentes de Lyapunov son un indicador de qu\'e tanto 
difieren trayectorias que iniciaron en puntos muy cercanos, es 
decir, nos indican que tan sensible es el sistema a las 
condiciones iniciales.
Los sistemas hyperca\'oticos tienen varios exponentes de lyapunov 
que son positivos.
El modelo de Rossler es un sistema de cuatro dimensiones que es 
hyperca\'otico.
 

\section{Sobre el sistema}
Las ecuaciones del modelo de Rossler son las siguientes:

\begin{eqnarray} 
	\dot{z_1}=-(z_2 + z_3) \\
	\dot{z_2}=z_1 + az_2 + z_4 \\
	\dot{z_3}=b + z_1 z_3 \\
	\dot{z_4}=cz_4 - dz_3
\end{eqnarray}

En este caso los parametros $a,b,c,d$ son $0.25,3.0,0.05,0.5$ 
respectivamente. El an\'alisis de los exponentes de lyapunov nos 
interesa cerca de los puntos fijos, entonces debemos calcularlos. 
Los puntos fijos cumplen que 
$\dot{z_1}=\dot{z_2}=\dot{z_3}=\dot{z_4}=0$, entonces tenemos que 
los puntos fijos $\tilde{z_1}, \tilde{z_4}, \tilde{z_4}, 
\tilde{z_4}$ cumplen lo siguiente

\begin{eqnarray} 
	0=-(\tilde{z_2} + \tilde{z_3}) \\
	0=\tilde{z_1} + a\tilde{z_2} + \tilde{z_4} \\
	0=b + \tilde{z_1} \tilde{z_3} \\
	0=c\tilde{z_4} - d\tilde{z_3}
\end{eqnarray}

Resolviendo por sustituci\'on obtenemos lo siguiente

\begin{eqnarray} 
	\tilde{z_2} = -\tilde{z_3} \\
	\tilde{z_1} = \tilde{z_3} (a-\frac{d}{c}) \\
	\tilde{z_4} = \frac{d}{c} \tilde{z_3} \\
	\tilde{z_3} = \pm \sqrt{\frac{b}{\frac{d}{c} - a}}
\end{eqnarray}

Resolviendo obtenemos dos puntos, (-5.4083,-0.5547,0.5547,5.5470) 
y (5.4083,0.5547,-0.5547,-5.5470) los cuales usaremos como orientaci\'on para determinar las condiciones 
iniciales.

\section{Resultados}

Para el primer punto fijo considerando el orden anterior usamos la condici\'on 
incial (-5.5083, -0.6574, 0.6547, 5.6470) y corremos el tiempo de $t=0$ a $t=4$ con un paso 
$\Delta 
t=0.002$ sobre el sistema din\'amico usando RK4. Las 4 dimensiones dificultan la visualizaci\'on, 
por eso se sobreponen las gr\'aficas de cada variable respecto al tiempo. Para el c\'alculo del 
espectro usamos el punto inicial antes del ciclo transitorio, pues si usamos el punto que nos 
devuelve el ciclo transitorio el programa regresa un resultado nan(not a number) que se refiere al 
infinito,es decir, el sistema diverge rapid\'isimo cuando est\'a lejos de los puntos fijos. Las 
gr\'aficas de la evoluci\'on del sistema en cada variable son

\begin{figure}[H]
    \centering
    % GNUPLOT: LaTeX picture
\setlength{\unitlength}{0.240900pt}
\ifx\plotpoint\undefined\newsavebox{\plotpoint}\fi
\sbox{\plotpoint}{\rule[-0.200pt]{0.400pt}{0.400pt}}%
\begin{picture}(1500,900)(0,0)
\sbox{\plotpoint}{\rule[-0.200pt]{0.400pt}{0.400pt}}%
\put(131.0,131.0){\rule[-0.200pt]{4.818pt}{0.400pt}}
\put(111,131){\makebox(0,0)[r]{$-8$}}
\put(1419.0,131.0){\rule[-0.200pt]{4.818pt}{0.400pt}}
\put(131.0,235.0){\rule[-0.200pt]{4.818pt}{0.400pt}}
\put(111,235){\makebox(0,0)[r]{$-6$}}
\put(1419.0,235.0){\rule[-0.200pt]{4.818pt}{0.400pt}}
\put(131.0,339.0){\rule[-0.200pt]{4.818pt}{0.400pt}}
\put(111,339){\makebox(0,0)[r]{$-4$}}
\put(1419.0,339.0){\rule[-0.200pt]{4.818pt}{0.400pt}}
\put(131.0,443.0){\rule[-0.200pt]{4.818pt}{0.400pt}}
\put(111,443){\makebox(0,0)[r]{$-2$}}
\put(1419.0,443.0){\rule[-0.200pt]{4.818pt}{0.400pt}}
\put(131.0,547.0){\rule[-0.200pt]{4.818pt}{0.400pt}}
\put(111,547){\makebox(0,0)[r]{$0$}}
\put(1419.0,547.0){\rule[-0.200pt]{4.818pt}{0.400pt}}
\put(131.0,651.0){\rule[-0.200pt]{4.818pt}{0.400pt}}
\put(111,651){\makebox(0,0)[r]{$2$}}
\put(1419.0,651.0){\rule[-0.200pt]{4.818pt}{0.400pt}}
\put(131.0,755.0){\rule[-0.200pt]{4.818pt}{0.400pt}}
\put(111,755){\makebox(0,0)[r]{$4$}}
\put(1419.0,755.0){\rule[-0.200pt]{4.818pt}{0.400pt}}
\put(131.0,859.0){\rule[-0.200pt]{4.818pt}{0.400pt}}
\put(111,859){\makebox(0,0)[r]{$6$}}
\put(1419.0,859.0){\rule[-0.200pt]{4.818pt}{0.400pt}}
\put(131.0,131.0){\rule[-0.200pt]{0.400pt}{4.818pt}}
\put(131,90){\makebox(0,0){$0$}}
\put(131.0,839.0){\rule[-0.200pt]{0.400pt}{4.818pt}}
\put(295.0,131.0){\rule[-0.200pt]{0.400pt}{4.818pt}}
\put(295,90){\makebox(0,0){$0.5$}}
\put(295.0,839.0){\rule[-0.200pt]{0.400pt}{4.818pt}}
\put(458.0,131.0){\rule[-0.200pt]{0.400pt}{4.818pt}}
\put(458,90){\makebox(0,0){$1$}}
\put(458.0,839.0){\rule[-0.200pt]{0.400pt}{4.818pt}}
\put(622.0,131.0){\rule[-0.200pt]{0.400pt}{4.818pt}}
\put(622,90){\makebox(0,0){$1.5$}}
\put(622.0,839.0){\rule[-0.200pt]{0.400pt}{4.818pt}}
\put(785.0,131.0){\rule[-0.200pt]{0.400pt}{4.818pt}}
\put(785,90){\makebox(0,0){$2$}}
\put(785.0,839.0){\rule[-0.200pt]{0.400pt}{4.818pt}}
\put(949.0,131.0){\rule[-0.200pt]{0.400pt}{4.818pt}}
\put(949,90){\makebox(0,0){$2.5$}}
\put(949.0,839.0){\rule[-0.200pt]{0.400pt}{4.818pt}}
\put(1112.0,131.0){\rule[-0.200pt]{0.400pt}{4.818pt}}
\put(1112,90){\makebox(0,0){$3$}}
\put(1112.0,839.0){\rule[-0.200pt]{0.400pt}{4.818pt}}
\put(1276.0,131.0){\rule[-0.200pt]{0.400pt}{4.818pt}}
\put(1276,90){\makebox(0,0){$3.5$}}
\put(1276.0,839.0){\rule[-0.200pt]{0.400pt}{4.818pt}}
\put(1439.0,131.0){\rule[-0.200pt]{0.400pt}{4.818pt}}
\put(1439,90){\makebox(0,0){$4$}}
\put(1439.0,839.0){\rule[-0.200pt]{0.400pt}{4.818pt}}
\put(131.0,131.0){\rule[-0.200pt]{0.400pt}{175.375pt}}
\put(131.0,131.0){\rule[-0.200pt]{315.097pt}{0.400pt}}
\put(1439.0,131.0){\rule[-0.200pt]{0.400pt}{175.375pt}}
\put(131.0,859.0){\rule[-0.200pt]{315.097pt}{0.400pt}}
\put(30,495){\makebox(0,0){valor}}
\put(785,29){\makebox(0,0){$t$}}
\put(1279,818){\makebox(0,0)[r]{$x$}}
\put(1299.0,818.0){\rule[-0.200pt]{24.090pt}{0.400pt}}
\put(132,261){\usebox{\plotpoint}}
\put(132,261){\usebox{\plotpoint}}
\put(135,260.67){\rule{0.241pt}{0.400pt}}
\multiput(135.00,260.17)(0.500,1.000){2}{\rule{0.120pt}{0.400pt}}
\put(132.0,261.0){\rule[-0.200pt]{0.723pt}{0.400pt}}
\put(136,262){\usebox{\plotpoint}}
\put(140,261.67){\rule{0.241pt}{0.400pt}}
\multiput(140.00,261.17)(0.500,1.000){2}{\rule{0.120pt}{0.400pt}}
\put(136.0,262.0){\rule[-0.200pt]{0.964pt}{0.400pt}}
\put(141,263){\usebox{\plotpoint}}
\put(141.0,263.0){\rule[-0.200pt]{0.964pt}{0.400pt}}
\put(145.0,263.0){\usebox{\plotpoint}}
\put(150,263.67){\rule{0.241pt}{0.400pt}}
\multiput(150.00,263.17)(0.500,1.000){2}{\rule{0.120pt}{0.400pt}}
\put(145.0,264.0){\rule[-0.200pt]{1.204pt}{0.400pt}}
\put(151,265){\usebox{\plotpoint}}
\put(151.0,265.0){\rule[-0.200pt]{0.964pt}{0.400pt}}
\put(155.0,265.0){\usebox{\plotpoint}}
\put(155.0,266.0){\rule[-0.200pt]{1.204pt}{0.400pt}}
\put(160.0,266.0){\usebox{\plotpoint}}
\put(164,266.67){\rule{0.241pt}{0.400pt}}
\multiput(164.00,266.17)(0.500,1.000){2}{\rule{0.120pt}{0.400pt}}
\put(160.0,267.0){\rule[-0.200pt]{0.964pt}{0.400pt}}
\put(165.0,268.0){\rule[-0.200pt]{1.204pt}{0.400pt}}
\put(170.0,268.0){\usebox{\plotpoint}}
\put(170.0,269.0){\rule[-0.200pt]{1.204pt}{0.400pt}}
\put(175.0,269.0){\usebox{\plotpoint}}
\put(179,269.67){\rule{0.241pt}{0.400pt}}
\multiput(179.00,269.17)(0.500,1.000){2}{\rule{0.120pt}{0.400pt}}
\put(175.0,270.0){\rule[-0.200pt]{0.964pt}{0.400pt}}
\put(180.0,271.0){\rule[-0.200pt]{1.204pt}{0.400pt}}
\put(185.0,271.0){\usebox{\plotpoint}}
\put(190,271.67){\rule{0.241pt}{0.400pt}}
\multiput(190.00,271.17)(0.500,1.000){2}{\rule{0.120pt}{0.400pt}}
\put(185.0,272.0){\rule[-0.200pt]{1.204pt}{0.400pt}}
\put(191,273){\usebox{\plotpoint}}
\put(195,272.67){\rule{0.241pt}{0.400pt}}
\multiput(195.00,272.17)(0.500,1.000){2}{\rule{0.120pt}{0.400pt}}
\put(191.0,273.0){\rule[-0.200pt]{0.964pt}{0.400pt}}
\put(196,274){\usebox{\plotpoint}}
\put(200,273.67){\rule{0.241pt}{0.400pt}}
\multiput(200.00,273.17)(0.500,1.000){2}{\rule{0.120pt}{0.400pt}}
\put(196.0,274.0){\rule[-0.200pt]{0.964pt}{0.400pt}}
\put(201.0,275.0){\rule[-0.200pt]{1.204pt}{0.400pt}}
\put(206.0,275.0){\usebox{\plotpoint}}
\put(210,275.67){\rule{0.241pt}{0.400pt}}
\multiput(210.00,275.17)(0.500,1.000){2}{\rule{0.120pt}{0.400pt}}
\put(206.0,276.0){\rule[-0.200pt]{0.964pt}{0.400pt}}
\put(211,277){\usebox{\plotpoint}}
\put(215,276.67){\rule{0.241pt}{0.400pt}}
\multiput(215.00,276.17)(0.500,1.000){2}{\rule{0.120pt}{0.400pt}}
\put(211.0,277.0){\rule[-0.200pt]{0.964pt}{0.400pt}}
\put(216.0,278.0){\rule[-0.200pt]{1.204pt}{0.400pt}}
\put(221.0,278.0){\usebox{\plotpoint}}
\put(221.0,279.0){\rule[-0.200pt]{1.204pt}{0.400pt}}
\put(226.0,279.0){\usebox{\plotpoint}}
\put(226.0,280.0){\rule[-0.200pt]{1.445pt}{0.400pt}}
\put(232.0,280.0){\usebox{\plotpoint}}
\put(237,280.67){\rule{0.241pt}{0.400pt}}
\multiput(237.00,280.17)(0.500,1.000){2}{\rule{0.120pt}{0.400pt}}
\put(232.0,281.0){\rule[-0.200pt]{1.204pt}{0.400pt}}
\put(238,282){\usebox{\plotpoint}}
\put(242,281.67){\rule{0.241pt}{0.400pt}}
\multiput(242.00,281.17)(0.500,1.000){2}{\rule{0.120pt}{0.400pt}}
\put(238.0,282.0){\rule[-0.200pt]{0.964pt}{0.400pt}}
\put(243,283){\usebox{\plotpoint}}
\put(247,282.67){\rule{0.241pt}{0.400pt}}
\multiput(247.00,282.17)(0.500,1.000){2}{\rule{0.120pt}{0.400pt}}
\put(243.0,283.0){\rule[-0.200pt]{0.964pt}{0.400pt}}
\put(248.0,284.0){\rule[-0.200pt]{1.204pt}{0.400pt}}
\put(253.0,284.0){\usebox{\plotpoint}}
\put(253.0,285.0){\rule[-0.200pt]{1.445pt}{0.400pt}}
\put(259.0,285.0){\usebox{\plotpoint}}
\put(259.0,286.0){\rule[-0.200pt]{1.204pt}{0.400pt}}
\put(264.0,286.0){\usebox{\plotpoint}}
\put(269,286.67){\rule{0.241pt}{0.400pt}}
\multiput(269.00,286.17)(0.500,1.000){2}{\rule{0.120pt}{0.400pt}}
\put(264.0,287.0){\rule[-0.200pt]{1.204pt}{0.400pt}}
\put(270,288){\usebox{\plotpoint}}
\put(275,287.67){\rule{0.241pt}{0.400pt}}
\multiput(275.00,287.17)(0.500,1.000){2}{\rule{0.120pt}{0.400pt}}
\put(270.0,288.0){\rule[-0.200pt]{1.204pt}{0.400pt}}
\put(276,289){\usebox{\plotpoint}}
\put(280,288.67){\rule{0.241pt}{0.400pt}}
\multiput(280.00,288.17)(0.500,1.000){2}{\rule{0.120pt}{0.400pt}}
\put(276.0,289.0){\rule[-0.200pt]{0.964pt}{0.400pt}}
\put(281,290){\usebox{\plotpoint}}
\put(286,289.67){\rule{0.241pt}{0.400pt}}
\multiput(286.00,289.17)(0.500,1.000){2}{\rule{0.120pt}{0.400pt}}
\put(281.0,290.0){\rule[-0.200pt]{1.204pt}{0.400pt}}
\put(287,291){\usebox{\plotpoint}}
\put(291,290.67){\rule{0.241pt}{0.400pt}}
\multiput(291.00,290.17)(0.500,1.000){2}{\rule{0.120pt}{0.400pt}}
\put(287.0,291.0){\rule[-0.200pt]{0.964pt}{0.400pt}}
\put(297,291.67){\rule{0.241pt}{0.400pt}}
\multiput(297.00,291.17)(0.500,1.000){2}{\rule{0.120pt}{0.400pt}}
\put(292.0,292.0){\rule[-0.200pt]{1.204pt}{0.400pt}}
\put(298,293){\usebox{\plotpoint}}
\put(303,292.67){\rule{0.241pt}{0.400pt}}
\multiput(303.00,292.17)(0.500,1.000){2}{\rule{0.120pt}{0.400pt}}
\put(298.0,293.0){\rule[-0.200pt]{1.204pt}{0.400pt}}
\put(304,294){\usebox{\plotpoint}}
\put(309,293.67){\rule{0.241pt}{0.400pt}}
\multiput(309.00,293.17)(0.500,1.000){2}{\rule{0.120pt}{0.400pt}}
\put(304.0,294.0){\rule[-0.200pt]{1.204pt}{0.400pt}}
\put(310,295){\usebox{\plotpoint}}
\put(310.0,295.0){\rule[-0.200pt]{1.204pt}{0.400pt}}
\put(315.0,295.0){\usebox{\plotpoint}}
\put(315.0,296.0){\rule[-0.200pt]{1.445pt}{0.400pt}}
\put(321.0,296.0){\usebox{\plotpoint}}
\put(321.0,297.0){\rule[-0.200pt]{1.445pt}{0.400pt}}
\put(327.0,297.0){\usebox{\plotpoint}}
\put(332,297.67){\rule{0.241pt}{0.400pt}}
\multiput(332.00,297.17)(0.500,1.000){2}{\rule{0.120pt}{0.400pt}}
\put(327.0,298.0){\rule[-0.200pt]{1.204pt}{0.400pt}}
\put(339,298.67){\rule{0.241pt}{0.400pt}}
\multiput(339.00,298.17)(0.500,1.000){2}{\rule{0.120pt}{0.400pt}}
\put(333.0,299.0){\rule[-0.200pt]{1.445pt}{0.400pt}}
\put(340,300){\usebox{\plotpoint}}
\put(345,299.67){\rule{0.241pt}{0.400pt}}
\multiput(345.00,299.17)(0.500,1.000){2}{\rule{0.120pt}{0.400pt}}
\put(340.0,300.0){\rule[-0.200pt]{1.204pt}{0.400pt}}
\put(346,301){\usebox{\plotpoint}}
\put(351,300.67){\rule{0.241pt}{0.400pt}}
\multiput(351.00,300.17)(0.500,1.000){2}{\rule{0.120pt}{0.400pt}}
\put(346.0,301.0){\rule[-0.200pt]{1.204pt}{0.400pt}}
\put(357,301.67){\rule{0.241pt}{0.400pt}}
\multiput(357.00,301.17)(0.500,1.000){2}{\rule{0.120pt}{0.400pt}}
\put(352.0,302.0){\rule[-0.200pt]{1.204pt}{0.400pt}}
\put(358.0,303.0){\rule[-0.200pt]{1.445pt}{0.400pt}}
\put(364.0,303.0){\usebox{\plotpoint}}
\put(370,303.67){\rule{0.241pt}{0.400pt}}
\multiput(370.00,303.17)(0.500,1.000){2}{\rule{0.120pt}{0.400pt}}
\put(364.0,304.0){\rule[-0.200pt]{1.445pt}{0.400pt}}
\put(377,304.67){\rule{0.241pt}{0.400pt}}
\multiput(377.00,304.17)(0.500,1.000){2}{\rule{0.120pt}{0.400pt}}
\put(371.0,305.0){\rule[-0.200pt]{1.445pt}{0.400pt}}
\put(378,306){\usebox{\plotpoint}}
\put(384,305.67){\rule{0.241pt}{0.400pt}}
\multiput(384.00,305.17)(0.500,1.000){2}{\rule{0.120pt}{0.400pt}}
\put(378.0,306.0){\rule[-0.200pt]{1.445pt}{0.400pt}}
\put(385,307){\usebox{\plotpoint}}
\put(385.0,307.0){\rule[-0.200pt]{1.445pt}{0.400pt}}
\put(391.0,307.0){\usebox{\plotpoint}}
\put(391.0,308.0){\rule[-0.200pt]{1.686pt}{0.400pt}}
\put(398.0,308.0){\usebox{\plotpoint}}
\put(405,308.67){\rule{0.241pt}{0.400pt}}
\multiput(405.00,308.17)(0.500,1.000){2}{\rule{0.120pt}{0.400pt}}
\put(398.0,309.0){\rule[-0.200pt]{1.686pt}{0.400pt}}
\put(406,310){\usebox{\plotpoint}}
\put(412,309.67){\rule{0.241pt}{0.400pt}}
\multiput(412.00,309.17)(0.500,1.000){2}{\rule{0.120pt}{0.400pt}}
\put(406.0,310.0){\rule[-0.200pt]{1.445pt}{0.400pt}}
\put(420,310.67){\rule{0.241pt}{0.400pt}}
\multiput(420.00,310.17)(0.500,1.000){2}{\rule{0.120pt}{0.400pt}}
\put(413.0,311.0){\rule[-0.200pt]{1.686pt}{0.400pt}}
\put(421,312){\usebox{\plotpoint}}
\put(428,311.67){\rule{0.241pt}{0.400pt}}
\multiput(428.00,311.17)(0.500,1.000){2}{\rule{0.120pt}{0.400pt}}
\put(421.0,312.0){\rule[-0.200pt]{1.686pt}{0.400pt}}
\put(429,313){\usebox{\plotpoint}}
\put(429.0,313.0){\rule[-0.200pt]{1.686pt}{0.400pt}}
\put(436.0,313.0){\usebox{\plotpoint}}
\put(444,313.67){\rule{0.241pt}{0.400pt}}
\multiput(444.00,313.17)(0.500,1.000){2}{\rule{0.120pt}{0.400pt}}
\put(436.0,314.0){\rule[-0.200pt]{1.927pt}{0.400pt}}
\put(445.0,315.0){\rule[-0.200pt]{1.927pt}{0.400pt}}
\put(453.0,315.0){\usebox{\plotpoint}}
\put(461,315.67){\rule{0.241pt}{0.400pt}}
\multiput(461.00,315.17)(0.500,1.000){2}{\rule{0.120pt}{0.400pt}}
\put(453.0,316.0){\rule[-0.200pt]{1.927pt}{0.400pt}}
\put(470,316.67){\rule{0.241pt}{0.400pt}}
\multiput(470.00,316.17)(0.500,1.000){2}{\rule{0.120pt}{0.400pt}}
\put(462.0,317.0){\rule[-0.200pt]{1.927pt}{0.400pt}}
\put(480,317.67){\rule{0.241pt}{0.400pt}}
\multiput(480.00,317.17)(0.500,1.000){2}{\rule{0.120pt}{0.400pt}}
\put(471.0,318.0){\rule[-0.200pt]{2.168pt}{0.400pt}}
\put(481.0,319.0){\rule[-0.200pt]{2.409pt}{0.400pt}}
\put(491.0,319.0){\usebox{\plotpoint}}
\put(501,319.67){\rule{0.241pt}{0.400pt}}
\multiput(501.00,319.17)(0.500,1.000){2}{\rule{0.120pt}{0.400pt}}
\put(491.0,320.0){\rule[-0.200pt]{2.409pt}{0.400pt}}
\put(502,321){\usebox{\plotpoint}}
\put(513,320.67){\rule{0.241pt}{0.400pt}}
\multiput(513.00,320.17)(0.500,1.000){2}{\rule{0.120pt}{0.400pt}}
\put(502.0,321.0){\rule[-0.200pt]{2.650pt}{0.400pt}}
\put(514,322){\usebox{\plotpoint}}
\put(526,321.67){\rule{0.241pt}{0.400pt}}
\multiput(526.00,321.17)(0.500,1.000){2}{\rule{0.120pt}{0.400pt}}
\put(514.0,322.0){\rule[-0.200pt]{2.891pt}{0.400pt}}
\put(527,323){\usebox{\plotpoint}}
\put(527.0,323.0){\rule[-0.200pt]{3.132pt}{0.400pt}}
\put(540.0,323.0){\usebox{\plotpoint}}
\put(540.0,324.0){\rule[-0.200pt]{4.095pt}{0.400pt}}
\put(557.0,324.0){\usebox{\plotpoint}}
\put(577,324.67){\rule{0.241pt}{0.400pt}}
\multiput(577.00,324.17)(0.500,1.000){2}{\rule{0.120pt}{0.400pt}}
\put(557.0,325.0){\rule[-0.200pt]{4.818pt}{0.400pt}}
\put(578,326){\usebox{\plotpoint}}
\put(610,325.67){\rule{0.241pt}{0.400pt}}
\multiput(610.00,325.17)(0.500,1.000){2}{\rule{0.120pt}{0.400pt}}
\put(578.0,326.0){\rule[-0.200pt]{7.709pt}{0.400pt}}
\put(663,325.67){\rule{0.241pt}{0.400pt}}
\multiput(663.00,326.17)(0.500,-1.000){2}{\rule{0.120pt}{0.400pt}}
\put(611.0,327.0){\rule[-0.200pt]{12.527pt}{0.400pt}}
\put(695,324.67){\rule{0.241pt}{0.400pt}}
\multiput(695.00,325.17)(0.500,-1.000){2}{\rule{0.120pt}{0.400pt}}
\put(664.0,326.0){\rule[-0.200pt]{7.468pt}{0.400pt}}
\put(715,323.67){\rule{0.241pt}{0.400pt}}
\multiput(715.00,324.17)(0.500,-1.000){2}{\rule{0.120pt}{0.400pt}}
\put(696.0,325.0){\rule[-0.200pt]{4.577pt}{0.400pt}}
\put(716,324){\usebox{\plotpoint}}
\put(716.0,324.0){\rule[-0.200pt]{3.613pt}{0.400pt}}
\put(731.0,323.0){\usebox{\plotpoint}}
\put(731.0,323.0){\rule[-0.200pt]{3.132pt}{0.400pt}}
\put(744.0,322.0){\usebox{\plotpoint}}
\put(744.0,322.0){\rule[-0.200pt]{2.891pt}{0.400pt}}
\put(756.0,321.0){\usebox{\plotpoint}}
\put(756.0,321.0){\rule[-0.200pt]{2.650pt}{0.400pt}}
\put(767.0,320.0){\usebox{\plotpoint}}
\put(776,318.67){\rule{0.241pt}{0.400pt}}
\multiput(776.00,319.17)(0.500,-1.000){2}{\rule{0.120pt}{0.400pt}}
\put(767.0,320.0){\rule[-0.200pt]{2.168pt}{0.400pt}}
\put(777.0,319.0){\rule[-0.200pt]{2.168pt}{0.400pt}}
\put(786.0,318.0){\usebox{\plotpoint}}
\put(794,316.67){\rule{0.241pt}{0.400pt}}
\multiput(794.00,317.17)(0.500,-1.000){2}{\rule{0.120pt}{0.400pt}}
\put(786.0,318.0){\rule[-0.200pt]{1.927pt}{0.400pt}}
\put(795,317){\usebox{\plotpoint}}
\put(802,315.67){\rule{0.241pt}{0.400pt}}
\multiput(802.00,316.17)(0.500,-1.000){2}{\rule{0.120pt}{0.400pt}}
\put(795.0,317.0){\rule[-0.200pt]{1.686pt}{0.400pt}}
\put(803,316){\usebox{\plotpoint}}
\put(810,314.67){\rule{0.241pt}{0.400pt}}
\multiput(810.00,315.17)(0.500,-1.000){2}{\rule{0.120pt}{0.400pt}}
\put(803.0,316.0){\rule[-0.200pt]{1.686pt}{0.400pt}}
\put(811,315){\usebox{\plotpoint}}
\put(811.0,315.0){\rule[-0.200pt]{1.686pt}{0.400pt}}
\put(818.0,314.0){\usebox{\plotpoint}}
\put(825,312.67){\rule{0.241pt}{0.400pt}}
\multiput(825.00,313.17)(0.500,-1.000){2}{\rule{0.120pt}{0.400pt}}
\put(818.0,314.0){\rule[-0.200pt]{1.686pt}{0.400pt}}
\put(826,313){\usebox{\plotpoint}}
\put(831,311.67){\rule{0.241pt}{0.400pt}}
\multiput(831.00,312.17)(0.500,-1.000){2}{\rule{0.120pt}{0.400pt}}
\put(826.0,313.0){\rule[-0.200pt]{1.204pt}{0.400pt}}
\put(832.0,312.0){\rule[-0.200pt]{1.686pt}{0.400pt}}
\put(839.0,311.0){\usebox{\plotpoint}}
\put(839.0,311.0){\rule[-0.200pt]{1.445pt}{0.400pt}}
\put(845.0,310.0){\usebox{\plotpoint}}
\put(851,308.67){\rule{0.241pt}{0.400pt}}
\multiput(851.00,309.17)(0.500,-1.000){2}{\rule{0.120pt}{0.400pt}}
\put(845.0,310.0){\rule[-0.200pt]{1.445pt}{0.400pt}}
\put(852,309){\usebox{\plotpoint}}
\put(857,307.67){\rule{0.241pt}{0.400pt}}
\multiput(857.00,308.17)(0.500,-1.000){2}{\rule{0.120pt}{0.400pt}}
\put(852.0,309.0){\rule[-0.200pt]{1.204pt}{0.400pt}}
\put(858,308){\usebox{\plotpoint}}
\put(863,306.67){\rule{0.241pt}{0.400pt}}
\multiput(863.00,307.17)(0.500,-1.000){2}{\rule{0.120pt}{0.400pt}}
\put(858.0,308.0){\rule[-0.200pt]{1.204pt}{0.400pt}}
\put(864.0,307.0){\rule[-0.200pt]{1.204pt}{0.400pt}}
\put(869.0,306.0){\usebox{\plotpoint}}
\put(869.0,306.0){\rule[-0.200pt]{1.445pt}{0.400pt}}
\put(875.0,305.0){\usebox{\plotpoint}}
\put(880,303.67){\rule{0.241pt}{0.400pt}}
\multiput(880.00,304.17)(0.500,-1.000){2}{\rule{0.120pt}{0.400pt}}
\put(875.0,305.0){\rule[-0.200pt]{1.204pt}{0.400pt}}
\put(881.0,304.0){\rule[-0.200pt]{1.204pt}{0.400pt}}
\put(886.0,303.0){\usebox{\plotpoint}}
\put(891,301.67){\rule{0.241pt}{0.400pt}}
\multiput(891.00,302.17)(0.500,-1.000){2}{\rule{0.120pt}{0.400pt}}
\put(886.0,303.0){\rule[-0.200pt]{1.204pt}{0.400pt}}
\put(892,302){\usebox{\plotpoint}}
\put(896,300.67){\rule{0.241pt}{0.400pt}}
\multiput(896.00,301.17)(0.500,-1.000){2}{\rule{0.120pt}{0.400pt}}
\put(892.0,302.0){\rule[-0.200pt]{0.964pt}{0.400pt}}
\put(897,301){\usebox{\plotpoint}}
\put(901,299.67){\rule{0.241pt}{0.400pt}}
\multiput(901.00,300.17)(0.500,-1.000){2}{\rule{0.120pt}{0.400pt}}
\put(897.0,301.0){\rule[-0.200pt]{0.964pt}{0.400pt}}
\put(906,298.67){\rule{0.241pt}{0.400pt}}
\multiput(906.00,299.17)(0.500,-1.000){2}{\rule{0.120pt}{0.400pt}}
\put(902.0,300.0){\rule[-0.200pt]{0.964pt}{0.400pt}}
\put(907,299){\usebox{\plotpoint}}
\put(911,297.67){\rule{0.241pt}{0.400pt}}
\multiput(911.00,298.17)(0.500,-1.000){2}{\rule{0.120pt}{0.400pt}}
\put(907.0,299.0){\rule[-0.200pt]{0.964pt}{0.400pt}}
\put(912.0,298.0){\rule[-0.200pt]{0.964pt}{0.400pt}}
\put(916.0,297.0){\usebox{\plotpoint}}
\put(921,295.67){\rule{0.241pt}{0.400pt}}
\multiput(921.00,296.17)(0.500,-1.000){2}{\rule{0.120pt}{0.400pt}}
\put(916.0,297.0){\rule[-0.200pt]{1.204pt}{0.400pt}}
\put(922,296){\usebox{\plotpoint}}
\put(922.0,296.0){\rule[-0.200pt]{0.964pt}{0.400pt}}
\put(926.0,295.0){\usebox{\plotpoint}}
\put(930,293.67){\rule{0.241pt}{0.400pt}}
\multiput(930.00,294.17)(0.500,-1.000){2}{\rule{0.120pt}{0.400pt}}
\put(926.0,295.0){\rule[-0.200pt]{0.964pt}{0.400pt}}
\put(931,294){\usebox{\plotpoint}}
\put(931.0,294.0){\rule[-0.200pt]{0.964pt}{0.400pt}}
\put(935.0,293.0){\usebox{\plotpoint}}
\put(939,291.67){\rule{0.241pt}{0.400pt}}
\multiput(939.00,292.17)(0.500,-1.000){2}{\rule{0.120pt}{0.400pt}}
\put(935.0,293.0){\rule[-0.200pt]{0.964pt}{0.400pt}}
\put(944,290.67){\rule{0.241pt}{0.400pt}}
\multiput(944.00,291.17)(0.500,-1.000){2}{\rule{0.120pt}{0.400pt}}
\put(940.0,292.0){\rule[-0.200pt]{0.964pt}{0.400pt}}
\put(945,291){\usebox{\plotpoint}}
\put(945.0,291.0){\rule[-0.200pt]{0.964pt}{0.400pt}}
\put(949.0,290.0){\usebox{\plotpoint}}
\put(952,288.67){\rule{0.241pt}{0.400pt}}
\multiput(952.00,289.17)(0.500,-1.000){2}{\rule{0.120pt}{0.400pt}}
\put(949.0,290.0){\rule[-0.200pt]{0.723pt}{0.400pt}}
\put(957,287.67){\rule{0.241pt}{0.400pt}}
\multiput(957.00,288.17)(0.500,-1.000){2}{\rule{0.120pt}{0.400pt}}
\put(953.0,289.0){\rule[-0.200pt]{0.964pt}{0.400pt}}
\put(958,288){\usebox{\plotpoint}}
\put(961,286.67){\rule{0.241pt}{0.400pt}}
\multiput(961.00,287.17)(0.500,-1.000){2}{\rule{0.120pt}{0.400pt}}
\put(958.0,288.0){\rule[-0.200pt]{0.723pt}{0.400pt}}
\put(962,287){\usebox{\plotpoint}}
\put(962.0,287.0){\rule[-0.200pt]{0.964pt}{0.400pt}}
\put(966.0,286.0){\usebox{\plotpoint}}
\put(969,284.67){\rule{0.241pt}{0.400pt}}
\multiput(969.00,285.17)(0.500,-1.000){2}{\rule{0.120pt}{0.400pt}}
\put(966.0,286.0){\rule[-0.200pt]{0.723pt}{0.400pt}}
\put(974,283.67){\rule{0.241pt}{0.400pt}}
\multiput(974.00,284.17)(0.500,-1.000){2}{\rule{0.120pt}{0.400pt}}
\put(970.0,285.0){\rule[-0.200pt]{0.964pt}{0.400pt}}
\put(975,284){\usebox{\plotpoint}}
\put(978,282.67){\rule{0.241pt}{0.400pt}}
\multiput(978.00,283.17)(0.500,-1.000){2}{\rule{0.120pt}{0.400pt}}
\put(975.0,284.0){\rule[-0.200pt]{0.723pt}{0.400pt}}
\put(979,283){\usebox{\plotpoint}}
\put(982,281.67){\rule{0.241pt}{0.400pt}}
\multiput(982.00,282.17)(0.500,-1.000){2}{\rule{0.120pt}{0.400pt}}
\put(979.0,283.0){\rule[-0.200pt]{0.723pt}{0.400pt}}
\put(983,282){\usebox{\plotpoint}}
\put(983.0,282.0){\rule[-0.200pt]{0.723pt}{0.400pt}}
\put(986.0,281.0){\usebox{\plotpoint}}
\put(986.0,281.0){\rule[-0.200pt]{0.964pt}{0.400pt}}
\put(990.0,280.0){\usebox{\plotpoint}}
\put(990.0,280.0){\rule[-0.200pt]{0.964pt}{0.400pt}}
\put(994.0,279.0){\usebox{\plotpoint}}
\put(994.0,279.0){\rule[-0.200pt]{0.964pt}{0.400pt}}
\put(998.0,278.0){\usebox{\plotpoint}}
\put(1001,276.67){\rule{0.241pt}{0.400pt}}
\multiput(1001.00,277.17)(0.500,-1.000){2}{\rule{0.120pt}{0.400pt}}
\put(998.0,278.0){\rule[-0.200pt]{0.723pt}{0.400pt}}
\put(1005,275.67){\rule{0.241pt}{0.400pt}}
\multiput(1005.00,276.17)(0.500,-1.000){2}{\rule{0.120pt}{0.400pt}}
\put(1002.0,277.0){\rule[-0.200pt]{0.723pt}{0.400pt}}
\put(1009,274.67){\rule{0.241pt}{0.400pt}}
\multiput(1009.00,275.17)(0.500,-1.000){2}{\rule{0.120pt}{0.400pt}}
\put(1006.0,276.0){\rule[-0.200pt]{0.723pt}{0.400pt}}
\put(1013,273.67){\rule{0.241pt}{0.400pt}}
\multiput(1013.00,274.17)(0.500,-1.000){2}{\rule{0.120pt}{0.400pt}}
\put(1010.0,275.0){\rule[-0.200pt]{0.723pt}{0.400pt}}
\put(1014.0,274.0){\rule[-0.200pt]{0.723pt}{0.400pt}}
\put(1017.0,273.0){\usebox{\plotpoint}}
\put(1020,271.67){\rule{0.241pt}{0.400pt}}
\multiput(1020.00,272.17)(0.500,-1.000){2}{\rule{0.120pt}{0.400pt}}
\put(1017.0,273.0){\rule[-0.200pt]{0.723pt}{0.400pt}}
\put(1024,270.67){\rule{0.241pt}{0.400pt}}
\multiput(1024.00,271.17)(0.500,-1.000){2}{\rule{0.120pt}{0.400pt}}
\put(1021.0,272.0){\rule[-0.200pt]{0.723pt}{0.400pt}}
\put(1028,269.67){\rule{0.241pt}{0.400pt}}
\multiput(1028.00,270.17)(0.500,-1.000){2}{\rule{0.120pt}{0.400pt}}
\put(1025.0,271.0){\rule[-0.200pt]{0.723pt}{0.400pt}}
\put(1029.0,270.0){\rule[-0.200pt]{0.723pt}{0.400pt}}
\put(1032.0,269.0){\usebox{\plotpoint}}
\put(1035,267.67){\rule{0.241pt}{0.400pt}}
\multiput(1035.00,268.17)(0.500,-1.000){2}{\rule{0.120pt}{0.400pt}}
\put(1032.0,269.0){\rule[-0.200pt]{0.723pt}{0.400pt}}
\put(1036.0,268.0){\rule[-0.200pt]{0.723pt}{0.400pt}}
\put(1039.0,267.0){\usebox{\plotpoint}}
\put(1039.0,267.0){\rule[-0.200pt]{0.964pt}{0.400pt}}
\put(1043.0,266.0){\usebox{\plotpoint}}
\put(1046,264.67){\rule{0.241pt}{0.400pt}}
\multiput(1046.00,265.17)(0.500,-1.000){2}{\rule{0.120pt}{0.400pt}}
\put(1043.0,266.0){\rule[-0.200pt]{0.723pt}{0.400pt}}
\put(1047,265){\usebox{\plotpoint}}
\put(1050,263.67){\rule{0.241pt}{0.400pt}}
\multiput(1050.00,264.17)(0.500,-1.000){2}{\rule{0.120pt}{0.400pt}}
\put(1047.0,265.0){\rule[-0.200pt]{0.723pt}{0.400pt}}
\put(1051,264){\usebox{\plotpoint}}
\put(1053,262.67){\rule{0.241pt}{0.400pt}}
\multiput(1053.00,263.17)(0.500,-1.000){2}{\rule{0.120pt}{0.400pt}}
\put(1051.0,264.0){\rule[-0.200pt]{0.482pt}{0.400pt}}
\put(1054,263){\usebox{\plotpoint}}
\put(1057,261.67){\rule{0.241pt}{0.400pt}}
\multiput(1057.00,262.17)(0.500,-1.000){2}{\rule{0.120pt}{0.400pt}}
\put(1054.0,263.0){\rule[-0.200pt]{0.723pt}{0.400pt}}
\put(1058,262){\usebox{\plotpoint}}
\put(1060,260.67){\rule{0.241pt}{0.400pt}}
\multiput(1060.00,261.17)(0.500,-1.000){2}{\rule{0.120pt}{0.400pt}}
\put(1058.0,262.0){\rule[-0.200pt]{0.482pt}{0.400pt}}
\put(1061.0,261.0){\rule[-0.200pt]{0.723pt}{0.400pt}}
\put(1064.0,260.0){\usebox{\plotpoint}}
\put(1064.0,260.0){\rule[-0.200pt]{0.964pt}{0.400pt}}
\put(1068.0,259.0){\usebox{\plotpoint}}
\put(1068.0,259.0){\rule[-0.200pt]{0.723pt}{0.400pt}}
\put(1071.0,258.0){\usebox{\plotpoint}}
\put(1074,256.67){\rule{0.241pt}{0.400pt}}
\multiput(1074.00,257.17)(0.500,-1.000){2}{\rule{0.120pt}{0.400pt}}
\put(1071.0,258.0){\rule[-0.200pt]{0.723pt}{0.400pt}}
\put(1075,257){\usebox{\plotpoint}}
\put(1077,255.67){\rule{0.241pt}{0.400pt}}
\multiput(1077.00,256.17)(0.500,-1.000){2}{\rule{0.120pt}{0.400pt}}
\put(1075.0,257.0){\rule[-0.200pt]{0.482pt}{0.400pt}}
\put(1081,254.67){\rule{0.241pt}{0.400pt}}
\multiput(1081.00,255.17)(0.500,-1.000){2}{\rule{0.120pt}{0.400pt}}
\put(1078.0,256.0){\rule[-0.200pt]{0.723pt}{0.400pt}}
\put(1082.0,255.0){\rule[-0.200pt]{0.723pt}{0.400pt}}
\put(1085.0,254.0){\usebox{\plotpoint}}
\put(1085.0,254.0){\rule[-0.200pt]{0.723pt}{0.400pt}}
\put(1088.0,253.0){\usebox{\plotpoint}}
\put(1091,251.67){\rule{0.241pt}{0.400pt}}
\multiput(1091.00,252.17)(0.500,-1.000){2}{\rule{0.120pt}{0.400pt}}
\put(1088.0,253.0){\rule[-0.200pt]{0.723pt}{0.400pt}}
\put(1092,252){\usebox{\plotpoint}}
\put(1094,250.67){\rule{0.241pt}{0.400pt}}
\multiput(1094.00,251.17)(0.500,-1.000){2}{\rule{0.120pt}{0.400pt}}
\put(1092.0,252.0){\rule[-0.200pt]{0.482pt}{0.400pt}}
\put(1095.0,251.0){\rule[-0.200pt]{0.723pt}{0.400pt}}
\put(1098.0,250.0){\usebox{\plotpoint}}
\put(1098.0,250.0){\rule[-0.200pt]{0.964pt}{0.400pt}}
\put(1102.0,249.0){\usebox{\plotpoint}}
\put(1102.0,249.0){\rule[-0.200pt]{0.723pt}{0.400pt}}
\put(1105.0,248.0){\usebox{\plotpoint}}
\put(1108,246.67){\rule{0.241pt}{0.400pt}}
\multiput(1108.00,247.17)(0.500,-1.000){2}{\rule{0.120pt}{0.400pt}}
\put(1105.0,248.0){\rule[-0.200pt]{0.723pt}{0.400pt}}
\put(1109,247){\usebox{\plotpoint}}
\put(1111,245.67){\rule{0.241pt}{0.400pt}}
\multiput(1111.00,246.17)(0.500,-1.000){2}{\rule{0.120pt}{0.400pt}}
\put(1109.0,247.0){\rule[-0.200pt]{0.482pt}{0.400pt}}
\put(1112.0,246.0){\rule[-0.200pt]{0.723pt}{0.400pt}}
\put(1115.0,245.0){\usebox{\plotpoint}}
\put(1118,243.67){\rule{0.241pt}{0.400pt}}
\multiput(1118.00,244.17)(0.500,-1.000){2}{\rule{0.120pt}{0.400pt}}
\put(1115.0,245.0){\rule[-0.200pt]{0.723pt}{0.400pt}}
\put(1119,244){\usebox{\plotpoint}}
\put(1121,242.67){\rule{0.241pt}{0.400pt}}
\multiput(1121.00,243.17)(0.500,-1.000){2}{\rule{0.120pt}{0.400pt}}
\put(1119.0,244.0){\rule[-0.200pt]{0.482pt}{0.400pt}}
\put(1122,243){\usebox{\plotpoint}}
\put(1124,241.67){\rule{0.241pt}{0.400pt}}
\multiput(1124.00,242.17)(0.500,-1.000){2}{\rule{0.120pt}{0.400pt}}
\put(1122.0,243.0){\rule[-0.200pt]{0.482pt}{0.400pt}}
\put(1125.0,242.0){\rule[-0.200pt]{0.723pt}{0.400pt}}
\put(1128.0,241.0){\usebox{\plotpoint}}
\put(1131,239.67){\rule{0.241pt}{0.400pt}}
\multiput(1131.00,240.17)(0.500,-1.000){2}{\rule{0.120pt}{0.400pt}}
\put(1128.0,241.0){\rule[-0.200pt]{0.723pt}{0.400pt}}
\put(1132,240){\usebox{\plotpoint}}
\put(1134,238.67){\rule{0.241pt}{0.400pt}}
\multiput(1134.00,239.17)(0.500,-1.000){2}{\rule{0.120pt}{0.400pt}}
\put(1132.0,240.0){\rule[-0.200pt]{0.482pt}{0.400pt}}
\put(1135.0,239.0){\rule[-0.200pt]{0.723pt}{0.400pt}}
\put(1138.0,238.0){\usebox{\plotpoint}}
\put(1138.0,238.0){\rule[-0.200pt]{0.723pt}{0.400pt}}
\put(1141.0,237.0){\usebox{\plotpoint}}
\put(1144,235.67){\rule{0.241pt}{0.400pt}}
\multiput(1144.00,236.17)(0.500,-1.000){2}{\rule{0.120pt}{0.400pt}}
\put(1141.0,237.0){\rule[-0.200pt]{0.723pt}{0.400pt}}
\put(1145,236){\usebox{\plotpoint}}
\put(1147,234.67){\rule{0.241pt}{0.400pt}}
\multiput(1147.00,235.17)(0.500,-1.000){2}{\rule{0.120pt}{0.400pt}}
\put(1145.0,236.0){\rule[-0.200pt]{0.482pt}{0.400pt}}
\put(1148.0,235.0){\rule[-0.200pt]{0.723pt}{0.400pt}}
\put(1151.0,234.0){\usebox{\plotpoint}}
\put(1154,232.67){\rule{0.241pt}{0.400pt}}
\multiput(1154.00,233.17)(0.500,-1.000){2}{\rule{0.120pt}{0.400pt}}
\put(1151.0,234.0){\rule[-0.200pt]{0.723pt}{0.400pt}}
\put(1155,233){\usebox{\plotpoint}}
\put(1156,231.67){\rule{0.241pt}{0.400pt}}
\multiput(1156.00,232.17)(0.500,-1.000){2}{\rule{0.120pt}{0.400pt}}
\put(1155.0,233.0){\usebox{\plotpoint}}
\put(1157.0,232.0){\rule[-0.200pt]{0.723pt}{0.400pt}}
\put(1160.0,231.0){\usebox{\plotpoint}}
\put(1163,229.67){\rule{0.241pt}{0.400pt}}
\multiput(1163.00,230.17)(0.500,-1.000){2}{\rule{0.120pt}{0.400pt}}
\put(1160.0,231.0){\rule[-0.200pt]{0.723pt}{0.400pt}}
\put(1164,230){\usebox{\plotpoint}}
\put(1166,228.67){\rule{0.241pt}{0.400pt}}
\multiput(1166.00,229.17)(0.500,-1.000){2}{\rule{0.120pt}{0.400pt}}
\put(1164.0,230.0){\rule[-0.200pt]{0.482pt}{0.400pt}}
\put(1167.0,229.0){\rule[-0.200pt]{0.723pt}{0.400pt}}
\put(1170.0,228.0){\usebox{\plotpoint}}
\put(1170.0,228.0){\rule[-0.200pt]{0.723pt}{0.400pt}}
\put(1173.0,227.0){\usebox{\plotpoint}}
\put(1176,225.67){\rule{0.241pt}{0.400pt}}
\multiput(1176.00,226.17)(0.500,-1.000){2}{\rule{0.120pt}{0.400pt}}
\put(1173.0,227.0){\rule[-0.200pt]{0.723pt}{0.400pt}}
\put(1177,226){\usebox{\plotpoint}}
\put(1179,224.67){\rule{0.241pt}{0.400pt}}
\multiput(1179.00,225.17)(0.500,-1.000){2}{\rule{0.120pt}{0.400pt}}
\put(1177.0,226.0){\rule[-0.200pt]{0.482pt}{0.400pt}}
\put(1180.0,225.0){\rule[-0.200pt]{0.723pt}{0.400pt}}
\put(1183.0,224.0){\usebox{\plotpoint}}
\put(1185,222.67){\rule{0.241pt}{0.400pt}}
\multiput(1185.00,223.17)(0.500,-1.000){2}{\rule{0.120pt}{0.400pt}}
\put(1183.0,224.0){\rule[-0.200pt]{0.482pt}{0.400pt}}
\put(1186.0,223.0){\rule[-0.200pt]{0.723pt}{0.400pt}}
\put(1189.0,222.0){\usebox{\plotpoint}}
\put(1189.0,222.0){\rule[-0.200pt]{0.723pt}{0.400pt}}
\put(1192.0,221.0){\usebox{\plotpoint}}
\put(1195,219.67){\rule{0.241pt}{0.400pt}}
\multiput(1195.00,220.17)(0.500,-1.000){2}{\rule{0.120pt}{0.400pt}}
\put(1192.0,221.0){\rule[-0.200pt]{0.723pt}{0.400pt}}
\put(1196,220){\usebox{\plotpoint}}
\put(1198,218.67){\rule{0.241pt}{0.400pt}}
\multiput(1198.00,219.17)(0.500,-1.000){2}{\rule{0.120pt}{0.400pt}}
\put(1196.0,220.0){\rule[-0.200pt]{0.482pt}{0.400pt}}
\put(1199.0,219.0){\rule[-0.200pt]{0.723pt}{0.400pt}}
\put(1202.0,218.0){\usebox{\plotpoint}}
\put(1205,216.67){\rule{0.241pt}{0.400pt}}
\multiput(1205.00,217.17)(0.500,-1.000){2}{\rule{0.120pt}{0.400pt}}
\put(1202.0,218.0){\rule[-0.200pt]{0.723pt}{0.400pt}}
\put(1206,217){\usebox{\plotpoint}}
\put(1208,215.67){\rule{0.241pt}{0.400pt}}
\multiput(1208.00,216.17)(0.500,-1.000){2}{\rule{0.120pt}{0.400pt}}
\put(1206.0,217.0){\rule[-0.200pt]{0.482pt}{0.400pt}}
\put(1209,216){\usebox{\plotpoint}}
\put(1209.0,216.0){\rule[-0.200pt]{0.482pt}{0.400pt}}
\put(1211.0,215.0){\usebox{\plotpoint}}
\put(1214,213.67){\rule{0.241pt}{0.400pt}}
\multiput(1214.00,214.17)(0.500,-1.000){2}{\rule{0.120pt}{0.400pt}}
\put(1211.0,215.0){\rule[-0.200pt]{0.723pt}{0.400pt}}
\put(1215,214){\usebox{\plotpoint}}
\put(1217,212.67){\rule{0.241pt}{0.400pt}}
\multiput(1217.00,213.17)(0.500,-1.000){2}{\rule{0.120pt}{0.400pt}}
\put(1215.0,214.0){\rule[-0.200pt]{0.482pt}{0.400pt}}
\put(1218.0,213.0){\rule[-0.200pt]{0.723pt}{0.400pt}}
\put(1221.0,212.0){\usebox{\plotpoint}}
\put(1221.0,212.0){\rule[-0.200pt]{0.723pt}{0.400pt}}
\put(1224.0,211.0){\usebox{\plotpoint}}
\put(1227,209.67){\rule{0.241pt}{0.400pt}}
\multiput(1227.00,210.17)(0.500,-1.000){2}{\rule{0.120pt}{0.400pt}}
\put(1224.0,211.0){\rule[-0.200pt]{0.723pt}{0.400pt}}
\put(1228,210){\usebox{\plotpoint}}
\put(1230,208.67){\rule{0.241pt}{0.400pt}}
\multiput(1230.00,209.17)(0.500,-1.000){2}{\rule{0.120pt}{0.400pt}}
\put(1228.0,210.0){\rule[-0.200pt]{0.482pt}{0.400pt}}
\put(1231.0,209.0){\rule[-0.200pt]{0.723pt}{0.400pt}}
\put(1234.0,208.0){\usebox{\plotpoint}}
\put(1236,206.67){\rule{0.241pt}{0.400pt}}
\multiput(1236.00,207.17)(0.500,-1.000){2}{\rule{0.120pt}{0.400pt}}
\put(1234.0,208.0){\rule[-0.200pt]{0.482pt}{0.400pt}}
\put(1237.0,207.0){\rule[-0.200pt]{0.723pt}{0.400pt}}
\put(1240.0,206.0){\usebox{\plotpoint}}
\put(1240.0,206.0){\rule[-0.200pt]{0.723pt}{0.400pt}}
\put(1243.0,205.0){\usebox{\plotpoint}}
\put(1246,203.67){\rule{0.241pt}{0.400pt}}
\multiput(1246.00,204.17)(0.500,-1.000){2}{\rule{0.120pt}{0.400pt}}
\put(1243.0,205.0){\rule[-0.200pt]{0.723pt}{0.400pt}}
\put(1247,204){\usebox{\plotpoint}}
\put(1249,202.67){\rule{0.241pt}{0.400pt}}
\multiput(1249.00,203.17)(0.500,-1.000){2}{\rule{0.120pt}{0.400pt}}
\put(1247.0,204.0){\rule[-0.200pt]{0.482pt}{0.400pt}}
\put(1250.0,203.0){\rule[-0.200pt]{0.723pt}{0.400pt}}
\put(1253.0,202.0){\usebox{\plotpoint}}
\put(1256,200.67){\rule{0.241pt}{0.400pt}}
\multiput(1256.00,201.17)(0.500,-1.000){2}{\rule{0.120pt}{0.400pt}}
\put(1253.0,202.0){\rule[-0.200pt]{0.723pt}{0.400pt}}
\put(1257,201){\usebox{\plotpoint}}
\put(1259,199.67){\rule{0.241pt}{0.400pt}}
\multiput(1259.00,200.17)(0.500,-1.000){2}{\rule{0.120pt}{0.400pt}}
\put(1257.0,201.0){\rule[-0.200pt]{0.482pt}{0.400pt}}
\put(1260,200){\usebox{\plotpoint}}
\put(1262,198.67){\rule{0.241pt}{0.400pt}}
\multiput(1262.00,199.17)(0.500,-1.000){2}{\rule{0.120pt}{0.400pt}}
\put(1260.0,200.0){\rule[-0.200pt]{0.482pt}{0.400pt}}
\put(1263.0,199.0){\rule[-0.200pt]{0.723pt}{0.400pt}}
\put(1266.0,198.0){\usebox{\plotpoint}}
\put(1269,196.67){\rule{0.241pt}{0.400pt}}
\multiput(1269.00,197.17)(0.500,-1.000){2}{\rule{0.120pt}{0.400pt}}
\put(1266.0,198.0){\rule[-0.200pt]{0.723pt}{0.400pt}}
\put(1270,197){\usebox{\plotpoint}}
\put(1272,195.67){\rule{0.241pt}{0.400pt}}
\multiput(1272.00,196.17)(0.500,-1.000){2}{\rule{0.120pt}{0.400pt}}
\put(1270.0,197.0){\rule[-0.200pt]{0.482pt}{0.400pt}}
\put(1273.0,196.0){\rule[-0.200pt]{0.723pt}{0.400pt}}
\put(1276.0,195.0){\usebox{\plotpoint}}
\put(1276.0,195.0){\rule[-0.200pt]{0.723pt}{0.400pt}}
\put(1279.0,194.0){\usebox{\plotpoint}}
\put(1282,192.67){\rule{0.241pt}{0.400pt}}
\multiput(1282.00,193.17)(0.500,-1.000){2}{\rule{0.120pt}{0.400pt}}
\put(1279.0,194.0){\rule[-0.200pt]{0.723pt}{0.400pt}}
\put(1283,193){\usebox{\plotpoint}}
\put(1285,191.67){\rule{0.241pt}{0.400pt}}
\multiput(1285.00,192.17)(0.500,-1.000){2}{\rule{0.120pt}{0.400pt}}
\put(1283.0,193.0){\rule[-0.200pt]{0.482pt}{0.400pt}}
\put(1286.0,192.0){\rule[-0.200pt]{0.723pt}{0.400pt}}
\put(1289.0,191.0){\usebox{\plotpoint}}
\put(1292,189.67){\rule{0.241pt}{0.400pt}}
\multiput(1292.00,190.17)(0.500,-1.000){2}{\rule{0.120pt}{0.400pt}}
\put(1289.0,191.0){\rule[-0.200pt]{0.723pt}{0.400pt}}
\put(1293,190){\usebox{\plotpoint}}
\put(1295,188.67){\rule{0.241pt}{0.400pt}}
\multiput(1295.00,189.17)(0.500,-1.000){2}{\rule{0.120pt}{0.400pt}}
\put(1293.0,190.0){\rule[-0.200pt]{0.482pt}{0.400pt}}
\put(1296,189){\usebox{\plotpoint}}
\put(1298,187.67){\rule{0.241pt}{0.400pt}}
\multiput(1298.00,188.17)(0.500,-1.000){2}{\rule{0.120pt}{0.400pt}}
\put(1296.0,189.0){\rule[-0.200pt]{0.482pt}{0.400pt}}
\put(1299.0,188.0){\rule[-0.200pt]{0.723pt}{0.400pt}}
\put(1302.0,187.0){\usebox{\plotpoint}}
\put(1305,185.67){\rule{0.241pt}{0.400pt}}
\multiput(1305.00,186.17)(0.500,-1.000){2}{\rule{0.120pt}{0.400pt}}
\put(1302.0,187.0){\rule[-0.200pt]{0.723pt}{0.400pt}}
\put(1306,186){\usebox{\plotpoint}}
\put(1309,184.67){\rule{0.241pt}{0.400pt}}
\multiput(1309.00,185.17)(0.500,-1.000){2}{\rule{0.120pt}{0.400pt}}
\put(1306.0,186.0){\rule[-0.200pt]{0.723pt}{0.400pt}}
\put(1310,185){\usebox{\plotpoint}}
\put(1312,183.67){\rule{0.241pt}{0.400pt}}
\multiput(1312.00,184.17)(0.500,-1.000){2}{\rule{0.120pt}{0.400pt}}
\put(1310.0,185.0){\rule[-0.200pt]{0.482pt}{0.400pt}}
\put(1313,184){\usebox{\plotpoint}}
\put(1315,182.67){\rule{0.241pt}{0.400pt}}
\multiput(1315.00,183.17)(0.500,-1.000){2}{\rule{0.120pt}{0.400pt}}
\put(1313.0,184.0){\rule[-0.200pt]{0.482pt}{0.400pt}}
\put(1316.0,183.0){\rule[-0.200pt]{0.723pt}{0.400pt}}
\put(1319.0,182.0){\usebox{\plotpoint}}
\put(1322,180.67){\rule{0.241pt}{0.400pt}}
\multiput(1322.00,181.17)(0.500,-1.000){2}{\rule{0.120pt}{0.400pt}}
\put(1319.0,182.0){\rule[-0.200pt]{0.723pt}{0.400pt}}
\put(1323,181){\usebox{\plotpoint}}
\put(1326,179.67){\rule{0.241pt}{0.400pt}}
\multiput(1326.00,180.17)(0.500,-1.000){2}{\rule{0.120pt}{0.400pt}}
\put(1323.0,181.0){\rule[-0.200pt]{0.723pt}{0.400pt}}
\put(1327,180){\usebox{\plotpoint}}
\put(1329,178.67){\rule{0.241pt}{0.400pt}}
\multiput(1329.00,179.17)(0.500,-1.000){2}{\rule{0.120pt}{0.400pt}}
\put(1327.0,180.0){\rule[-0.200pt]{0.482pt}{0.400pt}}
\put(1330,179){\usebox{\plotpoint}}
\put(1332,177.67){\rule{0.241pt}{0.400pt}}
\multiput(1332.00,178.17)(0.500,-1.000){2}{\rule{0.120pt}{0.400pt}}
\put(1330.0,179.0){\rule[-0.200pt]{0.482pt}{0.400pt}}
\put(1336,176.67){\rule{0.241pt}{0.400pt}}
\multiput(1336.00,177.17)(0.500,-1.000){2}{\rule{0.120pt}{0.400pt}}
\put(1333.0,178.0){\rule[-0.200pt]{0.723pt}{0.400pt}}
\put(1337.0,177.0){\rule[-0.200pt]{0.723pt}{0.400pt}}
\put(1340.0,176.0){\usebox{\plotpoint}}
\put(1343,174.67){\rule{0.241pt}{0.400pt}}
\multiput(1343.00,175.17)(0.500,-1.000){2}{\rule{0.120pt}{0.400pt}}
\put(1340.0,176.0){\rule[-0.200pt]{0.723pt}{0.400pt}}
\put(1344,175){\usebox{\plotpoint}}
\put(1344.0,175.0){\rule[-0.200pt]{0.723pt}{0.400pt}}
\put(1347.0,174.0){\usebox{\plotpoint}}
\put(1350,172.67){\rule{0.241pt}{0.400pt}}
\multiput(1350.00,173.17)(0.500,-1.000){2}{\rule{0.120pt}{0.400pt}}
\put(1347.0,174.0){\rule[-0.200pt]{0.723pt}{0.400pt}}
\put(1351,173){\usebox{\plotpoint}}
\put(1354,171.67){\rule{0.241pt}{0.400pt}}
\multiput(1354.00,172.17)(0.500,-1.000){2}{\rule{0.120pt}{0.400pt}}
\put(1351.0,173.0){\rule[-0.200pt]{0.723pt}{0.400pt}}
\put(1355,172){\usebox{\plotpoint}}
\put(1357,170.67){\rule{0.241pt}{0.400pt}}
\multiput(1357.00,171.17)(0.500,-1.000){2}{\rule{0.120pt}{0.400pt}}
\put(1355.0,172.0){\rule[-0.200pt]{0.482pt}{0.400pt}}
\put(1361,169.67){\rule{0.241pt}{0.400pt}}
\multiput(1361.00,170.17)(0.500,-1.000){2}{\rule{0.120pt}{0.400pt}}
\put(1358.0,171.0){\rule[-0.200pt]{0.723pt}{0.400pt}}
\put(1362,170){\usebox{\plotpoint}}
\put(1365,168.67){\rule{0.241pt}{0.400pt}}
\multiput(1365.00,169.17)(0.500,-1.000){2}{\rule{0.120pt}{0.400pt}}
\put(1362.0,170.0){\rule[-0.200pt]{0.723pt}{0.400pt}}
\put(1366,169){\usebox{\plotpoint}}
\put(1368,167.67){\rule{0.241pt}{0.400pt}}
\multiput(1368.00,168.17)(0.500,-1.000){2}{\rule{0.120pt}{0.400pt}}
\put(1366.0,169.0){\rule[-0.200pt]{0.482pt}{0.400pt}}
\put(1372,166.67){\rule{0.241pt}{0.400pt}}
\multiput(1372.00,167.17)(0.500,-1.000){2}{\rule{0.120pt}{0.400pt}}
\put(1369.0,168.0){\rule[-0.200pt]{0.723pt}{0.400pt}}
\put(1376,165.67){\rule{0.241pt}{0.400pt}}
\multiput(1376.00,166.17)(0.500,-1.000){2}{\rule{0.120pt}{0.400pt}}
\put(1373.0,167.0){\rule[-0.200pt]{0.723pt}{0.400pt}}
\put(1379,164.67){\rule{0.241pt}{0.400pt}}
\multiput(1379.00,165.17)(0.500,-1.000){2}{\rule{0.120pt}{0.400pt}}
\put(1377.0,166.0){\rule[-0.200pt]{0.482pt}{0.400pt}}
\put(1383,163.67){\rule{0.241pt}{0.400pt}}
\multiput(1383.00,164.17)(0.500,-1.000){2}{\rule{0.120pt}{0.400pt}}
\put(1380.0,165.0){\rule[-0.200pt]{0.723pt}{0.400pt}}
\put(1387,162.67){\rule{0.241pt}{0.400pt}}
\multiput(1387.00,163.17)(0.500,-1.000){2}{\rule{0.120pt}{0.400pt}}
\put(1384.0,164.0){\rule[-0.200pt]{0.723pt}{0.400pt}}
\put(1391,161.67){\rule{0.241pt}{0.400pt}}
\multiput(1391.00,162.17)(0.500,-1.000){2}{\rule{0.120pt}{0.400pt}}
\put(1388.0,163.0){\rule[-0.200pt]{0.723pt}{0.400pt}}
\put(1395,160.67){\rule{0.241pt}{0.400pt}}
\multiput(1395.00,161.17)(0.500,-1.000){2}{\rule{0.120pt}{0.400pt}}
\put(1392.0,162.0){\rule[-0.200pt]{0.723pt}{0.400pt}}
\put(1396,161){\usebox{\plotpoint}}
\put(1399,159.67){\rule{0.241pt}{0.400pt}}
\multiput(1399.00,160.17)(0.500,-1.000){2}{\rule{0.120pt}{0.400pt}}
\put(1396.0,161.0){\rule[-0.200pt]{0.723pt}{0.400pt}}
\put(1400,160){\usebox{\plotpoint}}
\put(1403,158.67){\rule{0.241pt}{0.400pt}}
\multiput(1403.00,159.17)(0.500,-1.000){2}{\rule{0.120pt}{0.400pt}}
\put(1400.0,160.0){\rule[-0.200pt]{0.723pt}{0.400pt}}
\put(1404,159){\usebox{\plotpoint}}
\put(1407,157.67){\rule{0.241pt}{0.400pt}}
\multiput(1407.00,158.17)(0.500,-1.000){2}{\rule{0.120pt}{0.400pt}}
\put(1404.0,159.0){\rule[-0.200pt]{0.723pt}{0.400pt}}
\put(1408,158){\usebox{\plotpoint}}
\put(1411,156.67){\rule{0.241pt}{0.400pt}}
\multiput(1411.00,157.17)(0.500,-1.000){2}{\rule{0.120pt}{0.400pt}}
\put(1408.0,158.0){\rule[-0.200pt]{0.723pt}{0.400pt}}
\put(1412,157){\usebox{\plotpoint}}
\put(1415,155.67){\rule{0.241pt}{0.400pt}}
\multiput(1415.00,156.17)(0.500,-1.000){2}{\rule{0.120pt}{0.400pt}}
\put(1412.0,157.0){\rule[-0.200pt]{0.723pt}{0.400pt}}
\put(1419,154.67){\rule{0.241pt}{0.400pt}}
\multiput(1419.00,155.17)(0.500,-1.000){2}{\rule{0.120pt}{0.400pt}}
\put(1416.0,156.0){\rule[-0.200pt]{0.723pt}{0.400pt}}
\put(1424,153.67){\rule{0.241pt}{0.400pt}}
\multiput(1424.00,154.17)(0.500,-1.000){2}{\rule{0.120pt}{0.400pt}}
\put(1420.0,155.0){\rule[-0.200pt]{0.964pt}{0.400pt}}
\put(1425,154){\usebox{\plotpoint}}
\put(1428,152.67){\rule{0.241pt}{0.400pt}}
\multiput(1428.00,153.17)(0.500,-1.000){2}{\rule{0.120pt}{0.400pt}}
\put(1425.0,154.0){\rule[-0.200pt]{0.723pt}{0.400pt}}
\put(1429,153){\usebox{\plotpoint}}
\put(1432,151.67){\rule{0.241pt}{0.400pt}}
\multiput(1432.00,152.17)(0.500,-1.000){2}{\rule{0.120pt}{0.400pt}}
\put(1429.0,153.0){\rule[-0.200pt]{0.723pt}{0.400pt}}
\put(1436,150.67){\rule{0.241pt}{0.400pt}}
\multiput(1436.00,151.17)(0.500,-1.000){2}{\rule{0.120pt}{0.400pt}}
\put(1433.0,152.0){\rule[-0.200pt]{0.723pt}{0.400pt}}
\put(1437.0,151.0){\rule[-0.200pt]{0.482pt}{0.400pt}}
\put(1279,777){\makebox(0,0)[r]{$y$}}
\multiput(1299,777)(20.756,0.000){5}{\usebox{\plotpoint}}
\put(1399,777){\usebox{\plotpoint}}
\put(132,513){\usebox{\plotpoint}}
\put(132.00,513.00){\usebox{\plotpoint}}
\put(152.76,513.00){\usebox{\plotpoint}}
\put(173.51,513.00){\usebox{\plotpoint}}
\put(193.85,514.00){\usebox{\plotpoint}}
\put(213.61,515.00){\usebox{\plotpoint}}
\put(233.36,516.00){\usebox{\plotpoint}}
\put(253.70,517.00){\usebox{\plotpoint}}
\put(273.05,519.00){\usebox{\plotpoint}}
\put(292.80,520.00){\usebox{\plotpoint}}
\put(312.14,522.00){\usebox{\plotpoint}}
\put(331.07,525.00){\usebox{\plotpoint}}
\put(350.41,527.00){\usebox{\plotpoint}}
\put(369.75,529.00){\usebox{\plotpoint}}
\put(388.68,532.00){\usebox{\plotpoint}}
\put(407.61,535.00){\usebox{\plotpoint}}
\put(427.08,538.08){\usebox{\plotpoint}}
\put(446.22,542.00){\usebox{\plotpoint}}
\put(465.14,545.00){\usebox{\plotpoint}}
\put(483.66,549.00){\usebox{\plotpoint}}
\put(501.71,552.71){\usebox{\plotpoint}}
\put(520.51,556.00){\usebox{\plotpoint}}
\put(538.44,560.00){\usebox{\plotpoint}}
\put(556.37,564.00){\usebox{\plotpoint}}
\put(575.05,569.00){\usebox{\plotpoint}}
\put(593.56,573.00){\usebox{\plotpoint}}
\put(612.08,577.00){\usebox{\plotpoint}}
\put(630.17,582.00){\usebox{\plotpoint}}
\put(648.69,586.00){\usebox{\plotpoint}}
\put(667.00,590.79){\usebox{\plotpoint}}
\put(685.30,595.00){\usebox{\plotpoint}}
\put(702.64,599.00){\usebox{\plotpoint}}
\put(720.15,604.00){\usebox{\plotpoint}}
\put(737.49,608.00){\usebox{\plotpoint}}
\put(756.18,613.00){\usebox{\plotpoint}}
\put(775.28,617.00){\usebox{\plotpoint}}
\put(792.44,621.44){\usebox{\plotpoint}}
\put(810.13,626.00){\usebox{\plotpoint}}
\put(828.06,630.00){\usebox{\plotpoint}}
\put(845.99,634.00){\usebox{\plotpoint}}
\put(863.91,638.00){\usebox{\plotpoint}}
\put(882.43,642.00){\usebox{\plotpoint}}
\put(900.94,646.00){\usebox{\plotpoint}}
\put(919.87,649.00){\usebox{\plotpoint}}
\put(938.38,653.00){\usebox{\plotpoint}}
\put(956.72,656.00){\usebox{\plotpoint}}
\put(975.06,659.00){\usebox{\plotpoint}}
\put(993.99,662.00){\usebox{\plotpoint}}
\put(1012.92,665.00){\usebox{\plotpoint}}
\put(1032.84,667.00){\usebox{\plotpoint}}
\put(1051.77,670.00){\usebox{\plotpoint}}
\put(1071.11,672.00){\usebox{\plotpoint}}
\put(1091.45,673.00){\usebox{\plotpoint}}
\put(1111.56,674.56){\usebox{\plotpoint}}
\put(1132.10,675.10){\usebox{\plotpoint}}
\put(1152.48,676.00){\usebox{\plotpoint}}
\put(1173.23,676.00){\usebox{\plotpoint}}
\put(1193.99,676.00){\usebox{\plotpoint}}
\put(1214.33,675.00){\usebox{\plotpoint}}
\put(1234.67,674.00){\usebox{\plotpoint}}
\put(1255.01,673.00){\usebox{\plotpoint}}
\put(1274.35,671.00){\usebox{\plotpoint}}
\put(1293.69,669.00){\usebox{\plotpoint}}
\put(1313.62,667.00){\usebox{\plotpoint}}
\put(1331.96,664.00){\usebox{\plotpoint}}
\put(1351.48,661.00){\usebox{\plotpoint}}
\put(1370.57,657.00){\usebox{\plotpoint}}
\put(1388.92,654.00){\usebox{\plotpoint}}
\put(1406.84,650.00){\usebox{\plotpoint}}
\put(1425.53,645.00){\usebox{\plotpoint}}
\put(1439,642){\usebox{\plotpoint}}
\sbox{\plotpoint}{\rule[-0.400pt]{0.800pt}{0.800pt}}%
\sbox{\plotpoint}{\rule[-0.200pt]{0.400pt}{0.400pt}}%
\put(1279,736){\makebox(0,0)[r]{$z$}}
\sbox{\plotpoint}{\rule[-0.400pt]{0.800pt}{0.800pt}}%
\put(1299.0,736.0){\rule[-0.400pt]{24.090pt}{0.800pt}}
\put(132,581){\usebox{\plotpoint}}
\put(132,581){\usebox{\plotpoint}}
\put(136,578.84){\rule{0.241pt}{0.800pt}}
\multiput(136.00,579.34)(0.500,-1.000){2}{\rule{0.120pt}{0.800pt}}
\put(132.0,581.0){\rule[-0.400pt]{0.964pt}{0.800pt}}
\put(150,577.84){\rule{0.241pt}{0.800pt}}
\multiput(150.00,578.34)(0.500,-1.000){2}{\rule{0.120pt}{0.800pt}}
\put(137.0,580.0){\rule[-0.400pt]{3.132pt}{0.800pt}}
\put(151,579){\usebox{\plotpoint}}
\put(151.0,579.0){\rule[-0.400pt]{4.577pt}{0.800pt}}
\put(170.0,578.0){\usebox{\plotpoint}}
\put(170.0,578.0){\rule[-0.400pt]{13.249pt}{0.800pt}}
\put(225.0,577.0){\usebox{\plotpoint}}
\put(225.0,577.0){\rule[-0.400pt]{4.336pt}{0.800pt}}
\put(243.0,577.0){\usebox{\plotpoint}}
\put(243.0,578.0){\rule[-0.400pt]{21.440pt}{0.800pt}}
\put(332.0,578.0){\usebox{\plotpoint}}
\put(388,577.84){\rule{0.241pt}{0.800pt}}
\multiput(388.00,577.34)(0.500,1.000){2}{\rule{0.120pt}{0.800pt}}
\put(332.0,579.0){\rule[-0.400pt]{13.490pt}{0.800pt}}
\put(389,580){\usebox{\plotpoint}}
\put(389.0,580.0){\rule[-0.400pt]{12.768pt}{0.800pt}}
\put(442.0,580.0){\usebox{\plotpoint}}
\put(498,579.84){\rule{0.241pt}{0.800pt}}
\multiput(498.00,579.34)(0.500,1.000){2}{\rule{0.120pt}{0.800pt}}
\put(442.0,581.0){\rule[-0.400pt]{13.490pt}{0.800pt}}
\put(499,582){\usebox{\plotpoint}}
\put(563,580.84){\rule{0.241pt}{0.800pt}}
\multiput(563.00,580.34)(0.500,1.000){2}{\rule{0.120pt}{0.800pt}}
\put(499.0,582.0){\rule[-0.400pt]{15.418pt}{0.800pt}}
\put(683,581.84){\rule{0.241pt}{0.800pt}}
\multiput(683.00,581.34)(0.500,1.000){2}{\rule{0.120pt}{0.800pt}}
\put(564.0,583.0){\rule[-0.400pt]{28.667pt}{0.800pt}}
\put(684,584){\usebox{\plotpoint}}
\put(684.0,584.0){\rule[-0.400pt]{11.322pt}{0.800pt}}
\put(731.0,583.0){\usebox{\plotpoint}}
\put(731.0,583.0){\rule[-0.400pt]{26.499pt}{0.800pt}}
\put(841.0,582.0){\usebox{\plotpoint}}
\put(841.0,582.0){\rule[-0.400pt]{13.249pt}{0.800pt}}
\put(896.0,581.0){\usebox{\plotpoint}}
\put(939,578.84){\rule{0.241pt}{0.800pt}}
\multiput(939.00,579.34)(0.500,-1.000){2}{\rule{0.120pt}{0.800pt}}
\put(896.0,581.0){\rule[-0.400pt]{10.359pt}{0.800pt}}
\put(940.0,580.0){\rule[-0.400pt]{9.395pt}{0.800pt}}
\put(979.0,579.0){\usebox{\plotpoint}}
\put(979.0,579.0){\rule[-0.400pt]{8.672pt}{0.800pt}}
\put(1015.0,578.0){\usebox{\plotpoint}}
\put(1049,575.84){\rule{0.241pt}{0.800pt}}
\multiput(1049.00,576.34)(0.500,-1.000){2}{\rule{0.120pt}{0.800pt}}
\put(1015.0,578.0){\rule[-0.400pt]{8.191pt}{0.800pt}}
\put(1083,574.84){\rule{0.241pt}{0.800pt}}
\multiput(1083.00,575.34)(0.500,-1.000){2}{\rule{0.120pt}{0.800pt}}
\put(1050.0,577.0){\rule[-0.400pt]{7.950pt}{0.800pt}}
\put(1084.0,576.0){\rule[-0.400pt]{7.950pt}{0.800pt}}
\put(1117.0,575.0){\usebox{\plotpoint}}
\put(1151,572.84){\rule{0.241pt}{0.800pt}}
\multiput(1151.00,573.34)(0.500,-1.000){2}{\rule{0.120pt}{0.800pt}}
\put(1117.0,575.0){\rule[-0.400pt]{8.191pt}{0.800pt}}
\put(1152.0,574.0){\rule[-0.400pt]{8.431pt}{0.800pt}}
\put(1187.0,573.0){\usebox{\plotpoint}}
\put(1187.0,573.0){\rule[-0.400pt]{8.913pt}{0.800pt}}
\put(1224.0,572.0){\usebox{\plotpoint}}
\put(1224.0,572.0){\rule[-0.400pt]{9.636pt}{0.800pt}}
\put(1264.0,571.0){\usebox{\plotpoint}}
\put(1307,568.84){\rule{0.241pt}{0.800pt}}
\multiput(1307.00,569.34)(0.500,-1.000){2}{\rule{0.120pt}{0.800pt}}
\put(1264.0,571.0){\rule[-0.400pt]{10.359pt}{0.800pt}}
\put(1308,570){\usebox{\plotpoint}}
\put(1356,567.84){\rule{0.241pt}{0.800pt}}
\multiput(1356.00,568.34)(0.500,-1.000){2}{\rule{0.120pt}{0.800pt}}
\put(1308.0,570.0){\rule[-0.400pt]{11.563pt}{0.800pt}}
\put(1357,569){\usebox{\plotpoint}}
\put(1357.0,569.0){\rule[-0.400pt]{13.490pt}{0.800pt}}
\put(1413.0,568.0){\usebox{\plotpoint}}
\put(1413.0,568.0){\rule[-0.400pt]{6.263pt}{0.800pt}}
\sbox{\plotpoint}{\rule[-0.500pt]{1.000pt}{1.000pt}}%
\sbox{\plotpoint}{\rule[-0.200pt]{0.400pt}{0.400pt}}%
\put(1279,695){\makebox(0,0)[r]{$w$}}
\sbox{\plotpoint}{\rule[-0.500pt]{1.000pt}{1.000pt}}%
\multiput(1299,695)(20.756,0.000){5}{\usebox{\plotpoint}}
\put(1399,695){\usebox{\plotpoint}}
\put(132,841){\usebox{\plotpoint}}
\put(132.00,841.00){\usebox{\plotpoint}}
\put(152.76,841.00){\usebox{\plotpoint}}
\put(172.51,840.00){\usebox{\plotpoint}}
\put(193.27,840.00){\usebox{\plotpoint}}
\put(214.02,840.00){\usebox{\plotpoint}}
\put(234.78,840.00){\usebox{\plotpoint}}
\put(255.53,840.00){\usebox{\plotpoint}}
\put(276.29,840.00){\usebox{\plotpoint}}
\put(297.04,840.00){\usebox{\plotpoint}}
\put(317.80,840.00){\usebox{\plotpoint}}
\put(338.56,840.00){\usebox{\plotpoint}}
\put(359.31,840.00){\usebox{\plotpoint}}
\put(380.07,840.00){\usebox{\plotpoint}}
\put(400.82,840.00){\usebox{\plotpoint}}
\put(421.58,840.00){\usebox{\plotpoint}}
\put(442.33,840.00){\usebox{\plotpoint}}
\put(462.09,839.00){\usebox{\plotpoint}}
\put(482.84,839.00){\usebox{\plotpoint}}
\put(503.60,839.00){\usebox{\plotpoint}}
\put(524.35,839.00){\usebox{\plotpoint}}
\put(545.11,839.00){\usebox{\plotpoint}}
\put(565.87,839.00){\usebox{\plotpoint}}
\put(586.21,838.00){\usebox{\plotpoint}}
\put(606.96,838.00){\usebox{\plotpoint}}
\put(627.72,838.00){\usebox{\plotpoint}}
\put(648.47,838.00){\usebox{\plotpoint}}
\put(668.81,837.00){\usebox{\plotpoint}}
\put(689.57,837.00){\usebox{\plotpoint}}
\put(710.33,837.00){\usebox{\plotpoint}}
\put(731.08,837.00){\usebox{\plotpoint}}
\put(751.42,836.00){\usebox{\plotpoint}}
\put(772.18,836.00){\usebox{\plotpoint}}
\put(792.93,836.00){\usebox{\plotpoint}}
\put(813.69,836.00){\usebox{\plotpoint}}
\put(834.44,836.00){\usebox{\plotpoint}}
\put(854.20,835.00){\usebox{\plotpoint}}
\put(874.96,835.00){\usebox{\plotpoint}}
\put(895.71,835.00){\usebox{\plotpoint}}
\put(916.47,835.00){\usebox{\plotpoint}}
\put(937.22,835.00){\usebox{\plotpoint}}
\put(957.98,835.00){\usebox{\plotpoint}}
\put(978.32,834.00){\usebox{\plotpoint}}
\put(999.07,834.00){\usebox{\plotpoint}}
\put(1019.83,834.00){\usebox{\plotpoint}}
\put(1040.59,834.00){\usebox{\plotpoint}}
\put(1061.34,834.00){\usebox{\plotpoint}}
\put(1082.10,834.00){\usebox{\plotpoint}}
\put(1102.85,834.00){\usebox{\plotpoint}}
\put(1123.61,834.00){\usebox{\plotpoint}}
\put(1144.36,834.00){\usebox{\plotpoint}}
\put(1165.12,834.00){\usebox{\plotpoint}}
\put(1185.87,834.00){\usebox{\plotpoint}}
\put(1206.63,834.00){\usebox{\plotpoint}}
\put(1227.38,834.00){\usebox{\plotpoint}}
\put(1248.14,834.00){\usebox{\plotpoint}}
\put(1268.48,835.00){\usebox{\plotpoint}}
\put(1289.24,835.00){\usebox{\plotpoint}}
\put(1309.99,835.00){\usebox{\plotpoint}}
\put(1330.75,835.00){\usebox{\plotpoint}}
\put(1351.50,835.00){\usebox{\plotpoint}}
\put(1372.26,835.00){\usebox{\plotpoint}}
\put(1392.60,836.00){\usebox{\plotpoint}}
\put(1413.36,836.00){\usebox{\plotpoint}}
\put(1434.11,836.00){\usebox{\plotpoint}}
\put(1439,836){\usebox{\plotpoint}}
\sbox{\plotpoint}{\rule[-0.200pt]{0.400pt}{0.400pt}}%
\put(131.0,131.0){\rule[-0.200pt]{0.400pt}{175.375pt}}
\put(131.0,131.0){\rule[-0.200pt]{315.097pt}{0.400pt}}
\put(1439.0,131.0){\rule[-0.200pt]{0.400pt}{175.375pt}}
\put(131.0,859.0){\rule[-0.200pt]{315.097pt}{0.400pt}}
\end{picture}

    \caption{Evolucion de las variables $z_1 , z_2 , z_3 , z_4$ de $0$ a $4$ desde el punto 
			inicial $(-5.508, -0.654, 0.654, 5.647)$}
\end{figure}

Para el segundo punto fijo usamos la condicion inicial $(5.4083,0.5547, 
-0.5547,-5.5470)$ que es el mismo punto fijo, a pesar de ser el punto fijo 
el error de almacenamiento es suficiente para hacer que el sistema se vaya al infinito si hacemos 
la transici\'on de $0$ a $4$, por eso el paso transitorio s\'olo se hace 700 veces que son 
suficientes para apreciar c\'omo diverge el sistema. 

\begin{figure}[H]
    \centering
    % GNUPLOT: LaTeX picture
\setlength{\unitlength}{0.240900pt}
\ifx\plotpoint\undefined\newsavebox{\plotpoint}\fi
\sbox{\plotpoint}{\rule[-0.200pt]{0.400pt}{0.400pt}}%
\begin{picture}(1500,900)(0,0)
\sbox{\plotpoint}{\rule[-0.200pt]{0.400pt}{0.400pt}}%
\put(151.0,131.0){\rule[-0.200pt]{4.818pt}{0.400pt}}
\put(131,131){\makebox(0,0)[r]{$-20$}}
\put(1419.0,131.0){\rule[-0.200pt]{4.818pt}{0.400pt}}
\put(151.0,222.0){\rule[-0.200pt]{4.818pt}{0.400pt}}
\put(131,222){\makebox(0,0)[r]{$-15$}}
\put(1419.0,222.0){\rule[-0.200pt]{4.818pt}{0.400pt}}
\put(151.0,313.0){\rule[-0.200pt]{4.818pt}{0.400pt}}
\put(131,313){\makebox(0,0)[r]{$-10$}}
\put(1419.0,313.0){\rule[-0.200pt]{4.818pt}{0.400pt}}
\put(151.0,404.0){\rule[-0.200pt]{4.818pt}{0.400pt}}
\put(131,404){\makebox(0,0)[r]{$-5$}}
\put(1419.0,404.0){\rule[-0.200pt]{4.818pt}{0.400pt}}
\put(151.0,495.0){\rule[-0.200pt]{4.818pt}{0.400pt}}
\put(131,495){\makebox(0,0)[r]{$0$}}
\put(1419.0,495.0){\rule[-0.200pt]{4.818pt}{0.400pt}}
\put(151.0,586.0){\rule[-0.200pt]{4.818pt}{0.400pt}}
\put(131,586){\makebox(0,0)[r]{$5$}}
\put(1419.0,586.0){\rule[-0.200pt]{4.818pt}{0.400pt}}
\put(151.0,677.0){\rule[-0.200pt]{4.818pt}{0.400pt}}
\put(131,677){\makebox(0,0)[r]{$10$}}
\put(1419.0,677.0){\rule[-0.200pt]{4.818pt}{0.400pt}}
\put(151.0,768.0){\rule[-0.200pt]{4.818pt}{0.400pt}}
\put(131,768){\makebox(0,0)[r]{$15$}}
\put(1419.0,768.0){\rule[-0.200pt]{4.818pt}{0.400pt}}
\put(151.0,859.0){\rule[-0.200pt]{4.818pt}{0.400pt}}
\put(131,859){\makebox(0,0)[r]{$20$}}
\put(1419.0,859.0){\rule[-0.200pt]{4.818pt}{0.400pt}}
\put(151.0,131.0){\rule[-0.200pt]{0.400pt}{4.818pt}}
\put(151,90){\makebox(0,0){$0$}}
\put(151.0,839.0){\rule[-0.200pt]{0.400pt}{4.818pt}}
\put(312.0,131.0){\rule[-0.200pt]{0.400pt}{4.818pt}}
\put(312,90){\makebox(0,0){$0.2$}}
\put(312.0,839.0){\rule[-0.200pt]{0.400pt}{4.818pt}}
\put(473.0,131.0){\rule[-0.200pt]{0.400pt}{4.818pt}}
\put(473,90){\makebox(0,0){$0.4$}}
\put(473.0,839.0){\rule[-0.200pt]{0.400pt}{4.818pt}}
\put(634.0,131.0){\rule[-0.200pt]{0.400pt}{4.818pt}}
\put(634,90){\makebox(0,0){$0.6$}}
\put(634.0,839.0){\rule[-0.200pt]{0.400pt}{4.818pt}}
\put(795.0,131.0){\rule[-0.200pt]{0.400pt}{4.818pt}}
\put(795,90){\makebox(0,0){$0.8$}}
\put(795.0,839.0){\rule[-0.200pt]{0.400pt}{4.818pt}}
\put(956.0,131.0){\rule[-0.200pt]{0.400pt}{4.818pt}}
\put(956,90){\makebox(0,0){$1$}}
\put(956.0,839.0){\rule[-0.200pt]{0.400pt}{4.818pt}}
\put(1117.0,131.0){\rule[-0.200pt]{0.400pt}{4.818pt}}
\put(1117,90){\makebox(0,0){$1.2$}}
\put(1117.0,839.0){\rule[-0.200pt]{0.400pt}{4.818pt}}
\put(1278.0,131.0){\rule[-0.200pt]{0.400pt}{4.818pt}}
\put(1278,90){\makebox(0,0){$1.4$}}
\put(1278.0,839.0){\rule[-0.200pt]{0.400pt}{4.818pt}}
\put(1439.0,131.0){\rule[-0.200pt]{0.400pt}{4.818pt}}
\put(1439,90){\makebox(0,0){$1.6$}}
\put(1439.0,839.0){\rule[-0.200pt]{0.400pt}{4.818pt}}
\put(151.0,131.0){\rule[-0.200pt]{0.400pt}{175.375pt}}
\put(151.0,131.0){\rule[-0.200pt]{310.279pt}{0.400pt}}
\put(1439.0,131.0){\rule[-0.200pt]{0.400pt}{175.375pt}}
\put(151.0,859.0){\rule[-0.200pt]{310.279pt}{0.400pt}}
\put(30,495){\makebox(0,0){valor}}
\put(795,29){\makebox(0,0){$t$}}
\put(1279,818){\makebox(0,0)[r]{$x$}}
\put(1299.0,818.0){\rule[-0.200pt]{24.090pt}{0.400pt}}
\put(153,593){\usebox{\plotpoint}}
\put(188,591.67){\rule{0.482pt}{0.400pt}}
\multiput(188.00,592.17)(1.000,-1.000){2}{\rule{0.241pt}{0.400pt}}
\put(153.0,593.0){\rule[-0.200pt]{8.431pt}{0.400pt}}
\put(227,590.67){\rule{0.241pt}{0.400pt}}
\multiput(227.00,591.17)(0.500,-1.000){2}{\rule{0.120pt}{0.400pt}}
\put(190.0,592.0){\rule[-0.200pt]{8.913pt}{0.400pt}}
\put(269,589.67){\rule{0.241pt}{0.400pt}}
\multiput(269.00,590.17)(0.500,-1.000){2}{\rule{0.120pt}{0.400pt}}
\put(228.0,591.0){\rule[-0.200pt]{9.877pt}{0.400pt}}
\put(309,588.67){\rule{0.241pt}{0.400pt}}
\multiput(309.00,589.17)(0.500,-1.000){2}{\rule{0.120pt}{0.400pt}}
\put(270.0,590.0){\rule[-0.200pt]{9.395pt}{0.400pt}}
\put(351,587.67){\rule{0.241pt}{0.400pt}}
\multiput(351.00,588.17)(0.500,-1.000){2}{\rule{0.120pt}{0.400pt}}
\put(310.0,589.0){\rule[-0.200pt]{9.877pt}{0.400pt}}
\put(394,586.67){\rule{0.482pt}{0.400pt}}
\multiput(394.00,587.17)(1.000,-1.000){2}{\rule{0.241pt}{0.400pt}}
\put(352.0,588.0){\rule[-0.200pt]{10.118pt}{0.400pt}}
\put(439,585.67){\rule{0.482pt}{0.400pt}}
\multiput(439.00,586.17)(1.000,-1.000){2}{\rule{0.241pt}{0.400pt}}
\put(396.0,587.0){\rule[-0.200pt]{10.359pt}{0.400pt}}
\put(486,584.67){\rule{0.241pt}{0.400pt}}
\multiput(486.00,585.17)(0.500,-1.000){2}{\rule{0.120pt}{0.400pt}}
\put(441.0,586.0){\rule[-0.200pt]{10.840pt}{0.400pt}}
\put(539,583.67){\rule{0.482pt}{0.400pt}}
\multiput(539.00,584.17)(1.000,-1.000){2}{\rule{0.241pt}{0.400pt}}
\put(487.0,585.0){\rule[-0.200pt]{12.527pt}{0.400pt}}
\put(600,582.67){\rule{0.482pt}{0.400pt}}
\multiput(600.00,583.17)(1.000,-1.000){2}{\rule{0.241pt}{0.400pt}}
\put(541.0,584.0){\rule[-0.200pt]{14.213pt}{0.400pt}}
\put(695,581.67){\rule{0.482pt}{0.400pt}}
\multiput(695.00,582.17)(1.000,-1.000){2}{\rule{0.241pt}{0.400pt}}
\put(602.0,583.0){\rule[-0.200pt]{22.404pt}{0.400pt}}
\put(792,581.67){\rule{0.241pt}{0.400pt}}
\multiput(792.00,581.17)(0.500,1.000){2}{\rule{0.120pt}{0.400pt}}
\put(697.0,582.0){\rule[-0.200pt]{22.885pt}{0.400pt}}
\put(861,582.67){\rule{0.482pt}{0.400pt}}
\multiput(861.00,582.17)(1.000,1.000){2}{\rule{0.241pt}{0.400pt}}
\put(793.0,583.0){\rule[-0.200pt]{16.381pt}{0.400pt}}
\put(898,583.67){\rule{0.482pt}{0.400pt}}
\multiput(898.00,583.17)(1.000,1.000){2}{\rule{0.241pt}{0.400pt}}
\put(863.0,584.0){\rule[-0.200pt]{8.431pt}{0.400pt}}
\put(924,584.67){\rule{0.241pt}{0.400pt}}
\multiput(924.00,584.17)(0.500,1.000){2}{\rule{0.120pt}{0.400pt}}
\put(900.0,585.0){\rule[-0.200pt]{5.782pt}{0.400pt}}
\put(946,585.67){\rule{0.482pt}{0.400pt}}
\multiput(946.00,585.17)(1.000,1.000){2}{\rule{0.241pt}{0.400pt}}
\put(925.0,586.0){\rule[-0.200pt]{5.059pt}{0.400pt}}
\put(964,586.67){\rule{0.482pt}{0.400pt}}
\multiput(964.00,586.17)(1.000,1.000){2}{\rule{0.241pt}{0.400pt}}
\put(948.0,587.0){\rule[-0.200pt]{3.854pt}{0.400pt}}
\put(980,587.67){\rule{0.482pt}{0.400pt}}
\multiput(980.00,587.17)(1.000,1.000){2}{\rule{0.241pt}{0.400pt}}
\put(966.0,588.0){\rule[-0.200pt]{3.373pt}{0.400pt}}
\put(993,588.67){\rule{0.482pt}{0.400pt}}
\multiput(993.00,588.17)(1.000,1.000){2}{\rule{0.241pt}{0.400pt}}
\put(982.0,589.0){\rule[-0.200pt]{2.650pt}{0.400pt}}
\put(1006,589.67){\rule{0.482pt}{0.400pt}}
\multiput(1006.00,589.17)(1.000,1.000){2}{\rule{0.241pt}{0.400pt}}
\put(995.0,590.0){\rule[-0.200pt]{2.650pt}{0.400pt}}
\put(1017,590.67){\rule{0.482pt}{0.400pt}}
\multiput(1017.00,590.17)(1.000,1.000){2}{\rule{0.241pt}{0.400pt}}
\put(1008.0,591.0){\rule[-0.200pt]{2.168pt}{0.400pt}}
\put(1028,591.67){\rule{0.482pt}{0.400pt}}
\multiput(1028.00,591.17)(1.000,1.000){2}{\rule{0.241pt}{0.400pt}}
\put(1019.0,592.0){\rule[-0.200pt]{2.168pt}{0.400pt}}
\put(1038,592.67){\rule{0.482pt}{0.400pt}}
\multiput(1038.00,592.17)(1.000,1.000){2}{\rule{0.241pt}{0.400pt}}
\put(1030.0,593.0){\rule[-0.200pt]{1.927pt}{0.400pt}}
\put(1048,593.67){\rule{0.241pt}{0.400pt}}
\multiput(1048.00,593.17)(0.500,1.000){2}{\rule{0.120pt}{0.400pt}}
\put(1040.0,594.0){\rule[-0.200pt]{1.927pt}{0.400pt}}
\put(1056,594.67){\rule{0.241pt}{0.400pt}}
\multiput(1056.00,594.17)(0.500,1.000){2}{\rule{0.120pt}{0.400pt}}
\put(1049.0,595.0){\rule[-0.200pt]{1.686pt}{0.400pt}}
\put(1064,595.67){\rule{0.241pt}{0.400pt}}
\multiput(1064.00,595.17)(0.500,1.000){2}{\rule{0.120pt}{0.400pt}}
\put(1057.0,596.0){\rule[-0.200pt]{1.686pt}{0.400pt}}
\put(1072,596.67){\rule{0.482pt}{0.400pt}}
\multiput(1072.00,596.17)(1.000,1.000){2}{\rule{0.241pt}{0.400pt}}
\put(1065.0,597.0){\rule[-0.200pt]{1.686pt}{0.400pt}}
\put(1078,597.67){\rule{0.482pt}{0.400pt}}
\multiput(1078.00,597.17)(1.000,1.000){2}{\rule{0.241pt}{0.400pt}}
\put(1074.0,598.0){\rule[-0.200pt]{0.964pt}{0.400pt}}
\put(1085,598.67){\rule{0.241pt}{0.400pt}}
\multiput(1085.00,598.17)(0.500,1.000){2}{\rule{0.120pt}{0.400pt}}
\put(1080.0,599.0){\rule[-0.200pt]{1.204pt}{0.400pt}}
\put(1091,599.67){\rule{0.482pt}{0.400pt}}
\multiput(1091.00,599.17)(1.000,1.000){2}{\rule{0.241pt}{0.400pt}}
\put(1086.0,600.0){\rule[-0.200pt]{1.204pt}{0.400pt}}
\put(1098,600.67){\rule{0.241pt}{0.400pt}}
\multiput(1098.00,600.17)(0.500,1.000){2}{\rule{0.120pt}{0.400pt}}
\put(1093.0,601.0){\rule[-0.200pt]{1.204pt}{0.400pt}}
\put(1104,601.67){\rule{0.482pt}{0.400pt}}
\multiput(1104.00,601.17)(1.000,1.000){2}{\rule{0.241pt}{0.400pt}}
\put(1099.0,602.0){\rule[-0.200pt]{1.204pt}{0.400pt}}
\put(1109,602.67){\rule{0.482pt}{0.400pt}}
\multiput(1109.00,602.17)(1.000,1.000){2}{\rule{0.241pt}{0.400pt}}
\put(1106.0,603.0){\rule[-0.200pt]{0.723pt}{0.400pt}}
\put(1114,603.67){\rule{0.241pt}{0.400pt}}
\multiput(1114.00,603.17)(0.500,1.000){2}{\rule{0.120pt}{0.400pt}}
\put(1111.0,604.0){\rule[-0.200pt]{0.723pt}{0.400pt}}
\put(1119,604.67){\rule{0.241pt}{0.400pt}}
\multiput(1119.00,604.17)(0.500,1.000){2}{\rule{0.120pt}{0.400pt}}
\put(1115.0,605.0){\rule[-0.200pt]{0.964pt}{0.400pt}}
\put(1125,605.67){\rule{0.482pt}{0.400pt}}
\multiput(1125.00,605.17)(1.000,1.000){2}{\rule{0.241pt}{0.400pt}}
\put(1120.0,606.0){\rule[-0.200pt]{1.204pt}{0.400pt}}
\put(1128,606.67){\rule{0.482pt}{0.400pt}}
\multiput(1128.00,606.17)(1.000,1.000){2}{\rule{0.241pt}{0.400pt}}
\put(1127.0,607.0){\usebox{\plotpoint}}
\put(1133,607.67){\rule{0.482pt}{0.400pt}}
\multiput(1133.00,607.17)(1.000,1.000){2}{\rule{0.241pt}{0.400pt}}
\put(1130.0,608.0){\rule[-0.200pt]{0.723pt}{0.400pt}}
\put(1138,608.67){\rule{0.482pt}{0.400pt}}
\multiput(1138.00,608.17)(1.000,1.000){2}{\rule{0.241pt}{0.400pt}}
\put(1135.0,609.0){\rule[-0.200pt]{0.723pt}{0.400pt}}
\put(1143,609.67){\rule{0.241pt}{0.400pt}}
\multiput(1143.00,609.17)(0.500,1.000){2}{\rule{0.120pt}{0.400pt}}
\put(1140.0,610.0){\rule[-0.200pt]{0.723pt}{0.400pt}}
\put(1146,610.67){\rule{0.482pt}{0.400pt}}
\multiput(1146.00,610.17)(1.000,1.000){2}{\rule{0.241pt}{0.400pt}}
\put(1144.0,611.0){\rule[-0.200pt]{0.482pt}{0.400pt}}
\put(1151,611.67){\rule{0.241pt}{0.400pt}}
\multiput(1151.00,611.17)(0.500,1.000){2}{\rule{0.120pt}{0.400pt}}
\put(1148.0,612.0){\rule[-0.200pt]{0.723pt}{0.400pt}}
\put(1154,612.67){\rule{0.482pt}{0.400pt}}
\multiput(1154.00,612.17)(1.000,1.000){2}{\rule{0.241pt}{0.400pt}}
\put(1152.0,613.0){\rule[-0.200pt]{0.482pt}{0.400pt}}
\put(1157,613.67){\rule{0.482pt}{0.400pt}}
\multiput(1157.00,613.17)(1.000,1.000){2}{\rule{0.241pt}{0.400pt}}
\put(1156.0,614.0){\usebox{\plotpoint}}
\put(1162,614.67){\rule{0.482pt}{0.400pt}}
\multiput(1162.00,614.17)(1.000,1.000){2}{\rule{0.241pt}{0.400pt}}
\put(1159.0,615.0){\rule[-0.200pt]{0.723pt}{0.400pt}}
\put(1165,615.67){\rule{0.482pt}{0.400pt}}
\multiput(1165.00,615.17)(1.000,1.000){2}{\rule{0.241pt}{0.400pt}}
\put(1164.0,616.0){\usebox{\plotpoint}}
\put(1169,616.67){\rule{0.241pt}{0.400pt}}
\multiput(1169.00,616.17)(0.500,1.000){2}{\rule{0.120pt}{0.400pt}}
\put(1167.0,617.0){\rule[-0.200pt]{0.482pt}{0.400pt}}
\put(1172,617.67){\rule{0.241pt}{0.400pt}}
\multiput(1172.00,617.17)(0.500,1.000){2}{\rule{0.120pt}{0.400pt}}
\put(1170.0,618.0){\rule[-0.200pt]{0.482pt}{0.400pt}}
\put(1175,618.67){\rule{0.482pt}{0.400pt}}
\multiput(1175.00,618.17)(1.000,1.000){2}{\rule{0.241pt}{0.400pt}}
\put(1173.0,619.0){\rule[-0.200pt]{0.482pt}{0.400pt}}
\put(1178,619.67){\rule{0.482pt}{0.400pt}}
\multiput(1178.00,619.17)(1.000,1.000){2}{\rule{0.241pt}{0.400pt}}
\put(1177.0,620.0){\usebox{\plotpoint}}
\put(1181,620.67){\rule{0.482pt}{0.400pt}}
\multiput(1181.00,620.17)(1.000,1.000){2}{\rule{0.241pt}{0.400pt}}
\put(1180.0,621.0){\usebox{\plotpoint}}
\put(1185,621.67){\rule{0.241pt}{0.400pt}}
\multiput(1185.00,621.17)(0.500,1.000){2}{\rule{0.120pt}{0.400pt}}
\put(1186,622.67){\rule{0.482pt}{0.400pt}}
\multiput(1186.00,622.17)(1.000,1.000){2}{\rule{0.241pt}{0.400pt}}
\put(1183.0,622.0){\rule[-0.200pt]{0.482pt}{0.400pt}}
\put(1189,623.67){\rule{0.482pt}{0.400pt}}
\multiput(1189.00,623.17)(1.000,1.000){2}{\rule{0.241pt}{0.400pt}}
\put(1188.0,624.0){\usebox{\plotpoint}}
\put(1193,624.67){\rule{0.241pt}{0.400pt}}
\multiput(1193.00,624.17)(0.500,1.000){2}{\rule{0.120pt}{0.400pt}}
\put(1191.0,625.0){\rule[-0.200pt]{0.482pt}{0.400pt}}
\put(1196,625.67){\rule{0.482pt}{0.400pt}}
\multiput(1196.00,625.17)(1.000,1.000){2}{\rule{0.241pt}{0.400pt}}
\put(1198,626.67){\rule{0.241pt}{0.400pt}}
\multiput(1198.00,626.17)(0.500,1.000){2}{\rule{0.120pt}{0.400pt}}
\put(1194.0,626.0){\rule[-0.200pt]{0.482pt}{0.400pt}}
\put(1201,627.67){\rule{0.241pt}{0.400pt}}
\multiput(1201.00,627.17)(0.500,1.000){2}{\rule{0.120pt}{0.400pt}}
\put(1202,628.67){\rule{0.482pt}{0.400pt}}
\multiput(1202.00,628.17)(1.000,1.000){2}{\rule{0.241pt}{0.400pt}}
\put(1199.0,628.0){\rule[-0.200pt]{0.482pt}{0.400pt}}
\put(1206,629.67){\rule{0.241pt}{0.400pt}}
\multiput(1206.00,629.17)(0.500,1.000){2}{\rule{0.120pt}{0.400pt}}
\put(1207,630.67){\rule{0.482pt}{0.400pt}}
\multiput(1207.00,630.17)(1.000,1.000){2}{\rule{0.241pt}{0.400pt}}
\put(1204.0,630.0){\rule[-0.200pt]{0.482pt}{0.400pt}}
\put(1210,631.67){\rule{0.482pt}{0.400pt}}
\multiput(1210.00,631.17)(1.000,1.000){2}{\rule{0.241pt}{0.400pt}}
\put(1212,632.67){\rule{0.482pt}{0.400pt}}
\multiput(1212.00,632.17)(1.000,1.000){2}{\rule{0.241pt}{0.400pt}}
\put(1209.0,632.0){\usebox{\plotpoint}}
\put(1215,633.67){\rule{0.482pt}{0.400pt}}
\multiput(1215.00,633.17)(1.000,1.000){2}{\rule{0.241pt}{0.400pt}}
\put(1217,634.67){\rule{0.241pt}{0.400pt}}
\multiput(1217.00,634.17)(0.500,1.000){2}{\rule{0.120pt}{0.400pt}}
\put(1218,635.67){\rule{0.482pt}{0.400pt}}
\multiput(1218.00,635.17)(1.000,1.000){2}{\rule{0.241pt}{0.400pt}}
\put(1214.0,634.0){\usebox{\plotpoint}}
\put(1222,636.67){\rule{0.241pt}{0.400pt}}
\multiput(1222.00,636.17)(0.500,1.000){2}{\rule{0.120pt}{0.400pt}}
\put(1223,637.67){\rule{0.482pt}{0.400pt}}
\multiput(1223.00,637.17)(1.000,1.000){2}{\rule{0.241pt}{0.400pt}}
\put(1225,638.67){\rule{0.241pt}{0.400pt}}
\multiput(1225.00,638.17)(0.500,1.000){2}{\rule{0.120pt}{0.400pt}}
\put(1226,639.67){\rule{0.482pt}{0.400pt}}
\multiput(1226.00,639.17)(1.000,1.000){2}{\rule{0.241pt}{0.400pt}}
\put(1228,640.67){\rule{0.482pt}{0.400pt}}
\multiput(1228.00,640.17)(1.000,1.000){2}{\rule{0.241pt}{0.400pt}}
\put(1220.0,637.0){\rule[-0.200pt]{0.482pt}{0.400pt}}
\put(1231,641.67){\rule{0.482pt}{0.400pt}}
\multiput(1231.00,641.17)(1.000,1.000){2}{\rule{0.241pt}{0.400pt}}
\put(1233,642.67){\rule{0.482pt}{0.400pt}}
\multiput(1233.00,642.17)(1.000,1.000){2}{\rule{0.241pt}{0.400pt}}
\put(1235,643.67){\rule{0.241pt}{0.400pt}}
\multiput(1235.00,643.17)(0.500,1.000){2}{\rule{0.120pt}{0.400pt}}
\put(1236,644.67){\rule{0.482pt}{0.400pt}}
\multiput(1236.00,644.17)(1.000,1.000){2}{\rule{0.241pt}{0.400pt}}
\put(1238,645.67){\rule{0.241pt}{0.400pt}}
\multiput(1238.00,645.17)(0.500,1.000){2}{\rule{0.120pt}{0.400pt}}
\put(1239,646.67){\rule{0.482pt}{0.400pt}}
\multiput(1239.00,646.17)(1.000,1.000){2}{\rule{0.241pt}{0.400pt}}
\put(1241,647.67){\rule{0.482pt}{0.400pt}}
\multiput(1241.00,647.17)(1.000,1.000){2}{\rule{0.241pt}{0.400pt}}
\put(1243,648.67){\rule{0.241pt}{0.400pt}}
\multiput(1243.00,648.17)(0.500,1.000){2}{\rule{0.120pt}{0.400pt}}
\put(1244,649.67){\rule{0.482pt}{0.400pt}}
\multiput(1244.00,649.17)(1.000,1.000){2}{\rule{0.241pt}{0.400pt}}
\put(1246,650.67){\rule{0.241pt}{0.400pt}}
\multiput(1246.00,650.17)(0.500,1.000){2}{\rule{0.120pt}{0.400pt}}
\put(1247,651.67){\rule{0.482pt}{0.400pt}}
\multiput(1247.00,651.17)(1.000,1.000){2}{\rule{0.241pt}{0.400pt}}
\put(1249,652.67){\rule{0.482pt}{0.400pt}}
\multiput(1249.00,652.17)(1.000,1.000){2}{\rule{0.241pt}{0.400pt}}
\put(1251,653.67){\rule{0.241pt}{0.400pt}}
\multiput(1251.00,653.17)(0.500,1.000){2}{\rule{0.120pt}{0.400pt}}
\put(1252,654.67){\rule{0.482pt}{0.400pt}}
\multiput(1252.00,654.17)(1.000,1.000){2}{\rule{0.241pt}{0.400pt}}
\put(1254,655.67){\rule{0.241pt}{0.400pt}}
\multiput(1254.00,655.17)(0.500,1.000){2}{\rule{0.120pt}{0.400pt}}
\put(1255,656.67){\rule{0.482pt}{0.400pt}}
\multiput(1255.00,656.17)(1.000,1.000){2}{\rule{0.241pt}{0.400pt}}
\put(1257,657.67){\rule{0.482pt}{0.400pt}}
\multiput(1257.00,657.17)(1.000,1.000){2}{\rule{0.241pt}{0.400pt}}
\put(1259,658.67){\rule{0.241pt}{0.400pt}}
\multiput(1259.00,658.17)(0.500,1.000){2}{\rule{0.120pt}{0.400pt}}
\put(1260,660.17){\rule{0.482pt}{0.400pt}}
\multiput(1260.00,659.17)(1.000,2.000){2}{\rule{0.241pt}{0.400pt}}
\put(1262,661.67){\rule{0.482pt}{0.400pt}}
\multiput(1262.00,661.17)(1.000,1.000){2}{\rule{0.241pt}{0.400pt}}
\put(1264,662.67){\rule{0.241pt}{0.400pt}}
\multiput(1264.00,662.17)(0.500,1.000){2}{\rule{0.120pt}{0.400pt}}
\put(1265,663.67){\rule{0.482pt}{0.400pt}}
\multiput(1265.00,663.17)(1.000,1.000){2}{\rule{0.241pt}{0.400pt}}
\put(1267,664.67){\rule{0.241pt}{0.400pt}}
\multiput(1267.00,664.17)(0.500,1.000){2}{\rule{0.120pt}{0.400pt}}
\put(1268,666.17){\rule{0.482pt}{0.400pt}}
\multiput(1268.00,665.17)(1.000,2.000){2}{\rule{0.241pt}{0.400pt}}
\put(1270,667.67){\rule{0.482pt}{0.400pt}}
\multiput(1270.00,667.17)(1.000,1.000){2}{\rule{0.241pt}{0.400pt}}
\put(1272,668.67){\rule{0.241pt}{0.400pt}}
\multiput(1272.00,668.17)(0.500,1.000){2}{\rule{0.120pt}{0.400pt}}
\put(1273,670.17){\rule{0.482pt}{0.400pt}}
\multiput(1273.00,669.17)(1.000,2.000){2}{\rule{0.241pt}{0.400pt}}
\put(1275,671.67){\rule{0.241pt}{0.400pt}}
\multiput(1275.00,671.17)(0.500,1.000){2}{\rule{0.120pt}{0.400pt}}
\put(1276,673.17){\rule{0.482pt}{0.400pt}}
\multiput(1276.00,672.17)(1.000,2.000){2}{\rule{0.241pt}{0.400pt}}
\put(1278,674.67){\rule{0.482pt}{0.400pt}}
\multiput(1278.00,674.17)(1.000,1.000){2}{\rule{0.241pt}{0.400pt}}
\put(1279.67,676){\rule{0.400pt}{0.482pt}}
\multiput(1279.17,676.00)(1.000,1.000){2}{\rule{0.400pt}{0.241pt}}
\put(1281,677.67){\rule{0.482pt}{0.400pt}}
\multiput(1281.00,677.17)(1.000,1.000){2}{\rule{0.241pt}{0.400pt}}
\put(1282.67,679){\rule{0.400pt}{0.482pt}}
\multiput(1282.17,679.00)(1.000,1.000){2}{\rule{0.400pt}{0.241pt}}
\put(1284,680.67){\rule{0.482pt}{0.400pt}}
\multiput(1284.00,680.17)(1.000,1.000){2}{\rule{0.241pt}{0.400pt}}
\put(1286,682.17){\rule{0.482pt}{0.400pt}}
\multiput(1286.00,681.17)(1.000,2.000){2}{\rule{0.241pt}{0.400pt}}
\put(1288,683.67){\rule{0.241pt}{0.400pt}}
\multiput(1288.00,683.17)(0.500,1.000){2}{\rule{0.120pt}{0.400pt}}
\put(1289,685.17){\rule{0.482pt}{0.400pt}}
\multiput(1289.00,684.17)(1.000,2.000){2}{\rule{0.241pt}{0.400pt}}
\put(1290.67,687){\rule{0.400pt}{0.482pt}}
\multiput(1290.17,687.00)(1.000,1.000){2}{\rule{0.400pt}{0.241pt}}
\put(1292,689.17){\rule{0.482pt}{0.400pt}}
\multiput(1292.00,688.17)(1.000,2.000){2}{\rule{0.241pt}{0.400pt}}
\put(1294,690.67){\rule{0.482pt}{0.400pt}}
\multiput(1294.00,690.17)(1.000,1.000){2}{\rule{0.241pt}{0.400pt}}
\put(1295.67,692){\rule{0.400pt}{0.482pt}}
\multiput(1295.17,692.00)(1.000,1.000){2}{\rule{0.400pt}{0.241pt}}
\put(1297,694.17){\rule{0.482pt}{0.400pt}}
\multiput(1297.00,693.17)(1.000,2.000){2}{\rule{0.241pt}{0.400pt}}
\put(1299,696.17){\rule{0.482pt}{0.400pt}}
\multiput(1299.00,695.17)(1.000,2.000){2}{\rule{0.241pt}{0.400pt}}
\put(1300.67,698){\rule{0.400pt}{0.482pt}}
\multiput(1300.17,698.00)(1.000,1.000){2}{\rule{0.400pt}{0.241pt}}
\put(1302,700.17){\rule{0.482pt}{0.400pt}}
\multiput(1302.00,699.17)(1.000,2.000){2}{\rule{0.241pt}{0.400pt}}
\put(1303.67,702){\rule{0.400pt}{0.482pt}}
\multiput(1303.17,702.00)(1.000,1.000){2}{\rule{0.400pt}{0.241pt}}
\put(1305,704.17){\rule{0.482pt}{0.400pt}}
\multiput(1305.00,703.17)(1.000,2.000){2}{\rule{0.241pt}{0.400pt}}
\put(1307,706.17){\rule{0.482pt}{0.400pt}}
\multiput(1307.00,705.17)(1.000,2.000){2}{\rule{0.241pt}{0.400pt}}
\put(1308.67,708){\rule{0.400pt}{0.482pt}}
\multiput(1308.17,708.00)(1.000,1.000){2}{\rule{0.400pt}{0.241pt}}
\put(1310.17,710){\rule{0.400pt}{0.700pt}}
\multiput(1309.17,710.00)(2.000,1.547){2}{\rule{0.400pt}{0.350pt}}
\put(1311.67,713){\rule{0.400pt}{0.482pt}}
\multiput(1311.17,713.00)(1.000,1.000){2}{\rule{0.400pt}{0.241pt}}
\put(1313,715.17){\rule{0.482pt}{0.400pt}}
\multiput(1313.00,714.17)(1.000,2.000){2}{\rule{0.241pt}{0.400pt}}
\put(1315.17,717){\rule{0.400pt}{0.700pt}}
\multiput(1314.17,717.00)(2.000,1.547){2}{\rule{0.400pt}{0.350pt}}
\put(1316.67,720){\rule{0.400pt}{0.482pt}}
\multiput(1316.17,720.00)(1.000,1.000){2}{\rule{0.400pt}{0.241pt}}
\put(1318.17,722){\rule{0.400pt}{0.700pt}}
\multiput(1317.17,722.00)(2.000,1.547){2}{\rule{0.400pt}{0.350pt}}
\put(1319.67,725){\rule{0.400pt}{0.482pt}}
\multiput(1319.17,725.00)(1.000,1.000){2}{\rule{0.400pt}{0.241pt}}
\put(1321.17,727){\rule{0.400pt}{0.700pt}}
\multiput(1320.17,727.00)(2.000,1.547){2}{\rule{0.400pt}{0.350pt}}
\put(1323.17,730){\rule{0.400pt}{0.700pt}}
\multiput(1322.17,730.00)(2.000,1.547){2}{\rule{0.400pt}{0.350pt}}
\put(1324.67,733){\rule{0.400pt}{0.482pt}}
\multiput(1324.17,733.00)(1.000,1.000){2}{\rule{0.400pt}{0.241pt}}
\put(1326.17,735){\rule{0.400pt}{0.700pt}}
\multiput(1325.17,735.00)(2.000,1.547){2}{\rule{0.400pt}{0.350pt}}
\put(1328.17,738){\rule{0.400pt}{0.700pt}}
\multiput(1327.17,738.00)(2.000,1.547){2}{\rule{0.400pt}{0.350pt}}
\put(1329.67,741){\rule{0.400pt}{0.723pt}}
\multiput(1329.17,741.00)(1.000,1.500){2}{\rule{0.400pt}{0.361pt}}
\put(1331.17,744){\rule{0.400pt}{0.700pt}}
\multiput(1330.17,744.00)(2.000,1.547){2}{\rule{0.400pt}{0.350pt}}
\put(1332.67,747){\rule{0.400pt}{0.964pt}}
\multiput(1332.17,747.00)(1.000,2.000){2}{\rule{0.400pt}{0.482pt}}
\put(1334.17,751){\rule{0.400pt}{0.700pt}}
\multiput(1333.17,751.00)(2.000,1.547){2}{\rule{0.400pt}{0.350pt}}
\put(1336.17,754){\rule{0.400pt}{0.700pt}}
\multiput(1335.17,754.00)(2.000,1.547){2}{\rule{0.400pt}{0.350pt}}
\put(1337.67,757){\rule{0.400pt}{0.964pt}}
\multiput(1337.17,757.00)(1.000,2.000){2}{\rule{0.400pt}{0.482pt}}
\put(1339.17,761){\rule{0.400pt}{0.700pt}}
\multiput(1338.17,761.00)(2.000,1.547){2}{\rule{0.400pt}{0.350pt}}
\put(1340.67,764){\rule{0.400pt}{0.964pt}}
\multiput(1340.17,764.00)(1.000,2.000){2}{\rule{0.400pt}{0.482pt}}
\put(1342.17,768){\rule{0.400pt}{0.900pt}}
\multiput(1341.17,768.00)(2.000,2.132){2}{\rule{0.400pt}{0.450pt}}
\put(1344.17,772){\rule{0.400pt}{0.900pt}}
\multiput(1343.17,772.00)(2.000,2.132){2}{\rule{0.400pt}{0.450pt}}
\put(1345.67,776){\rule{0.400pt}{0.964pt}}
\multiput(1345.17,776.00)(1.000,2.000){2}{\rule{0.400pt}{0.482pt}}
\put(1347.17,780){\rule{0.400pt}{0.900pt}}
\multiput(1346.17,780.00)(2.000,2.132){2}{\rule{0.400pt}{0.450pt}}
\put(1348.67,784){\rule{0.400pt}{1.204pt}}
\multiput(1348.17,784.00)(1.000,2.500){2}{\rule{0.400pt}{0.602pt}}
\put(1350.17,789){\rule{0.400pt}{0.900pt}}
\multiput(1349.17,789.00)(2.000,2.132){2}{\rule{0.400pt}{0.450pt}}
\put(1352.17,793){\rule{0.400pt}{1.100pt}}
\multiput(1351.17,793.00)(2.000,2.717){2}{\rule{0.400pt}{0.550pt}}
\put(1353.67,798){\rule{0.400pt}{0.964pt}}
\multiput(1353.17,798.00)(1.000,2.000){2}{\rule{0.400pt}{0.482pt}}
\put(1355.17,802){\rule{0.400pt}{1.100pt}}
\multiput(1354.17,802.00)(2.000,2.717){2}{\rule{0.400pt}{0.550pt}}
\put(1357.17,807){\rule{0.400pt}{1.300pt}}
\multiput(1356.17,807.00)(2.000,3.302){2}{\rule{0.400pt}{0.650pt}}
\put(1358.67,813){\rule{0.400pt}{1.204pt}}
\multiput(1358.17,813.00)(1.000,2.500){2}{\rule{0.400pt}{0.602pt}}
\put(1360.17,818){\rule{0.400pt}{1.100pt}}
\multiput(1359.17,818.00)(2.000,2.717){2}{\rule{0.400pt}{0.550pt}}
\put(1361.67,823){\rule{0.400pt}{1.445pt}}
\multiput(1361.17,823.00)(1.000,3.000){2}{\rule{0.400pt}{0.723pt}}
\put(1363.17,829){\rule{0.400pt}{1.300pt}}
\multiput(1362.17,829.00)(2.000,3.302){2}{\rule{0.400pt}{0.650pt}}
\put(1365.17,835){\rule{0.400pt}{1.300pt}}
\multiput(1364.17,835.00)(2.000,3.302){2}{\rule{0.400pt}{0.650pt}}
\put(1366.67,841){\rule{0.400pt}{1.445pt}}
\multiput(1366.17,841.00)(1.000,3.000){2}{\rule{0.400pt}{0.723pt}}
\put(1368.17,847){\rule{0.400pt}{1.500pt}}
\multiput(1367.17,847.00)(2.000,3.887){2}{\rule{0.400pt}{0.750pt}}
\put(1369.67,854){\rule{0.400pt}{1.204pt}}
\multiput(1369.17,854.00)(1.000,2.500){2}{\rule{0.400pt}{0.602pt}}
\put(1230.0,642.0){\usebox{\plotpoint}}
\put(1279,777){\makebox(0,0)[r]{$y$}}
\multiput(1299,777)(20.756,0.000){5}{\usebox{\plotpoint}}
\put(1399,777){\usebox{\plotpoint}}
\put(153,505){\usebox{\plotpoint}}
\put(153.00,505.00){\usebox{\plotpoint}}
\put(173.76,505.00){\usebox{\plotpoint}}
\put(194.51,505.00){\usebox{\plotpoint}}
\put(215.27,505.00){\usebox{\plotpoint}}
\put(236.02,505.00){\usebox{\plotpoint}}
\put(256.78,505.00){\usebox{\plotpoint}}
\put(277.53,505.00){\usebox{\plotpoint}}
\put(298.29,505.00){\usebox{\plotpoint}}
\put(319.04,505.00){\usebox{\plotpoint}}
\put(339.80,505.00){\usebox{\plotpoint}}
\put(360.32,504.00){\usebox{\plotpoint}}
\put(381.07,504.00){\usebox{\plotpoint}}
\put(401.83,504.00){\usebox{\plotpoint}}
\put(422.59,504.00){\usebox{\plotpoint}}
\put(443.34,504.00){\usebox{\plotpoint}}
\put(464.10,504.00){\usebox{\plotpoint}}
\put(484.62,503.00){\usebox{\plotpoint}}
\put(505.37,503.00){\usebox{\plotpoint}}
\put(526.13,503.00){\usebox{\plotpoint}}
\put(546.88,503.00){\usebox{\plotpoint}}
\put(567.22,502.00){\usebox{\plotpoint}}
\put(587.98,502.00){\usebox{\plotpoint}}
\put(608.73,502.00){\usebox{\plotpoint}}
\put(629.44,501.78){\usebox{\plotpoint}}
\put(650.01,501.00){\usebox{\plotpoint}}
\put(670.77,501.00){\usebox{\plotpoint}}
\put(691.52,501.00){\usebox{\plotpoint}}
\put(712.04,500.00){\usebox{\plotpoint}}
\put(732.80,500.00){\usebox{\plotpoint}}
\put(753.55,500.00){\usebox{\plotpoint}}
\put(773.89,499.00){\usebox{\plotpoint}}
\put(794.65,499.00){\usebox{\plotpoint}}
\put(814.99,498.00){\usebox{\plotpoint}}
\put(835.74,498.00){\usebox{\plotpoint}}
\put(856.50,498.00){\usebox{\plotpoint}}
\put(877.02,497.00){\usebox{\plotpoint}}
\put(897.77,497.00){\usebox{\plotpoint}}
\put(918.53,497.00){\usebox{\plotpoint}}
\put(939.05,496.00){\usebox{\plotpoint}}
\put(959.81,496.00){\usebox{\plotpoint}}
\put(980.50,495.75){\usebox{\plotpoint}}
\put(1001.08,495.00){\usebox{\plotpoint}}
\put(1021.84,495.00){\usebox{\plotpoint}}
\put(1042.59,495.00){\usebox{\plotpoint}}
\put(1062.93,494.00){\usebox{\plotpoint}}
\put(1083.69,494.00){\usebox{\plotpoint}}
\put(1104.44,494.00){\usebox{\plotpoint}}
\put(1125.20,494.00){\usebox{\plotpoint}}
\put(1145.54,493.00){\usebox{\plotpoint}}
\put(1166.30,493.00){\usebox{\plotpoint}}
\put(1187.05,493.00){\usebox{\plotpoint}}
\put(1207.39,494.00){\usebox{\plotpoint}}
\put(1228.15,494.00){\usebox{\plotpoint}}
\put(1248.90,494.00){\usebox{\plotpoint}}
\put(1269.24,495.00){\usebox{\plotpoint}}
\put(1290.00,495.00){\usebox{\plotpoint}}
\put(1310.52,496.00){\usebox{\plotpoint}}
\put(1330.80,498.00){\usebox{\plotpoint}}
\put(1351.02,499.51){\usebox{\plotpoint}}
\put(1371.01,502.00){\usebox{\plotpoint}}
\put(1390.84,505.92){\usebox{\plotpoint}}
\put(1410.22,511.00){\usebox{\plotpoint}}
\put(1428.39,519.39){\usebox{\plotpoint}}
\put(1439,530){\usebox{\plotpoint}}
\sbox{\plotpoint}{\rule[-0.400pt]{0.800pt}{0.800pt}}%
\sbox{\plotpoint}{\rule[-0.200pt]{0.400pt}{0.400pt}}%
\put(1279,736){\makebox(0,0)[r]{$z$}}
\sbox{\plotpoint}{\rule[-0.400pt]{0.800pt}{0.800pt}}%
\put(1299.0,736.0){\rule[-0.400pt]{24.090pt}{0.800pt}}
\put(153,485){\usebox{\plotpoint}}
\put(355,483.84){\rule{0.482pt}{0.800pt}}
\multiput(355.00,483.34)(1.000,1.000){2}{\rule{0.241pt}{0.800pt}}
\put(153.0,485.0){\rule[-0.400pt]{48.662pt}{0.800pt}}
\put(449,484.84){\rule{0.241pt}{0.800pt}}
\multiput(449.00,484.34)(0.500,1.000){2}{\rule{0.120pt}{0.800pt}}
\put(357.0,486.0){\rule[-0.400pt]{22.163pt}{0.800pt}}
\put(504,485.84){\rule{0.241pt}{0.800pt}}
\multiput(504.00,485.34)(0.500,1.000){2}{\rule{0.120pt}{0.800pt}}
\put(450.0,487.0){\rule[-0.400pt]{13.009pt}{0.800pt}}
\put(544,486.84){\rule{0.241pt}{0.800pt}}
\multiput(544.00,486.34)(0.500,1.000){2}{\rule{0.120pt}{0.800pt}}
\put(505.0,488.0){\rule[-0.400pt]{9.395pt}{0.800pt}}
\put(576,487.84){\rule{0.482pt}{0.800pt}}
\multiput(576.00,487.34)(1.000,1.000){2}{\rule{0.241pt}{0.800pt}}
\put(545.0,489.0){\rule[-0.400pt]{7.468pt}{0.800pt}}
\put(603,488.84){\rule{0.482pt}{0.800pt}}
\multiput(603.00,488.34)(1.000,1.000){2}{\rule{0.241pt}{0.800pt}}
\put(578.0,490.0){\rule[-0.400pt]{6.022pt}{0.800pt}}
\put(626,489.84){\rule{0.482pt}{0.800pt}}
\multiput(626.00,489.34)(1.000,1.000){2}{\rule{0.241pt}{0.800pt}}
\put(605.0,491.0){\rule[-0.400pt]{5.059pt}{0.800pt}}
\put(647,490.84){\rule{0.241pt}{0.800pt}}
\multiput(647.00,490.34)(0.500,1.000){2}{\rule{0.120pt}{0.800pt}}
\put(628.0,492.0){\rule[-0.400pt]{4.577pt}{0.800pt}}
\put(665,491.84){\rule{0.241pt}{0.800pt}}
\multiput(665.00,491.34)(0.500,1.000){2}{\rule{0.120pt}{0.800pt}}
\put(648.0,493.0){\rule[-0.400pt]{4.095pt}{0.800pt}}
\put(681,492.84){\rule{0.241pt}{0.800pt}}
\multiput(681.00,492.34)(0.500,1.000){2}{\rule{0.120pt}{0.800pt}}
\put(666.0,494.0){\rule[-0.400pt]{3.613pt}{0.800pt}}
\put(695,493.84){\rule{0.482pt}{0.800pt}}
\multiput(695.00,493.34)(1.000,1.000){2}{\rule{0.241pt}{0.800pt}}
\put(682.0,495.0){\rule[-0.400pt]{3.132pt}{0.800pt}}
\put(708,494.84){\rule{0.482pt}{0.800pt}}
\multiput(708.00,494.34)(1.000,1.000){2}{\rule{0.241pt}{0.800pt}}
\put(697.0,496.0){\rule[-0.400pt]{2.650pt}{0.800pt}}
\put(721,495.84){\rule{0.482pt}{0.800pt}}
\multiput(721.00,495.34)(1.000,1.000){2}{\rule{0.241pt}{0.800pt}}
\put(710.0,497.0){\rule[-0.400pt]{2.650pt}{0.800pt}}
\put(732,496.84){\rule{0.482pt}{0.800pt}}
\multiput(732.00,496.34)(1.000,1.000){2}{\rule{0.241pt}{0.800pt}}
\put(723.0,498.0){\rule[-0.400pt]{2.168pt}{0.800pt}}
\put(743,497.84){\rule{0.482pt}{0.800pt}}
\multiput(743.00,497.34)(1.000,1.000){2}{\rule{0.241pt}{0.800pt}}
\put(734.0,499.0){\rule[-0.400pt]{2.168pt}{0.800pt}}
\put(755,498.84){\rule{0.241pt}{0.800pt}}
\multiput(755.00,498.34)(0.500,1.000){2}{\rule{0.120pt}{0.800pt}}
\put(745.0,500.0){\rule[-0.400pt]{2.409pt}{0.800pt}}
\put(764,499.84){\rule{0.482pt}{0.800pt}}
\multiput(764.00,499.34)(1.000,1.000){2}{\rule{0.241pt}{0.800pt}}
\put(756.0,501.0){\rule[-0.400pt]{1.927pt}{0.800pt}}
\put(772,500.84){\rule{0.482pt}{0.800pt}}
\multiput(772.00,500.34)(1.000,1.000){2}{\rule{0.241pt}{0.800pt}}
\put(766.0,502.0){\rule[-0.400pt]{1.445pt}{0.800pt}}
\put(782,501.84){\rule{0.482pt}{0.800pt}}
\multiput(782.00,501.34)(1.000,1.000){2}{\rule{0.241pt}{0.800pt}}
\put(774.0,503.0){\rule[-0.400pt]{1.927pt}{0.800pt}}
\put(790,502.84){\rule{0.482pt}{0.800pt}}
\multiput(790.00,502.34)(1.000,1.000){2}{\rule{0.241pt}{0.800pt}}
\put(784.0,504.0){\rule[-0.400pt]{1.445pt}{0.800pt}}
\put(798,503.84){\rule{0.482pt}{0.800pt}}
\multiput(798.00,503.34)(1.000,1.000){2}{\rule{0.241pt}{0.800pt}}
\put(792.0,505.0){\rule[-0.400pt]{1.445pt}{0.800pt}}
\put(805,504.84){\rule{0.241pt}{0.800pt}}
\multiput(805.00,504.34)(0.500,1.000){2}{\rule{0.120pt}{0.800pt}}
\put(800.0,506.0){\rule[-0.400pt]{1.204pt}{0.800pt}}
\put(813,505.84){\rule{0.241pt}{0.800pt}}
\multiput(813.00,505.34)(0.500,1.000){2}{\rule{0.120pt}{0.800pt}}
\put(806.0,507.0){\rule[-0.400pt]{1.686pt}{0.800pt}}
\put(819,506.84){\rule{0.482pt}{0.800pt}}
\multiput(819.00,506.34)(1.000,1.000){2}{\rule{0.241pt}{0.800pt}}
\put(814.0,508.0){\rule[-0.400pt]{1.204pt}{0.800pt}}
\put(826,507.84){\rule{0.241pt}{0.800pt}}
\multiput(826.00,507.34)(0.500,1.000){2}{\rule{0.120pt}{0.800pt}}
\put(821.0,509.0){\rule[-0.400pt]{1.204pt}{0.800pt}}
\put(832,508.84){\rule{0.482pt}{0.800pt}}
\multiput(832.00,508.34)(1.000,1.000){2}{\rule{0.241pt}{0.800pt}}
\put(827.0,510.0){\rule[-0.400pt]{1.204pt}{0.800pt}}
\put(838,509.84){\rule{0.482pt}{0.800pt}}
\multiput(838.00,509.34)(1.000,1.000){2}{\rule{0.241pt}{0.800pt}}
\put(834.0,511.0){\rule[-0.400pt]{0.964pt}{0.800pt}}
\put(843,510.84){\rule{0.482pt}{0.800pt}}
\multiput(843.00,510.34)(1.000,1.000){2}{\rule{0.241pt}{0.800pt}}
\put(840.0,512.0){\usebox{\plotpoint}}
\put(850,511.84){\rule{0.241pt}{0.800pt}}
\multiput(850.00,511.34)(0.500,1.000){2}{\rule{0.120pt}{0.800pt}}
\put(845.0,513.0){\rule[-0.400pt]{1.204pt}{0.800pt}}
\put(855,512.84){\rule{0.241pt}{0.800pt}}
\multiput(855.00,512.34)(0.500,1.000){2}{\rule{0.120pt}{0.800pt}}
\put(851.0,514.0){\rule[-0.400pt]{0.964pt}{0.800pt}}
\put(861,513.84){\rule{0.482pt}{0.800pt}}
\multiput(861.00,513.34)(1.000,1.000){2}{\rule{0.241pt}{0.800pt}}
\put(856.0,515.0){\rule[-0.400pt]{1.204pt}{0.800pt}}
\put(866,514.84){\rule{0.241pt}{0.800pt}}
\multiput(866.00,514.34)(0.500,1.000){2}{\rule{0.120pt}{0.800pt}}
\put(863.0,516.0){\usebox{\plotpoint}}
\put(871,515.84){\rule{0.241pt}{0.800pt}}
\multiput(871.00,515.34)(0.500,1.000){2}{\rule{0.120pt}{0.800pt}}
\put(867.0,517.0){\rule[-0.400pt]{0.964pt}{0.800pt}}
\put(876,516.84){\rule{0.241pt}{0.800pt}}
\multiput(876.00,516.34)(0.500,1.000){2}{\rule{0.120pt}{0.800pt}}
\put(872.0,518.0){\rule[-0.400pt]{0.964pt}{0.800pt}}
\put(880,517.84){\rule{0.482pt}{0.800pt}}
\multiput(880.00,517.34)(1.000,1.000){2}{\rule{0.241pt}{0.800pt}}
\put(877.0,519.0){\usebox{\plotpoint}}
\put(885,518.84){\rule{0.482pt}{0.800pt}}
\multiput(885.00,518.34)(1.000,1.000){2}{\rule{0.241pt}{0.800pt}}
\put(882.0,520.0){\usebox{\plotpoint}}
\put(888,519.84){\rule{0.482pt}{0.800pt}}
\multiput(888.00,519.34)(1.000,1.000){2}{\rule{0.241pt}{0.800pt}}
\put(887.0,521.0){\usebox{\plotpoint}}
\put(893,520.84){\rule{0.482pt}{0.800pt}}
\multiput(893.00,520.34)(1.000,1.000){2}{\rule{0.241pt}{0.800pt}}
\put(890.0,522.0){\usebox{\plotpoint}}
\put(898,521.84){\rule{0.482pt}{0.800pt}}
\multiput(898.00,521.34)(1.000,1.000){2}{\rule{0.241pt}{0.800pt}}
\put(895.0,523.0){\usebox{\plotpoint}}
\put(901,522.84){\rule{0.482pt}{0.800pt}}
\multiput(901.00,522.34)(1.000,1.000){2}{\rule{0.241pt}{0.800pt}}
\put(900.0,524.0){\usebox{\plotpoint}}
\put(906,523.84){\rule{0.482pt}{0.800pt}}
\multiput(906.00,523.34)(1.000,1.000){2}{\rule{0.241pt}{0.800pt}}
\put(903.0,525.0){\usebox{\plotpoint}}
\put(909,524.84){\rule{0.482pt}{0.800pt}}
\multiput(909.00,524.34)(1.000,1.000){2}{\rule{0.241pt}{0.800pt}}
\put(908.0,526.0){\usebox{\plotpoint}}
\put(913,525.84){\rule{0.241pt}{0.800pt}}
\multiput(913.00,525.34)(0.500,1.000){2}{\rule{0.120pt}{0.800pt}}
\put(911.0,527.0){\usebox{\plotpoint}}
\put(917,526.84){\rule{0.482pt}{0.800pt}}
\multiput(917.00,526.34)(1.000,1.000){2}{\rule{0.241pt}{0.800pt}}
\put(914.0,528.0){\usebox{\plotpoint}}
\put(921,527.84){\rule{0.241pt}{0.800pt}}
\multiput(921.00,527.34)(0.500,1.000){2}{\rule{0.120pt}{0.800pt}}
\put(919.0,529.0){\usebox{\plotpoint}}
\put(924,528.84){\rule{0.241pt}{0.800pt}}
\multiput(924.00,528.34)(0.500,1.000){2}{\rule{0.120pt}{0.800pt}}
\put(922.0,530.0){\usebox{\plotpoint}}
\put(927,529.84){\rule{0.482pt}{0.800pt}}
\multiput(927.00,529.34)(1.000,1.000){2}{\rule{0.241pt}{0.800pt}}
\put(925.0,531.0){\usebox{\plotpoint}}
\put(930,530.84){\rule{0.482pt}{0.800pt}}
\multiput(930.00,530.34)(1.000,1.000){2}{\rule{0.241pt}{0.800pt}}
\put(929.0,532.0){\usebox{\plotpoint}}
\put(933,531.84){\rule{0.482pt}{0.800pt}}
\multiput(933.00,531.34)(1.000,1.000){2}{\rule{0.241pt}{0.800pt}}
\put(932.0,533.0){\usebox{\plotpoint}}
\put(937,532.84){\rule{0.241pt}{0.800pt}}
\multiput(937.00,532.34)(0.500,1.000){2}{\rule{0.120pt}{0.800pt}}
\put(935.0,534.0){\usebox{\plotpoint}}
\put(940,533.84){\rule{0.482pt}{0.800pt}}
\multiput(940.00,533.34)(1.000,1.000){2}{\rule{0.241pt}{0.800pt}}
\put(938.0,535.0){\usebox{\plotpoint}}
\put(943,534.84){\rule{0.482pt}{0.800pt}}
\multiput(943.00,534.34)(1.000,1.000){2}{\rule{0.241pt}{0.800pt}}
\put(942.0,536.0){\usebox{\plotpoint}}
\put(946,535.84){\rule{0.482pt}{0.800pt}}
\multiput(946.00,535.34)(1.000,1.000){2}{\rule{0.241pt}{0.800pt}}
\put(945.0,537.0){\usebox{\plotpoint}}
\put(950,536.84){\rule{0.241pt}{0.800pt}}
\multiput(950.00,536.34)(0.500,1.000){2}{\rule{0.120pt}{0.800pt}}
\put(948.0,538.0){\usebox{\plotpoint}}
\put(953,537.84){\rule{0.241pt}{0.800pt}}
\multiput(953.00,537.34)(0.500,1.000){2}{\rule{0.120pt}{0.800pt}}
\put(951.0,539.0){\usebox{\plotpoint}}
\put(956,538.84){\rule{0.482pt}{0.800pt}}
\multiput(956.00,538.34)(1.000,1.000){2}{\rule{0.241pt}{0.800pt}}
\put(958,539.84){\rule{0.241pt}{0.800pt}}
\multiput(958.00,539.34)(0.500,1.000){2}{\rule{0.120pt}{0.800pt}}
\put(954.0,540.0){\usebox{\plotpoint}}
\put(961,540.84){\rule{0.241pt}{0.800pt}}
\multiput(961.00,540.34)(0.500,1.000){2}{\rule{0.120pt}{0.800pt}}
\put(959.0,542.0){\usebox{\plotpoint}}
\put(964,541.84){\rule{0.482pt}{0.800pt}}
\multiput(964.00,541.34)(1.000,1.000){2}{\rule{0.241pt}{0.800pt}}
\put(966,542.84){\rule{0.241pt}{0.800pt}}
\multiput(966.00,542.34)(0.500,1.000){2}{\rule{0.120pt}{0.800pt}}
\put(962.0,543.0){\usebox{\plotpoint}}
\put(969,543.84){\rule{0.241pt}{0.800pt}}
\multiput(969.00,543.34)(0.500,1.000){2}{\rule{0.120pt}{0.800pt}}
\put(967.0,545.0){\usebox{\plotpoint}}
\put(972,544.84){\rule{0.482pt}{0.800pt}}
\multiput(972.00,544.34)(1.000,1.000){2}{\rule{0.241pt}{0.800pt}}
\put(974,545.84){\rule{0.241pt}{0.800pt}}
\multiput(974.00,545.34)(0.500,1.000){2}{\rule{0.120pt}{0.800pt}}
\put(970.0,546.0){\usebox{\plotpoint}}
\put(977,546.84){\rule{0.482pt}{0.800pt}}
\multiput(977.00,546.34)(1.000,1.000){2}{\rule{0.241pt}{0.800pt}}
\put(979,547.84){\rule{0.241pt}{0.800pt}}
\multiput(979.00,547.34)(0.500,1.000){2}{\rule{0.120pt}{0.800pt}}
\put(975.0,548.0){\usebox{\plotpoint}}
\put(982,548.84){\rule{0.241pt}{0.800pt}}
\multiput(982.00,548.34)(0.500,1.000){2}{\rule{0.120pt}{0.800pt}}
\put(983,549.84){\rule{0.482pt}{0.800pt}}
\multiput(983.00,549.34)(1.000,1.000){2}{\rule{0.241pt}{0.800pt}}
\put(980.0,550.0){\usebox{\plotpoint}}
\put(987,550.84){\rule{0.241pt}{0.800pt}}
\multiput(987.00,550.34)(0.500,1.000){2}{\rule{0.120pt}{0.800pt}}
\put(988,551.84){\rule{0.482pt}{0.800pt}}
\multiput(988.00,551.34)(1.000,1.000){2}{\rule{0.241pt}{0.800pt}}
\put(990,552.84){\rule{0.241pt}{0.800pt}}
\multiput(990.00,552.34)(0.500,1.000){2}{\rule{0.120pt}{0.800pt}}
\put(985.0,552.0){\usebox{\plotpoint}}
\put(993,553.84){\rule{0.482pt}{0.800pt}}
\multiput(993.00,553.34)(1.000,1.000){2}{\rule{0.241pt}{0.800pt}}
\put(995,554.84){\rule{0.241pt}{0.800pt}}
\multiput(995.00,554.34)(0.500,1.000){2}{\rule{0.120pt}{0.800pt}}
\put(996,555.84){\rule{0.482pt}{0.800pt}}
\multiput(996.00,555.34)(1.000,1.000){2}{\rule{0.241pt}{0.800pt}}
\put(991.0,555.0){\usebox{\plotpoint}}
\put(999,556.84){\rule{0.482pt}{0.800pt}}
\multiput(999.00,556.34)(1.000,1.000){2}{\rule{0.241pt}{0.800pt}}
\put(1001,557.84){\rule{0.482pt}{0.800pt}}
\multiput(1001.00,557.34)(1.000,1.000){2}{\rule{0.241pt}{0.800pt}}
\put(1003,558.84){\rule{0.241pt}{0.800pt}}
\multiput(1003.00,558.34)(0.500,1.000){2}{\rule{0.120pt}{0.800pt}}
\put(998.0,558.0){\usebox{\plotpoint}}
\put(1006,559.84){\rule{0.482pt}{0.800pt}}
\multiput(1006.00,559.34)(1.000,1.000){2}{\rule{0.241pt}{0.800pt}}
\put(1008,560.84){\rule{0.241pt}{0.800pt}}
\multiput(1008.00,560.34)(0.500,1.000){2}{\rule{0.120pt}{0.800pt}}
\put(1009,561.84){\rule{0.482pt}{0.800pt}}
\multiput(1009.00,561.34)(1.000,1.000){2}{\rule{0.241pt}{0.800pt}}
\put(1011,562.84){\rule{0.241pt}{0.800pt}}
\multiput(1011.00,562.34)(0.500,1.000){2}{\rule{0.120pt}{0.800pt}}
\put(1012,563.84){\rule{0.482pt}{0.800pt}}
\multiput(1012.00,563.34)(1.000,1.000){2}{\rule{0.241pt}{0.800pt}}
\put(1004.0,561.0){\usebox{\plotpoint}}
\put(1016,564.84){\rule{0.241pt}{0.800pt}}
\multiput(1016.00,564.34)(0.500,1.000){2}{\rule{0.120pt}{0.800pt}}
\put(1017,565.84){\rule{0.482pt}{0.800pt}}
\multiput(1017.00,565.34)(1.000,1.000){2}{\rule{0.241pt}{0.800pt}}
\put(1019,566.84){\rule{0.241pt}{0.800pt}}
\multiput(1019.00,566.34)(0.500,1.000){2}{\rule{0.120pt}{0.800pt}}
\put(1020,567.84){\rule{0.482pt}{0.800pt}}
\multiput(1020.00,567.34)(1.000,1.000){2}{\rule{0.241pt}{0.800pt}}
\put(1022,568.84){\rule{0.482pt}{0.800pt}}
\multiput(1022.00,568.34)(1.000,1.000){2}{\rule{0.241pt}{0.800pt}}
\put(1024,569.84){\rule{0.241pt}{0.800pt}}
\multiput(1024.00,569.34)(0.500,1.000){2}{\rule{0.120pt}{0.800pt}}
\put(1025,570.84){\rule{0.482pt}{0.800pt}}
\multiput(1025.00,570.34)(1.000,1.000){2}{\rule{0.241pt}{0.800pt}}
\put(1027,571.84){\rule{0.241pt}{0.800pt}}
\multiput(1027.00,571.34)(0.500,1.000){2}{\rule{0.120pt}{0.800pt}}
\put(1028,572.84){\rule{0.482pt}{0.800pt}}
\multiput(1028.00,572.34)(1.000,1.000){2}{\rule{0.241pt}{0.800pt}}
\put(1030,573.84){\rule{0.482pt}{0.800pt}}
\multiput(1030.00,573.34)(1.000,1.000){2}{\rule{0.241pt}{0.800pt}}
\put(1032,574.84){\rule{0.241pt}{0.800pt}}
\multiput(1032.00,574.34)(0.500,1.000){2}{\rule{0.120pt}{0.800pt}}
\put(1033,575.84){\rule{0.482pt}{0.800pt}}
\multiput(1033.00,575.34)(1.000,1.000){2}{\rule{0.241pt}{0.800pt}}
\put(1035,576.84){\rule{0.482pt}{0.800pt}}
\multiput(1035.00,576.34)(1.000,1.000){2}{\rule{0.241pt}{0.800pt}}
\put(1037,577.84){\rule{0.241pt}{0.800pt}}
\multiput(1037.00,577.34)(0.500,1.000){2}{\rule{0.120pt}{0.800pt}}
\put(1038,578.84){\rule{0.482pt}{0.800pt}}
\multiput(1038.00,578.34)(1.000,1.000){2}{\rule{0.241pt}{0.800pt}}
\put(1040,579.84){\rule{0.241pt}{0.800pt}}
\multiput(1040.00,579.34)(0.500,1.000){2}{\rule{0.120pt}{0.800pt}}
\put(1041,580.84){\rule{0.482pt}{0.800pt}}
\multiput(1041.00,580.34)(1.000,1.000){2}{\rule{0.241pt}{0.800pt}}
\put(1043,581.84){\rule{0.482pt}{0.800pt}}
\multiput(1043.00,581.34)(1.000,1.000){2}{\rule{0.241pt}{0.800pt}}
\put(1045,582.84){\rule{0.241pt}{0.800pt}}
\multiput(1045.00,582.34)(0.500,1.000){2}{\rule{0.120pt}{0.800pt}}
\put(1046,583.84){\rule{0.482pt}{0.800pt}}
\multiput(1046.00,583.34)(1.000,1.000){2}{\rule{0.241pt}{0.800pt}}
\put(1048,584.84){\rule{0.241pt}{0.800pt}}
\multiput(1048.00,584.34)(0.500,1.000){2}{\rule{0.120pt}{0.800pt}}
\put(1049,585.84){\rule{0.482pt}{0.800pt}}
\multiput(1049.00,585.34)(1.000,1.000){2}{\rule{0.241pt}{0.800pt}}
\put(1051,586.84){\rule{0.482pt}{0.800pt}}
\multiput(1051.00,586.34)(1.000,1.000){2}{\rule{0.241pt}{0.800pt}}
\put(1051.84,589){\rule{0.800pt}{0.482pt}}
\multiput(1051.34,589.00)(1.000,1.000){2}{\rule{0.800pt}{0.241pt}}
\put(1054,589.84){\rule{0.482pt}{0.800pt}}
\multiput(1054.00,589.34)(1.000,1.000){2}{\rule{0.241pt}{0.800pt}}
\put(1056,590.84){\rule{0.241pt}{0.800pt}}
\multiput(1056.00,590.34)(0.500,1.000){2}{\rule{0.120pt}{0.800pt}}
\put(1057,591.84){\rule{0.482pt}{0.800pt}}
\multiput(1057.00,591.34)(1.000,1.000){2}{\rule{0.241pt}{0.800pt}}
\put(1059,592.84){\rule{0.482pt}{0.800pt}}
\multiput(1059.00,592.34)(1.000,1.000){2}{\rule{0.241pt}{0.800pt}}
\put(1059.84,595){\rule{0.800pt}{0.482pt}}
\multiput(1059.34,595.00)(1.000,1.000){2}{\rule{0.800pt}{0.241pt}}
\put(1062,595.84){\rule{0.482pt}{0.800pt}}
\multiput(1062.00,595.34)(1.000,1.000){2}{\rule{0.241pt}{0.800pt}}
\put(1064,596.84){\rule{0.241pt}{0.800pt}}
\multiput(1064.00,596.34)(0.500,1.000){2}{\rule{0.120pt}{0.800pt}}
\put(1065,597.84){\rule{0.482pt}{0.800pt}}
\multiput(1065.00,597.34)(1.000,1.000){2}{\rule{0.241pt}{0.800pt}}
\put(1067,599.34){\rule{0.482pt}{0.800pt}}
\multiput(1067.00,598.34)(1.000,2.000){2}{\rule{0.241pt}{0.800pt}}
\put(1069,600.84){\rule{0.241pt}{0.800pt}}
\multiput(1069.00,600.34)(0.500,1.000){2}{\rule{0.120pt}{0.800pt}}
\put(1070,601.84){\rule{0.482pt}{0.800pt}}
\multiput(1070.00,601.34)(1.000,1.000){2}{\rule{0.241pt}{0.800pt}}
\put(1072,603.34){\rule{0.482pt}{0.800pt}}
\multiput(1072.00,602.34)(1.000,2.000){2}{\rule{0.241pt}{0.800pt}}
\put(1074,604.84){\rule{0.241pt}{0.800pt}}
\multiput(1074.00,604.34)(0.500,1.000){2}{\rule{0.120pt}{0.800pt}}
\put(1075,605.84){\rule{0.482pt}{0.800pt}}
\multiput(1075.00,605.34)(1.000,1.000){2}{\rule{0.241pt}{0.800pt}}
\put(1075.84,608){\rule{0.800pt}{0.482pt}}
\multiput(1075.34,608.00)(1.000,1.000){2}{\rule{0.800pt}{0.241pt}}
\put(1078,608.84){\rule{0.482pt}{0.800pt}}
\multiput(1078.00,608.34)(1.000,1.000){2}{\rule{0.241pt}{0.800pt}}
\put(1080,610.34){\rule{0.482pt}{0.800pt}}
\multiput(1080.00,609.34)(1.000,2.000){2}{\rule{0.241pt}{0.800pt}}
\put(1082,611.84){\rule{0.241pt}{0.800pt}}
\multiput(1082.00,611.34)(0.500,1.000){2}{\rule{0.120pt}{0.800pt}}
\put(1083,613.34){\rule{0.482pt}{0.800pt}}
\multiput(1083.00,612.34)(1.000,2.000){2}{\rule{0.241pt}{0.800pt}}
\put(1085,614.84){\rule{0.241pt}{0.800pt}}
\multiput(1085.00,614.34)(0.500,1.000){2}{\rule{0.120pt}{0.800pt}}
\put(1086,616.34){\rule{0.482pt}{0.800pt}}
\multiput(1086.00,615.34)(1.000,2.000){2}{\rule{0.241pt}{0.800pt}}
\put(1088,617.84){\rule{0.482pt}{0.800pt}}
\multiput(1088.00,617.34)(1.000,1.000){2}{\rule{0.241pt}{0.800pt}}
\put(1088.84,620){\rule{0.800pt}{0.482pt}}
\multiput(1088.34,620.00)(1.000,1.000){2}{\rule{0.800pt}{0.241pt}}
\put(1091,620.84){\rule{0.482pt}{0.800pt}}
\multiput(1091.00,620.34)(1.000,1.000){2}{\rule{0.241pt}{0.800pt}}
\put(1091.84,623){\rule{0.800pt}{0.482pt}}
\multiput(1091.34,623.00)(1.000,1.000){2}{\rule{0.800pt}{0.241pt}}
\put(1094,624.34){\rule{0.482pt}{0.800pt}}
\multiput(1094.00,623.34)(1.000,2.000){2}{\rule{0.241pt}{0.800pt}}
\put(1096,625.84){\rule{0.482pt}{0.800pt}}
\multiput(1096.00,625.34)(1.000,1.000){2}{\rule{0.241pt}{0.800pt}}
\put(1096.84,628){\rule{0.800pt}{0.482pt}}
\multiput(1096.34,628.00)(1.000,1.000){2}{\rule{0.800pt}{0.241pt}}
\put(1099,629.34){\rule{0.482pt}{0.800pt}}
\multiput(1099.00,628.34)(1.000,2.000){2}{\rule{0.241pt}{0.800pt}}
\put(1101,630.84){\rule{0.482pt}{0.800pt}}
\multiput(1101.00,630.34)(1.000,1.000){2}{\rule{0.241pt}{0.800pt}}
\put(1101.84,633){\rule{0.800pt}{0.482pt}}
\multiput(1101.34,633.00)(1.000,1.000){2}{\rule{0.800pt}{0.241pt}}
\put(1104,634.34){\rule{0.482pt}{0.800pt}}
\multiput(1104.00,633.34)(1.000,2.000){2}{\rule{0.241pt}{0.800pt}}
\put(1104.84,637){\rule{0.800pt}{0.482pt}}
\multiput(1104.34,637.00)(1.000,1.000){2}{\rule{0.800pt}{0.241pt}}
\put(1107,637.84){\rule{0.482pt}{0.800pt}}
\multiput(1107.00,637.34)(1.000,1.000){2}{\rule{0.241pt}{0.800pt}}
\put(1109,639.34){\rule{0.482pt}{0.800pt}}
\multiput(1109.00,638.34)(1.000,2.000){2}{\rule{0.241pt}{0.800pt}}
\put(1109.84,642){\rule{0.800pt}{0.482pt}}
\multiput(1109.34,642.00)(1.000,1.000){2}{\rule{0.800pt}{0.241pt}}
\put(1112,643.34){\rule{0.482pt}{0.800pt}}
\multiput(1112.00,642.34)(1.000,2.000){2}{\rule{0.241pt}{0.800pt}}
\put(1112.84,646){\rule{0.800pt}{0.482pt}}
\multiput(1112.34,646.00)(1.000,1.000){2}{\rule{0.800pt}{0.241pt}}
\put(1115,647.34){\rule{0.482pt}{0.800pt}}
\multiput(1115.00,646.34)(1.000,2.000){2}{\rule{0.241pt}{0.800pt}}
\put(1117,649.34){\rule{0.482pt}{0.800pt}}
\multiput(1117.00,648.34)(1.000,2.000){2}{\rule{0.241pt}{0.800pt}}
\put(1117.84,652){\rule{0.800pt}{0.482pt}}
\multiput(1117.34,652.00)(1.000,1.000){2}{\rule{0.800pt}{0.241pt}}
\put(1120,653.34){\rule{0.482pt}{0.800pt}}
\multiput(1120.00,652.34)(1.000,2.000){2}{\rule{0.241pt}{0.800pt}}
\put(1120.84,656){\rule{0.800pt}{0.482pt}}
\multiput(1120.34,656.00)(1.000,1.000){2}{\rule{0.800pt}{0.241pt}}
\put(1123,657.34){\rule{0.482pt}{0.800pt}}
\multiput(1123.00,656.34)(1.000,2.000){2}{\rule{0.241pt}{0.800pt}}
\put(1125,659.34){\rule{0.482pt}{0.800pt}}
\multiput(1125.00,658.34)(1.000,2.000){2}{\rule{0.241pt}{0.800pt}}
\put(1125.84,662){\rule{0.800pt}{0.723pt}}
\multiput(1125.34,662.00)(1.000,1.500){2}{\rule{0.800pt}{0.361pt}}
\put(1128,664.34){\rule{0.482pt}{0.800pt}}
\multiput(1128.00,663.34)(1.000,2.000){2}{\rule{0.241pt}{0.800pt}}
\put(1128.84,667){\rule{0.800pt}{0.482pt}}
\multiput(1128.34,667.00)(1.000,1.000){2}{\rule{0.800pt}{0.241pt}}
\put(1131,668.34){\rule{0.482pt}{0.800pt}}
\multiput(1131.00,667.34)(1.000,2.000){2}{\rule{0.241pt}{0.800pt}}
\put(1132.34,671){\rule{0.800pt}{0.723pt}}
\multiput(1131.34,671.00)(2.000,1.500){2}{\rule{0.800pt}{0.361pt}}
\put(1133.84,674){\rule{0.800pt}{0.482pt}}
\multiput(1133.34,674.00)(1.000,1.000){2}{\rule{0.800pt}{0.241pt}}
\put(1136,675.34){\rule{0.482pt}{0.800pt}}
\multiput(1136.00,674.34)(1.000,2.000){2}{\rule{0.241pt}{0.800pt}}
\put(1137.34,678){\rule{0.800pt}{0.723pt}}
\multiput(1136.34,678.00)(2.000,1.500){2}{\rule{0.800pt}{0.361pt}}
\put(1138.84,681){\rule{0.800pt}{0.482pt}}
\multiput(1138.34,681.00)(1.000,1.000){2}{\rule{0.800pt}{0.241pt}}
\put(1140.34,683){\rule{0.800pt}{0.723pt}}
\multiput(1139.34,683.00)(2.000,1.500){2}{\rule{0.800pt}{0.361pt}}
\put(1141.84,686){\rule{0.800pt}{0.482pt}}
\multiput(1141.34,686.00)(1.000,1.000){2}{\rule{0.800pt}{0.241pt}}
\put(1143.34,688){\rule{0.800pt}{0.723pt}}
\multiput(1142.34,688.00)(2.000,1.500){2}{\rule{0.800pt}{0.361pt}}
\put(1145.34,691){\rule{0.800pt}{0.723pt}}
\multiput(1144.34,691.00)(2.000,1.500){2}{\rule{0.800pt}{0.361pt}}
\put(1146.84,694){\rule{0.800pt}{0.482pt}}
\multiput(1146.34,694.00)(1.000,1.000){2}{\rule{0.800pt}{0.241pt}}
\put(1148.34,696){\rule{0.800pt}{0.723pt}}
\multiput(1147.34,696.00)(2.000,1.500){2}{\rule{0.800pt}{0.361pt}}
\put(1149.84,699){\rule{0.800pt}{0.723pt}}
\multiput(1149.34,699.00)(1.000,1.500){2}{\rule{0.800pt}{0.361pt}}
\put(1151.34,702){\rule{0.800pt}{0.723pt}}
\multiput(1150.34,702.00)(2.000,1.500){2}{\rule{0.800pt}{0.361pt}}
\put(1154,704.34){\rule{0.482pt}{0.800pt}}
\multiput(1154.00,703.34)(1.000,2.000){2}{\rule{0.241pt}{0.800pt}}
\put(1154.84,707){\rule{0.800pt}{0.723pt}}
\multiput(1154.34,707.00)(1.000,1.500){2}{\rule{0.800pt}{0.361pt}}
\put(1156.34,710){\rule{0.800pt}{0.723pt}}
\multiput(1155.34,710.00)(2.000,1.500){2}{\rule{0.800pt}{0.361pt}}
\put(1157.84,713){\rule{0.800pt}{0.723pt}}
\multiput(1157.34,713.00)(1.000,1.500){2}{\rule{0.800pt}{0.361pt}}
\put(1159.34,716){\rule{0.800pt}{0.723pt}}
\multiput(1158.34,716.00)(2.000,1.500){2}{\rule{0.800pt}{0.361pt}}
\put(1161.34,719){\rule{0.800pt}{0.723pt}}
\multiput(1160.34,719.00)(2.000,1.500){2}{\rule{0.800pt}{0.361pt}}
\put(1162.84,722){\rule{0.800pt}{0.964pt}}
\multiput(1162.34,722.00)(1.000,2.000){2}{\rule{0.800pt}{0.482pt}}
\put(1164.34,726){\rule{0.800pt}{0.723pt}}
\multiput(1163.34,726.00)(2.000,1.500){2}{\rule{0.800pt}{0.361pt}}
\put(1166.34,729){\rule{0.800pt}{0.723pt}}
\multiput(1165.34,729.00)(2.000,1.500){2}{\rule{0.800pt}{0.361pt}}
\put(1167.84,732){\rule{0.800pt}{0.723pt}}
\multiput(1167.34,732.00)(1.000,1.500){2}{\rule{0.800pt}{0.361pt}}
\put(1169.34,735){\rule{0.800pt}{0.964pt}}
\multiput(1168.34,735.00)(2.000,2.000){2}{\rule{0.800pt}{0.482pt}}
\put(1170.84,739){\rule{0.800pt}{0.723pt}}
\multiput(1170.34,739.00)(1.000,1.500){2}{\rule{0.800pt}{0.361pt}}
\put(1172.34,742){\rule{0.800pt}{0.964pt}}
\multiput(1171.34,742.00)(2.000,2.000){2}{\rule{0.800pt}{0.482pt}}
\put(1174.34,746){\rule{0.800pt}{0.723pt}}
\multiput(1173.34,746.00)(2.000,1.500){2}{\rule{0.800pt}{0.361pt}}
\put(1175.84,749){\rule{0.800pt}{0.964pt}}
\multiput(1175.34,749.00)(1.000,2.000){2}{\rule{0.800pt}{0.482pt}}
\put(1177.34,753){\rule{0.800pt}{0.964pt}}
\multiput(1176.34,753.00)(2.000,2.000){2}{\rule{0.800pt}{0.482pt}}
\put(1178.84,757){\rule{0.800pt}{0.723pt}}
\multiput(1178.34,757.00)(1.000,1.500){2}{\rule{0.800pt}{0.361pt}}
\put(1180.34,760){\rule{0.800pt}{0.964pt}}
\multiput(1179.34,760.00)(2.000,2.000){2}{\rule{0.800pt}{0.482pt}}
\put(1182.34,764){\rule{0.800pt}{0.964pt}}
\multiput(1181.34,764.00)(2.000,2.000){2}{\rule{0.800pt}{0.482pt}}
\put(1183.84,768){\rule{0.800pt}{0.964pt}}
\multiput(1183.34,768.00)(1.000,2.000){2}{\rule{0.800pt}{0.482pt}}
\put(1185.34,772){\rule{0.800pt}{0.964pt}}
\multiput(1184.34,772.00)(2.000,2.000){2}{\rule{0.800pt}{0.482pt}}
\put(1186.84,776){\rule{0.800pt}{0.964pt}}
\multiput(1186.34,776.00)(1.000,2.000){2}{\rule{0.800pt}{0.482pt}}
\put(1188.34,780){\rule{0.800pt}{0.964pt}}
\multiput(1187.34,780.00)(2.000,2.000){2}{\rule{0.800pt}{0.482pt}}
\put(1190.34,784){\rule{0.800pt}{1.204pt}}
\multiput(1189.34,784.00)(2.000,2.500){2}{\rule{0.800pt}{0.602pt}}
\put(1191.84,789){\rule{0.800pt}{0.964pt}}
\multiput(1191.34,789.00)(1.000,2.000){2}{\rule{0.800pt}{0.482pt}}
\put(1193.34,793){\rule{0.800pt}{1.204pt}}
\multiput(1192.34,793.00)(2.000,2.500){2}{\rule{0.800pt}{0.602pt}}
\put(1195.34,798){\rule{0.800pt}{0.964pt}}
\multiput(1194.34,798.00)(2.000,2.000){2}{\rule{0.800pt}{0.482pt}}
\put(1196.84,802){\rule{0.800pt}{1.204pt}}
\multiput(1196.34,802.00)(1.000,2.500){2}{\rule{0.800pt}{0.602pt}}
\put(1198.34,807){\rule{0.800pt}{0.964pt}}
\multiput(1197.34,807.00)(2.000,2.000){2}{\rule{0.800pt}{0.482pt}}
\put(1199.84,811){\rule{0.800pt}{1.204pt}}
\multiput(1199.34,811.00)(1.000,2.500){2}{\rule{0.800pt}{0.602pt}}
\put(1201.34,816){\rule{0.800pt}{1.204pt}}
\multiput(1200.34,816.00)(2.000,2.500){2}{\rule{0.800pt}{0.602pt}}
\put(1203.34,821){\rule{0.800pt}{1.204pt}}
\multiput(1202.34,821.00)(2.000,2.500){2}{\rule{0.800pt}{0.602pt}}
\put(1204.84,826){\rule{0.800pt}{1.204pt}}
\multiput(1204.34,826.00)(1.000,2.500){2}{\rule{0.800pt}{0.602pt}}
\put(1206.34,831){\rule{0.800pt}{1.204pt}}
\multiput(1205.34,831.00)(2.000,2.500){2}{\rule{0.800pt}{0.602pt}}
\put(1207.84,836){\rule{0.800pt}{1.445pt}}
\multiput(1207.34,836.00)(1.000,3.000){2}{\rule{0.800pt}{0.723pt}}
\put(1209.34,842){\rule{0.800pt}{1.204pt}}
\multiput(1208.34,842.00)(2.000,2.500){2}{\rule{0.800pt}{0.602pt}}
\put(1211.34,847){\rule{0.800pt}{1.204pt}}
\multiput(1210.34,847.00)(2.000,2.500){2}{\rule{0.800pt}{0.602pt}}
\put(1212.84,852){\rule{0.800pt}{1.445pt}}
\multiput(1212.34,852.00)(1.000,3.000){2}{\rule{0.800pt}{0.723pt}}
\put(1014.0,566.0){\usebox{\plotpoint}}
\put(1215.0,858.0){\usebox{\plotpoint}}
\sbox{\plotpoint}{\rule[-0.500pt]{1.000pt}{1.000pt}}%
\sbox{\plotpoint}{\rule[-0.200pt]{0.400pt}{0.400pt}}%
\put(1279,695){\makebox(0,0)[r]{$w$}}
\sbox{\plotpoint}{\rule[-0.500pt]{1.000pt}{1.000pt}}%
\multiput(1299,695)(20.756,0.000){5}{\usebox{\plotpoint}}
\put(1399,695){\usebox{\plotpoint}}
\put(153,394){\usebox{\plotpoint}}
\put(153.00,394.00){\usebox{\plotpoint}}
\put(173.76,394.00){\usebox{\plotpoint}}
\put(194.51,394.00){\usebox{\plotpoint}}
\put(215.27,394.00){\usebox{\plotpoint}}
\put(236.02,394.00){\usebox{\plotpoint}}
\put(256.78,394.00){\usebox{\plotpoint}}
\put(277.53,394.00){\usebox{\plotpoint}}
\put(298.29,394.00){\usebox{\plotpoint}}
\put(319.04,394.00){\usebox{\plotpoint}}
\put(339.80,394.00){\usebox{\plotpoint}}
\put(360.55,394.00){\usebox{\plotpoint}}
\put(381.31,394.00){\usebox{\plotpoint}}
\put(402.07,394.00){\usebox{\plotpoint}}
\put(422.82,394.00){\usebox{\plotpoint}}
\put(443.58,394.00){\usebox{\plotpoint}}
\put(464.33,394.00){\usebox{\plotpoint}}
\put(485.09,394.00){\usebox{\plotpoint}}
\put(505.84,394.00){\usebox{\plotpoint}}
\put(526.60,394.00){\usebox{\plotpoint}}
\put(547.35,394.00){\usebox{\plotpoint}}
\put(568.11,394.00){\usebox{\plotpoint}}
\put(588.87,394.00){\usebox{\plotpoint}}
\put(609.62,394.00){\usebox{\plotpoint}}
\put(630.38,394.00){\usebox{\plotpoint}}
\put(650.90,393.00){\usebox{\plotpoint}}
\put(671.65,393.00){\usebox{\plotpoint}}
\put(692.41,393.00){\usebox{\plotpoint}}
\put(713.16,393.00){\usebox{\plotpoint}}
\put(733.92,393.00){\usebox{\plotpoint}}
\put(754.67,393.00){\usebox{\plotpoint}}
\put(775.28,392.36){\usebox{\plotpoint}}
\put(795.95,392.00){\usebox{\plotpoint}}
\put(816.70,392.00){\usebox{\plotpoint}}
\put(837.46,392.00){\usebox{\plotpoint}}
\put(857.80,391.00){\usebox{\plotpoint}}
\put(878.56,391.00){\usebox{\plotpoint}}
\put(898.90,390.00){\usebox{\plotpoint}}
\put(919.65,390.00){\usebox{\plotpoint}}
\put(939.99,389.00){\usebox{\plotpoint}}
\put(960.75,389.00){\usebox{\plotpoint}}
\put(981.27,388.00){\usebox{\plotpoint}}
\put(1001.79,387.00){\usebox{\plotpoint}}
\put(1022.31,386.00){\usebox{\plotpoint}}
\put(1042.83,385.00){\usebox{\plotpoint}}
\put(1062.75,383.00){\usebox{\plotpoint}}
\put(1083.10,382.00){\usebox{\plotpoint}}
\put(1103.20,380.00){\usebox{\plotpoint}}
\put(1123.31,378.00){\usebox{\plotpoint}}
\put(1143.59,376.00){\usebox{\plotpoint}}
\put(1163.46,373.00){\usebox{\plotpoint}}
\put(1183.33,370.00){\usebox{\plotpoint}}
\put(1202.96,366.00){\usebox{\plotpoint}}
\put(1222.29,361.71){\usebox{\plotpoint}}
\put(1241.40,356.00){\usebox{\plotpoint}}
\put(1260.18,349.91){\usebox{\plotpoint}}
\put(1278.53,342.73){\usebox{\plotpoint}}
\put(1296.49,333.51){\usebox{\plotpoint}}
\put(1313.33,321.84){\usebox{\plotpoint}}
\put(1329.51,309.25){\usebox{\plotpoint}}
\put(1342.80,294.20){\usebox{\plotpoint}}
\put(1355.14,277.78){\usebox{\plotpoint}}
\put(1365.58,260.13){\usebox{\plotpoint}}
\put(1374.65,241.69){\usebox{\plotpoint}}
\put(1381.70,222.26){\usebox{\plotpoint}}
\put(1387.54,202.38){\usebox{\plotpoint}}
\put(1393.02,182.42){\usebox{\plotpoint}}
\put(1397.69,162.24){\usebox{\plotpoint}}
\put(1401.58,141.88){\usebox{\plotpoint}}
\put(1403,131){\usebox{\plotpoint}}
\sbox{\plotpoint}{\rule[-0.200pt]{0.400pt}{0.400pt}}%
\put(151.0,131.0){\rule[-0.200pt]{0.400pt}{175.375pt}}
\put(151.0,131.0){\rule[-0.200pt]{310.279pt}{0.400pt}}
\put(1439.0,131.0){\rule[-0.200pt]{0.400pt}{175.375pt}}
\put(151.0,859.0){\rule[-0.200pt]{310.279pt}{0.400pt}}
\end{picture}

    \caption{Evolucion de las variables $z_1 , z_2 , z_3 , z_4$ de $0$ a $1.4$ desde el punto 
			inicial $(5.4083, 0.5547,-0.5547,-5.5470)$}
\end{figure}

Las gr\'aficas obtenidas para el espectro de Lyapunov son las siguientes

\begin{figure}[H]
    \centering
    % GNUPLOT: LaTeX picture
\setlength{\unitlength}{0.240900pt}
\ifx\plotpoint\undefined\newsavebox{\plotpoint}\fi
\sbox{\plotpoint}{\rule[-0.200pt]{0.400pt}{0.400pt}}%
\begin{picture}(1500,900)(0,0)
\sbox{\plotpoint}{\rule[-0.200pt]{0.400pt}{0.400pt}}%
\put(130.0,131.0){\rule[-0.200pt]{4.818pt}{0.400pt}}
\put(110,131){\makebox(0,0)[r]{$0.05$}}
\put(1419.0,131.0){\rule[-0.200pt]{4.818pt}{0.400pt}}
\put(130.0,235.0){\rule[-0.200pt]{4.818pt}{0.400pt}}
\put(110,235){\makebox(0,0)[r]{$0.06$}}
\put(1419.0,235.0){\rule[-0.200pt]{4.818pt}{0.400pt}}
\put(130.0,339.0){\rule[-0.200pt]{4.818pt}{0.400pt}}
\put(110,339){\makebox(0,0)[r]{$0.07$}}
\put(1419.0,339.0){\rule[-0.200pt]{4.818pt}{0.400pt}}
\put(130.0,443.0){\rule[-0.200pt]{4.818pt}{0.400pt}}
\put(110,443){\makebox(0,0)[r]{$0.08$}}
\put(1419.0,443.0){\rule[-0.200pt]{4.818pt}{0.400pt}}
\put(130.0,547.0){\rule[-0.200pt]{4.818pt}{0.400pt}}
\put(110,547){\makebox(0,0)[r]{$0.09$}}
\put(1419.0,547.0){\rule[-0.200pt]{4.818pt}{0.400pt}}
\put(130.0,651.0){\rule[-0.200pt]{4.818pt}{0.400pt}}
\put(110,651){\makebox(0,0)[r]{$0.1$}}
\put(1419.0,651.0){\rule[-0.200pt]{4.818pt}{0.400pt}}
\put(130.0,755.0){\rule[-0.200pt]{4.818pt}{0.400pt}}
\put(110,755){\makebox(0,0)[r]{$0.11$}}
\put(1419.0,755.0){\rule[-0.200pt]{4.818pt}{0.400pt}}
\put(130.0,859.0){\rule[-0.200pt]{4.818pt}{0.400pt}}
\put(110,859){\makebox(0,0)[r]{$0.12$}}
\put(1419.0,859.0){\rule[-0.200pt]{4.818pt}{0.400pt}}
\put(130.0,131.0){\rule[-0.200pt]{0.400pt}{4.818pt}}
\put(130,90){\makebox(0,0){$0$}}
\put(130.0,839.0){\rule[-0.200pt]{0.400pt}{4.818pt}}
\put(261.0,131.0){\rule[-0.200pt]{0.400pt}{4.818pt}}
\put(261,90){\makebox(0,0){$1000$}}
\put(261.0,839.0){\rule[-0.200pt]{0.400pt}{4.818pt}}
\put(392.0,131.0){\rule[-0.200pt]{0.400pt}{4.818pt}}
\put(392,90){\makebox(0,0){$2000$}}
\put(392.0,839.0){\rule[-0.200pt]{0.400pt}{4.818pt}}
\put(523.0,131.0){\rule[-0.200pt]{0.400pt}{4.818pt}}
\put(523,90){\makebox(0,0){$3000$}}
\put(523.0,839.0){\rule[-0.200pt]{0.400pt}{4.818pt}}
\put(654.0,131.0){\rule[-0.200pt]{0.400pt}{4.818pt}}
\put(654,90){\makebox(0,0){$4000$}}
\put(654.0,839.0){\rule[-0.200pt]{0.400pt}{4.818pt}}
\put(784.0,131.0){\rule[-0.200pt]{0.400pt}{4.818pt}}
\put(784,90){\makebox(0,0){$5000$}}
\put(784.0,839.0){\rule[-0.200pt]{0.400pt}{4.818pt}}
\put(915.0,131.0){\rule[-0.200pt]{0.400pt}{4.818pt}}
\put(915,90){\makebox(0,0){$6000$}}
\put(915.0,839.0){\rule[-0.200pt]{0.400pt}{4.818pt}}
\put(1046.0,131.0){\rule[-0.200pt]{0.400pt}{4.818pt}}
\put(1046,90){\makebox(0,0){$7000$}}
\put(1046.0,839.0){\rule[-0.200pt]{0.400pt}{4.818pt}}
\put(1177.0,131.0){\rule[-0.200pt]{0.400pt}{4.818pt}}
\put(1177,90){\makebox(0,0){$8000$}}
\put(1177.0,839.0){\rule[-0.200pt]{0.400pt}{4.818pt}}
\put(1308.0,131.0){\rule[-0.200pt]{0.400pt}{4.818pt}}
\put(1308,90){\makebox(0,0){$9000$}}
\put(1308.0,839.0){\rule[-0.200pt]{0.400pt}{4.818pt}}
\put(1439.0,131.0){\rule[-0.200pt]{0.400pt}{4.818pt}}
\put(1439,90){\makebox(0,0){$10000$}}
\put(1439.0,839.0){\rule[-0.200pt]{0.400pt}{4.818pt}}
\put(130.0,131.0){\rule[-0.200pt]{0.400pt}{175.375pt}}
\put(130.0,131.0){\rule[-0.200pt]{315.338pt}{0.400pt}}
\put(1439.0,131.0){\rule[-0.200pt]{0.400pt}{175.375pt}}
\put(130.0,859.0){\rule[-0.200pt]{315.338pt}{0.400pt}}
\put(784,29){\makebox(0,0){$t$}}
\put(1259,819){\makebox(0,0){Lyapunov exponents}}
\put(133,219){\usebox{\plotpoint}}
\multiput(133.60,219.00)(0.468,25.485){5}{\rule{0.113pt}{17.600pt}}
\multiput(132.17,219.00)(4.000,138.470){2}{\rule{0.400pt}{8.800pt}}
\multiput(137.61,394.00)(0.447,28.146){3}{\rule{0.108pt}{17.033pt}}
\multiput(136.17,394.00)(3.000,91.646){2}{\rule{0.400pt}{8.517pt}}
\multiput(140.61,521.00)(0.447,34.844){3}{\rule{0.108pt}{21.033pt}}
\multiput(139.17,521.00)(3.000,113.344){2}{\rule{0.400pt}{10.517pt}}
\multiput(143.61,660.98)(0.447,-6.490){3}{\rule{0.108pt}{4.100pt}}
\multiput(142.17,669.49)(3.000,-21.490){2}{\rule{0.400pt}{2.050pt}}
\multiput(146.60,620.60)(0.468,-9.401){5}{\rule{0.113pt}{6.600pt}}
\multiput(145.17,634.30)(4.000,-51.301){2}{\rule{0.400pt}{3.300pt}}
\multiput(150.61,583.00)(0.447,24.574){3}{\rule{0.108pt}{14.900pt}}
\multiput(149.17,583.00)(3.000,80.074){2}{\rule{0.400pt}{7.450pt}}
\multiput(153.61,694.00)(0.447,1.802){3}{\rule{0.108pt}{1.300pt}}
\multiput(152.17,694.00)(3.000,6.302){2}{\rule{0.400pt}{0.650pt}}
\multiput(156.61,703.00)(0.447,9.393){3}{\rule{0.108pt}{5.833pt}}
\multiput(155.17,703.00)(3.000,30.893){2}{\rule{0.400pt}{2.917pt}}
\multiput(159.60,740.60)(0.468,-1.651){5}{\rule{0.113pt}{1.300pt}}
\multiput(158.17,743.30)(4.000,-9.302){2}{\rule{0.400pt}{0.650pt}}
\multiput(163.00,734.61)(0.462,0.447){3}{\rule{0.500pt}{0.108pt}}
\multiput(163.00,733.17)(1.962,3.000){2}{\rule{0.250pt}{0.400pt}}
\multiput(166.61,737.00)(0.447,12.965){3}{\rule{0.108pt}{7.967pt}}
\multiput(165.17,737.00)(3.000,42.465){2}{\rule{0.400pt}{3.983pt}}
\multiput(169.60,796.00)(0.468,1.651){5}{\rule{0.113pt}{1.300pt}}
\multiput(168.17,796.00)(4.000,9.302){2}{\rule{0.400pt}{0.650pt}}
\multiput(173.61,802.60)(0.447,-1.802){3}{\rule{0.108pt}{1.300pt}}
\multiput(172.17,805.30)(3.000,-6.302){2}{\rule{0.400pt}{0.650pt}}
\multiput(176.61,799.00)(0.447,4.481){3}{\rule{0.108pt}{2.900pt}}
\multiput(175.17,799.00)(3.000,14.981){2}{\rule{0.400pt}{1.450pt}}
\multiput(179.61,771.43)(0.447,-19.216){3}{\rule{0.108pt}{11.700pt}}
\multiput(178.17,795.72)(3.000,-62.716){2}{\rule{0.400pt}{5.850pt}}
\multiput(182.60,733.00)(0.468,4.283){5}{\rule{0.113pt}{3.100pt}}
\multiput(181.17,733.00)(4.000,23.566){2}{\rule{0.400pt}{1.550pt}}
\multiput(186.61,744.87)(0.447,-6.937){3}{\rule{0.108pt}{4.367pt}}
\multiput(185.17,753.94)(3.000,-22.937){2}{\rule{0.400pt}{2.183pt}}
\multiput(192.61,731.00)(0.447,3.141){3}{\rule{0.108pt}{2.100pt}}
\multiput(191.17,731.00)(3.000,10.641){2}{\rule{0.400pt}{1.050pt}}
\multiput(195.60,746.00)(0.468,2.382){5}{\rule{0.113pt}{1.800pt}}
\multiput(194.17,746.00)(4.000,13.264){2}{\rule{0.400pt}{0.900pt}}
\multiput(199.61,750.41)(0.447,-4.704){3}{\rule{0.108pt}{3.033pt}}
\multiput(198.17,756.70)(3.000,-15.704){2}{\rule{0.400pt}{1.517pt}}
\multiput(202.61,741.00)(0.447,4.258){3}{\rule{0.108pt}{2.767pt}}
\multiput(201.17,741.00)(3.000,14.258){2}{\rule{0.400pt}{1.383pt}}
\multiput(205.60,747.72)(0.468,-4.429){5}{\rule{0.113pt}{3.200pt}}
\multiput(204.17,754.36)(4.000,-24.358){2}{\rule{0.400pt}{1.600pt}}
\multiput(209.61,730.00)(0.447,4.927){3}{\rule{0.108pt}{3.167pt}}
\multiput(208.17,730.00)(3.000,16.427){2}{\rule{0.400pt}{1.583pt}}
\multiput(212.61,746.50)(0.447,-2.248){3}{\rule{0.108pt}{1.567pt}}
\multiput(211.17,749.75)(3.000,-7.748){2}{\rule{0.400pt}{0.783pt}}
\multiput(215.61,742.00)(0.447,4.034){3}{\rule{0.108pt}{2.633pt}}
\multiput(214.17,742.00)(3.000,13.534){2}{\rule{0.400pt}{1.317pt}}
\multiput(218.60,752.28)(0.468,-2.821){5}{\rule{0.113pt}{2.100pt}}
\multiput(217.17,756.64)(4.000,-15.641){2}{\rule{0.400pt}{1.050pt}}
\multiput(222.61,741.00)(0.447,1.355){3}{\rule{0.108pt}{1.033pt}}
\multiput(221.17,741.00)(3.000,4.855){2}{\rule{0.400pt}{0.517pt}}
\multiput(225.61,748.00)(0.447,1.802){3}{\rule{0.108pt}{1.300pt}}
\multiput(224.17,748.00)(3.000,6.302){2}{\rule{0.400pt}{0.650pt}}
\multiput(228.61,743.30)(0.447,-5.151){3}{\rule{0.108pt}{3.300pt}}
\multiput(227.17,750.15)(3.000,-17.151){2}{\rule{0.400pt}{1.650pt}}
\multiput(231.60,733.00)(0.468,2.821){5}{\rule{0.113pt}{2.100pt}}
\multiput(230.17,733.00)(4.000,15.641){2}{\rule{0.400pt}{1.050pt}}
\multiput(235.61,744.84)(0.447,-2.918){3}{\rule{0.108pt}{1.967pt}}
\multiput(234.17,748.92)(3.000,-9.918){2}{\rule{0.400pt}{0.983pt}}
\multiput(238.61,731.94)(0.447,-2.472){3}{\rule{0.108pt}{1.700pt}}
\multiput(237.17,735.47)(3.000,-8.472){2}{\rule{0.400pt}{0.850pt}}
\multiput(241.60,727.00)(0.468,0.774){5}{\rule{0.113pt}{0.700pt}}
\multiput(240.17,727.00)(4.000,4.547){2}{\rule{0.400pt}{0.350pt}}
\multiput(245.61,733.00)(0.447,2.025){3}{\rule{0.108pt}{1.433pt}}
\multiput(244.17,733.00)(3.000,7.025){2}{\rule{0.400pt}{0.717pt}}
\multiput(248.61,743.00)(0.447,4.258){3}{\rule{0.108pt}{2.767pt}}
\multiput(247.17,743.00)(3.000,14.258){2}{\rule{0.400pt}{1.383pt}}
\multiput(251.61,763.00)(0.447,4.704){3}{\rule{0.108pt}{3.033pt}}
\multiput(250.17,763.00)(3.000,15.704){2}{\rule{0.400pt}{1.517pt}}
\multiput(254.60,785.00)(0.468,0.920){5}{\rule{0.113pt}{0.800pt}}
\multiput(253.17,785.00)(4.000,5.340){2}{\rule{0.400pt}{0.400pt}}
\multiput(258.61,792.00)(0.447,0.909){3}{\rule{0.108pt}{0.767pt}}
\multiput(257.17,792.00)(3.000,3.409){2}{\rule{0.400pt}{0.383pt}}
\multiput(261.61,797.00)(0.447,6.490){3}{\rule{0.108pt}{4.100pt}}
\multiput(260.17,797.00)(3.000,21.490){2}{\rule{0.400pt}{2.050pt}}
\multiput(264.61,823.82)(0.447,-0.909){3}{\rule{0.108pt}{0.767pt}}
\multiput(263.17,825.41)(3.000,-3.409){2}{\rule{0.400pt}{0.383pt}}
\multiput(267.60,817.85)(0.468,-1.212){5}{\rule{0.113pt}{1.000pt}}
\multiput(266.17,819.92)(4.000,-6.924){2}{\rule{0.400pt}{0.500pt}}
\multiput(271.61,809.26)(0.447,-1.132){3}{\rule{0.108pt}{0.900pt}}
\multiput(270.17,811.13)(3.000,-4.132){2}{\rule{0.400pt}{0.450pt}}
\multiput(274.61,804.37)(0.447,-0.685){3}{\rule{0.108pt}{0.633pt}}
\multiput(273.17,805.69)(3.000,-2.685){2}{\rule{0.400pt}{0.317pt}}
\multiput(277.00,801.95)(0.685,-0.447){3}{\rule{0.633pt}{0.108pt}}
\multiput(277.00,802.17)(2.685,-3.000){2}{\rule{0.317pt}{0.400pt}}
\multiput(281.61,793.50)(0.447,-2.248){3}{\rule{0.108pt}{1.567pt}}
\multiput(280.17,796.75)(3.000,-7.748){2}{\rule{0.400pt}{0.783pt}}
\multiput(284.61,785.82)(0.447,-0.909){3}{\rule{0.108pt}{0.767pt}}
\multiput(283.17,787.41)(3.000,-3.409){2}{\rule{0.400pt}{0.383pt}}
\multiput(287.61,784.00)(0.447,2.918){3}{\rule{0.108pt}{1.967pt}}
\multiput(286.17,784.00)(3.000,9.918){2}{\rule{0.400pt}{0.983pt}}
\multiput(290.00,796.95)(0.685,-0.447){3}{\rule{0.633pt}{0.108pt}}
\multiput(290.00,797.17)(2.685,-3.000){2}{\rule{0.317pt}{0.400pt}}
\multiput(294.61,795.00)(0.447,4.704){3}{\rule{0.108pt}{3.033pt}}
\multiput(293.17,795.00)(3.000,15.704){2}{\rule{0.400pt}{1.517pt}}
\multiput(297.61,806.62)(0.447,-3.811){3}{\rule{0.108pt}{2.500pt}}
\multiput(296.17,811.81)(3.000,-12.811){2}{\rule{0.400pt}{1.250pt}}
\put(300,797.17){\rule{0.700pt}{0.400pt}}
\multiput(300.00,798.17)(1.547,-2.000){2}{\rule{0.350pt}{0.400pt}}
\multiput(303.00,797.61)(0.685,0.447){3}{\rule{0.633pt}{0.108pt}}
\multiput(303.00,796.17)(2.685,3.000){2}{\rule{0.317pt}{0.400pt}}
\multiput(307.61,797.37)(0.447,-0.685){3}{\rule{0.108pt}{0.633pt}}
\multiput(306.17,798.69)(3.000,-2.685){2}{\rule{0.400pt}{0.317pt}}
\put(310,794.17){\rule{0.700pt}{0.400pt}}
\multiput(310.00,795.17)(1.547,-2.000){2}{\rule{0.350pt}{0.400pt}}
\multiput(313.60,791.09)(0.468,-0.774){5}{\rule{0.113pt}{0.700pt}}
\multiput(312.17,792.55)(4.000,-4.547){2}{\rule{0.400pt}{0.350pt}}
\put(317,788.17){\rule{0.700pt}{0.400pt}}
\multiput(317.00,787.17)(1.547,2.000){2}{\rule{0.350pt}{0.400pt}}
\multiput(320.00,790.61)(0.462,0.447){3}{\rule{0.500pt}{0.108pt}}
\multiput(320.00,789.17)(1.962,3.000){2}{\rule{0.250pt}{0.400pt}}
\multiput(323.61,793.00)(0.447,2.025){3}{\rule{0.108pt}{1.433pt}}
\multiput(322.17,793.00)(3.000,7.025){2}{\rule{0.400pt}{0.717pt}}
\multiput(326.00,803.61)(0.685,0.447){3}{\rule{0.633pt}{0.108pt}}
\multiput(326.00,802.17)(2.685,3.000){2}{\rule{0.317pt}{0.400pt}}
\multiput(330.61,795.07)(0.447,-4.034){3}{\rule{0.108pt}{2.633pt}}
\multiput(329.17,800.53)(3.000,-13.534){2}{\rule{0.400pt}{1.317pt}}
\multiput(333.61,781.05)(0.447,-2.025){3}{\rule{0.108pt}{1.433pt}}
\multiput(332.17,784.03)(3.000,-7.025){2}{\rule{0.400pt}{0.717pt}}
\put(336,775.67){\rule{0.723pt}{0.400pt}}
\multiput(336.00,776.17)(1.500,-1.000){2}{\rule{0.361pt}{0.400pt}}
\put(339,774.67){\rule{0.964pt}{0.400pt}}
\multiput(339.00,775.17)(2.000,-1.000){2}{\rule{0.482pt}{0.400pt}}
\multiput(343.61,769.60)(0.447,-1.802){3}{\rule{0.108pt}{1.300pt}}
\multiput(342.17,772.30)(3.000,-6.302){2}{\rule{0.400pt}{0.650pt}}
\multiput(346.00,766.61)(0.462,0.447){3}{\rule{0.500pt}{0.108pt}}
\multiput(346.00,765.17)(1.962,3.000){2}{\rule{0.250pt}{0.400pt}}
\multiput(349.60,769.00)(0.468,1.505){5}{\rule{0.113pt}{1.200pt}}
\multiput(348.17,769.00)(4.000,8.509){2}{\rule{0.400pt}{0.600pt}}
\put(353,778.67){\rule{0.723pt}{0.400pt}}
\multiput(353.00,779.17)(1.500,-1.000){2}{\rule{0.361pt}{0.400pt}}
\multiput(356.61,779.00)(0.447,0.685){3}{\rule{0.108pt}{0.633pt}}
\multiput(355.17,779.00)(3.000,2.685){2}{\rule{0.400pt}{0.317pt}}
\multiput(359.61,777.60)(0.447,-1.802){3}{\rule{0.108pt}{1.300pt}}
\multiput(358.17,780.30)(3.000,-6.302){2}{\rule{0.400pt}{0.650pt}}
\multiput(362.00,774.61)(0.685,0.447){3}{\rule{0.633pt}{0.108pt}}
\multiput(362.00,773.17)(2.685,3.000){2}{\rule{0.317pt}{0.400pt}}
\multiput(366.61,777.00)(0.447,0.909){3}{\rule{0.108pt}{0.767pt}}
\multiput(365.17,777.00)(3.000,3.409){2}{\rule{0.400pt}{0.383pt}}
\multiput(369.00,780.95)(0.462,-0.447){3}{\rule{0.500pt}{0.108pt}}
\multiput(369.00,781.17)(1.962,-3.000){2}{\rule{0.250pt}{0.400pt}}
\multiput(372.00,777.95)(0.462,-0.447){3}{\rule{0.500pt}{0.108pt}}
\multiput(372.00,778.17)(1.962,-3.000){2}{\rule{0.250pt}{0.400pt}}
\put(375,774.67){\rule{0.964pt}{0.400pt}}
\multiput(375.00,775.17)(2.000,-1.000){2}{\rule{0.482pt}{0.400pt}}
\multiput(379.00,775.61)(0.462,0.447){3}{\rule{0.500pt}{0.108pt}}
\multiput(379.00,774.17)(1.962,3.000){2}{\rule{0.250pt}{0.400pt}}
\multiput(382.61,778.00)(0.447,1.355){3}{\rule{0.108pt}{1.033pt}}
\multiput(381.17,778.00)(3.000,4.855){2}{\rule{0.400pt}{0.517pt}}
\put(189.0,731.0){\rule[-0.200pt]{0.723pt}{0.400pt}}
\multiput(389.61,782.37)(0.447,-0.685){3}{\rule{0.108pt}{0.633pt}}
\multiput(388.17,783.69)(3.000,-2.685){2}{\rule{0.400pt}{0.317pt}}
\put(385.0,785.0){\rule[-0.200pt]{0.964pt}{0.400pt}}
\multiput(395.61,781.00)(0.447,1.132){3}{\rule{0.108pt}{0.900pt}}
\multiput(394.17,781.00)(3.000,4.132){2}{\rule{0.400pt}{0.450pt}}
\put(392.0,781.0){\rule[-0.200pt]{0.723pt}{0.400pt}}
\multiput(402.61,787.00)(0.447,2.248){3}{\rule{0.108pt}{1.567pt}}
\multiput(401.17,787.00)(3.000,7.748){2}{\rule{0.400pt}{0.783pt}}
\put(405,798.17){\rule{0.700pt}{0.400pt}}
\multiput(405.00,797.17)(1.547,2.000){2}{\rule{0.350pt}{0.400pt}}
\multiput(408.61,796.82)(0.447,-0.909){3}{\rule{0.108pt}{0.767pt}}
\multiput(407.17,798.41)(3.000,-3.409){2}{\rule{0.400pt}{0.383pt}}
\multiput(411.60,791.68)(0.468,-0.920){5}{\rule{0.113pt}{0.800pt}}
\multiput(410.17,793.34)(4.000,-5.340){2}{\rule{0.400pt}{0.400pt}}
\multiput(415.61,784.82)(0.447,-0.909){3}{\rule{0.108pt}{0.767pt}}
\multiput(414.17,786.41)(3.000,-3.409){2}{\rule{0.400pt}{0.383pt}}
\multiput(418.61,783.00)(0.447,2.025){3}{\rule{0.108pt}{1.433pt}}
\multiput(417.17,783.00)(3.000,7.025){2}{\rule{0.400pt}{0.717pt}}
\multiput(421.60,793.00)(0.468,0.774){5}{\rule{0.113pt}{0.700pt}}
\multiput(420.17,793.00)(4.000,4.547){2}{\rule{0.400pt}{0.350pt}}
\put(425,797.67){\rule{0.723pt}{0.400pt}}
\multiput(425.00,798.17)(1.500,-1.000){2}{\rule{0.361pt}{0.400pt}}
\multiput(428.61,798.00)(0.447,0.909){3}{\rule{0.108pt}{0.767pt}}
\multiput(427.17,798.00)(3.000,3.409){2}{\rule{0.400pt}{0.383pt}}
\multiput(431.00,801.95)(0.462,-0.447){3}{\rule{0.500pt}{0.108pt}}
\multiput(431.00,802.17)(1.962,-3.000){2}{\rule{0.250pt}{0.400pt}}
\multiput(434.60,800.00)(0.468,0.627){5}{\rule{0.113pt}{0.600pt}}
\multiput(433.17,800.00)(4.000,3.755){2}{\rule{0.400pt}{0.300pt}}
\multiput(438.61,796.28)(0.447,-3.141){3}{\rule{0.108pt}{2.100pt}}
\multiput(437.17,800.64)(3.000,-10.641){2}{\rule{0.400pt}{1.050pt}}
\multiput(441.00,788.95)(0.462,-0.447){3}{\rule{0.500pt}{0.108pt}}
\multiput(441.00,789.17)(1.962,-3.000){2}{\rule{0.250pt}{0.400pt}}
\put(444,785.17){\rule{0.700pt}{0.400pt}}
\multiput(444.00,786.17)(1.547,-2.000){2}{\rule{0.350pt}{0.400pt}}
\put(447,783.17){\rule{0.900pt}{0.400pt}}
\multiput(447.00,784.17)(2.132,-2.000){2}{\rule{0.450pt}{0.400pt}}
\put(451,781.67){\rule{0.723pt}{0.400pt}}
\multiput(451.00,782.17)(1.500,-1.000){2}{\rule{0.361pt}{0.400pt}}
\put(398.0,787.0){\rule[-0.200pt]{0.964pt}{0.400pt}}
\multiput(457.60,777.85)(0.468,-1.212){5}{\rule{0.113pt}{1.000pt}}
\multiput(456.17,779.92)(4.000,-6.924){2}{\rule{0.400pt}{0.500pt}}
\multiput(461.00,771.95)(0.462,-0.447){3}{\rule{0.500pt}{0.108pt}}
\multiput(461.00,772.17)(1.962,-3.000){2}{\rule{0.250pt}{0.400pt}}
\multiput(464.61,762.39)(0.447,-2.695){3}{\rule{0.108pt}{1.833pt}}
\multiput(463.17,766.19)(3.000,-9.195){2}{\rule{0.400pt}{0.917pt}}
\multiput(467.61,753.82)(0.447,-0.909){3}{\rule{0.108pt}{0.767pt}}
\multiput(466.17,755.41)(3.000,-3.409){2}{\rule{0.400pt}{0.383pt}}
\put(470,752.17){\rule{0.900pt}{0.400pt}}
\multiput(470.00,751.17)(2.132,2.000){2}{\rule{0.450pt}{0.400pt}}
\put(454.0,782.0){\rule[-0.200pt]{0.723pt}{0.400pt}}
\multiput(477.61,750.82)(0.447,-0.909){3}{\rule{0.108pt}{0.767pt}}
\multiput(476.17,752.41)(3.000,-3.409){2}{\rule{0.400pt}{0.383pt}}
\multiput(480.61,744.71)(0.447,-1.355){3}{\rule{0.108pt}{1.033pt}}
\multiput(479.17,746.86)(3.000,-4.855){2}{\rule{0.400pt}{0.517pt}}
\put(483,740.17){\rule{0.900pt}{0.400pt}}
\multiput(483.00,741.17)(2.132,-2.000){2}{\rule{0.450pt}{0.400pt}}
\multiput(487.61,732.39)(0.447,-2.695){3}{\rule{0.108pt}{1.833pt}}
\multiput(486.17,736.19)(3.000,-9.195){2}{\rule{0.400pt}{0.917pt}}
\multiput(490.61,722.16)(0.447,-1.579){3}{\rule{0.108pt}{1.167pt}}
\multiput(489.17,724.58)(3.000,-5.579){2}{\rule{0.400pt}{0.583pt}}
\put(493,717.67){\rule{0.964pt}{0.400pt}}
\multiput(493.00,718.17)(2.000,-1.000){2}{\rule{0.482pt}{0.400pt}}
\multiput(497.61,714.26)(0.447,-1.132){3}{\rule{0.108pt}{0.900pt}}
\multiput(496.17,716.13)(3.000,-4.132){2}{\rule{0.400pt}{0.450pt}}
\multiput(500.61,712.00)(0.447,0.685){3}{\rule{0.108pt}{0.633pt}}
\multiput(499.17,712.00)(3.000,2.685){2}{\rule{0.400pt}{0.317pt}}
\put(503,715.67){\rule{0.723pt}{0.400pt}}
\multiput(503.00,715.17)(1.500,1.000){2}{\rule{0.361pt}{0.400pt}}
\multiput(506.60,717.00)(0.468,0.920){5}{\rule{0.113pt}{0.800pt}}
\multiput(505.17,717.00)(4.000,5.340){2}{\rule{0.400pt}{0.400pt}}
\multiput(510.61,724.00)(0.447,1.579){3}{\rule{0.108pt}{1.167pt}}
\multiput(509.17,724.00)(3.000,5.579){2}{\rule{0.400pt}{0.583pt}}
\put(513,732.17){\rule{0.700pt}{0.400pt}}
\multiput(513.00,731.17)(1.547,2.000){2}{\rule{0.350pt}{0.400pt}}
\multiput(516.61,730.82)(0.447,-0.909){3}{\rule{0.108pt}{0.767pt}}
\multiput(515.17,732.41)(3.000,-3.409){2}{\rule{0.400pt}{0.383pt}}
\multiput(519.60,726.09)(0.468,-0.774){5}{\rule{0.113pt}{0.700pt}}
\multiput(518.17,727.55)(4.000,-4.547){2}{\rule{0.400pt}{0.350pt}}
\put(474.0,754.0){\rule[-0.200pt]{0.723pt}{0.400pt}}
\multiput(526.00,723.61)(0.462,0.447){3}{\rule{0.500pt}{0.108pt}}
\multiput(526.00,722.17)(1.962,3.000){2}{\rule{0.250pt}{0.400pt}}
\multiput(529.00,724.94)(0.481,-0.468){5}{\rule{0.500pt}{0.113pt}}
\multiput(529.00,725.17)(2.962,-4.000){2}{\rule{0.250pt}{0.400pt}}
\put(533,720.17){\rule{0.700pt}{0.400pt}}
\multiput(533.00,721.17)(1.547,-2.000){2}{\rule{0.350pt}{0.400pt}}
\put(523.0,723.0){\rule[-0.200pt]{0.723pt}{0.400pt}}
\put(539,718.17){\rule{0.700pt}{0.400pt}}
\multiput(539.00,719.17)(1.547,-2.000){2}{\rule{0.350pt}{0.400pt}}
\multiput(542.60,718.00)(0.468,0.627){5}{\rule{0.113pt}{0.600pt}}
\multiput(541.17,718.00)(4.000,3.755){2}{\rule{0.400pt}{0.300pt}}
\multiput(546.61,723.00)(0.447,0.685){3}{\rule{0.108pt}{0.633pt}}
\multiput(545.17,723.00)(3.000,2.685){2}{\rule{0.400pt}{0.317pt}}
\multiput(549.61,727.00)(0.447,1.132){3}{\rule{0.108pt}{0.900pt}}
\multiput(548.17,727.00)(3.000,4.132){2}{\rule{0.400pt}{0.450pt}}
\multiput(552.00,731.95)(0.462,-0.447){3}{\rule{0.500pt}{0.108pt}}
\multiput(552.00,732.17)(1.962,-3.000){2}{\rule{0.250pt}{0.400pt}}
\multiput(555.60,725.02)(0.468,-1.505){5}{\rule{0.113pt}{1.200pt}}
\multiput(554.17,727.51)(4.000,-8.509){2}{\rule{0.400pt}{0.600pt}}
\put(559,717.17){\rule{0.700pt}{0.400pt}}
\multiput(559.00,718.17)(1.547,-2.000){2}{\rule{0.350pt}{0.400pt}}
\put(562,716.67){\rule{0.723pt}{0.400pt}}
\multiput(562.00,716.17)(1.500,1.000){2}{\rule{0.361pt}{0.400pt}}
\put(565,716.67){\rule{0.964pt}{0.400pt}}
\multiput(565.00,717.17)(2.000,-1.000){2}{\rule{0.482pt}{0.400pt}}
\multiput(569.00,715.95)(0.462,-0.447){3}{\rule{0.500pt}{0.108pt}}
\multiput(569.00,716.17)(1.962,-3.000){2}{\rule{0.250pt}{0.400pt}}
\multiput(572.00,712.95)(0.462,-0.447){3}{\rule{0.500pt}{0.108pt}}
\multiput(572.00,713.17)(1.962,-3.000){2}{\rule{0.250pt}{0.400pt}}
\put(575,710.67){\rule{0.723pt}{0.400pt}}
\multiput(575.00,710.17)(1.500,1.000){2}{\rule{0.361pt}{0.400pt}}
\put(578,710.67){\rule{0.964pt}{0.400pt}}
\multiput(578.00,711.17)(2.000,-1.000){2}{\rule{0.482pt}{0.400pt}}
\multiput(582.00,711.61)(0.462,0.447){3}{\rule{0.500pt}{0.108pt}}
\multiput(582.00,710.17)(1.962,3.000){2}{\rule{0.250pt}{0.400pt}}
\put(585,713.67){\rule{0.723pt}{0.400pt}}
\multiput(585.00,713.17)(1.500,1.000){2}{\rule{0.361pt}{0.400pt}}
\multiput(588.61,711.82)(0.447,-0.909){3}{\rule{0.108pt}{0.767pt}}
\multiput(587.17,713.41)(3.000,-3.409){2}{\rule{0.400pt}{0.383pt}}
\multiput(591.60,707.51)(0.468,-0.627){5}{\rule{0.113pt}{0.600pt}}
\multiput(590.17,708.75)(4.000,-3.755){2}{\rule{0.400pt}{0.300pt}}
\put(595,704.67){\rule{0.723pt}{0.400pt}}
\multiput(595.00,704.17)(1.500,1.000){2}{\rule{0.361pt}{0.400pt}}
\put(598,705.67){\rule{0.723pt}{0.400pt}}
\multiput(598.00,705.17)(1.500,1.000){2}{\rule{0.361pt}{0.400pt}}
\multiput(601.60,702.85)(0.468,-1.212){5}{\rule{0.113pt}{1.000pt}}
\multiput(600.17,704.92)(4.000,-6.924){2}{\rule{0.400pt}{0.500pt}}
\multiput(605.00,698.61)(0.462,0.447){3}{\rule{0.500pt}{0.108pt}}
\multiput(605.00,697.17)(1.962,3.000){2}{\rule{0.250pt}{0.400pt}}
\multiput(608.61,693.94)(0.447,-2.472){3}{\rule{0.108pt}{1.700pt}}
\multiput(607.17,697.47)(3.000,-8.472){2}{\rule{0.400pt}{0.850pt}}
\put(611,687.17){\rule{0.700pt}{0.400pt}}
\multiput(611.00,688.17)(1.547,-2.000){2}{\rule{0.350pt}{0.400pt}}
\multiput(614.00,687.60)(0.481,0.468){5}{\rule{0.500pt}{0.113pt}}
\multiput(614.00,686.17)(2.962,4.000){2}{\rule{0.250pt}{0.400pt}}
\put(618,690.67){\rule{0.723pt}{0.400pt}}
\multiput(618.00,690.17)(1.500,1.000){2}{\rule{0.361pt}{0.400pt}}
\multiput(621.61,692.00)(0.447,0.685){3}{\rule{0.108pt}{0.633pt}}
\multiput(620.17,692.00)(3.000,2.685){2}{\rule{0.400pt}{0.317pt}}
\multiput(624.00,696.61)(0.462,0.447){3}{\rule{0.500pt}{0.108pt}}
\multiput(624.00,695.17)(1.962,3.000){2}{\rule{0.250pt}{0.400pt}}
\multiput(627.60,696.09)(0.468,-0.774){5}{\rule{0.113pt}{0.700pt}}
\multiput(626.17,697.55)(4.000,-4.547){2}{\rule{0.400pt}{0.350pt}}
\put(631,693.17){\rule{0.700pt}{0.400pt}}
\multiput(631.00,692.17)(1.547,2.000){2}{\rule{0.350pt}{0.400pt}}
\put(634,693.17){\rule{0.700pt}{0.400pt}}
\multiput(634.00,694.17)(1.547,-2.000){2}{\rule{0.350pt}{0.400pt}}
\multiput(637.00,693.61)(0.685,0.447){3}{\rule{0.633pt}{0.108pt}}
\multiput(637.00,692.17)(2.685,3.000){2}{\rule{0.317pt}{0.400pt}}
\put(641,695.67){\rule{0.723pt}{0.400pt}}
\multiput(641.00,695.17)(1.500,1.000){2}{\rule{0.361pt}{0.400pt}}
\put(644,696.67){\rule{0.723pt}{0.400pt}}
\multiput(644.00,696.17)(1.500,1.000){2}{\rule{0.361pt}{0.400pt}}
\put(647,696.17){\rule{0.700pt}{0.400pt}}
\multiput(647.00,697.17)(1.547,-2.000){2}{\rule{0.350pt}{0.400pt}}
\multiput(650.00,696.60)(0.481,0.468){5}{\rule{0.500pt}{0.113pt}}
\multiput(650.00,695.17)(2.962,4.000){2}{\rule{0.250pt}{0.400pt}}
\put(654,700.17){\rule{0.700pt}{0.400pt}}
\multiput(654.00,699.17)(1.547,2.000){2}{\rule{0.350pt}{0.400pt}}
\multiput(657.61,698.82)(0.447,-0.909){3}{\rule{0.108pt}{0.767pt}}
\multiput(656.17,700.41)(3.000,-3.409){2}{\rule{0.400pt}{0.383pt}}
\multiput(660.00,697.61)(0.462,0.447){3}{\rule{0.500pt}{0.108pt}}
\multiput(660.00,696.17)(1.962,3.000){2}{\rule{0.250pt}{0.400pt}}
\multiput(663.00,700.60)(0.481,0.468){5}{\rule{0.500pt}{0.113pt}}
\multiput(663.00,699.17)(2.962,4.000){2}{\rule{0.250pt}{0.400pt}}
\put(667,702.17){\rule{0.700pt}{0.400pt}}
\multiput(667.00,703.17)(1.547,-2.000){2}{\rule{0.350pt}{0.400pt}}
\put(670,701.67){\rule{0.723pt}{0.400pt}}
\multiput(670.00,701.17)(1.500,1.000){2}{\rule{0.361pt}{0.400pt}}
\multiput(673.00,701.94)(0.481,-0.468){5}{\rule{0.500pt}{0.113pt}}
\multiput(673.00,702.17)(2.962,-4.000){2}{\rule{0.250pt}{0.400pt}}
\multiput(677.00,699.61)(0.462,0.447){3}{\rule{0.500pt}{0.108pt}}
\multiput(677.00,698.17)(1.962,3.000){2}{\rule{0.250pt}{0.400pt}}
\multiput(680.61,698.82)(0.447,-0.909){3}{\rule{0.108pt}{0.767pt}}
\multiput(679.17,700.41)(3.000,-3.409){2}{\rule{0.400pt}{0.383pt}}
\multiput(683.00,695.95)(0.462,-0.447){3}{\rule{0.500pt}{0.108pt}}
\multiput(683.00,696.17)(1.962,-3.000){2}{\rule{0.250pt}{0.400pt}}
\put(536.0,720.0){\rule[-0.200pt]{0.723pt}{0.400pt}}
\put(690,692.17){\rule{0.700pt}{0.400pt}}
\multiput(690.00,693.17)(1.547,-2.000){2}{\rule{0.350pt}{0.400pt}}
\multiput(693.61,689.37)(0.447,-0.685){3}{\rule{0.108pt}{0.633pt}}
\multiput(692.17,690.69)(3.000,-2.685){2}{\rule{0.400pt}{0.317pt}}
\put(696,686.17){\rule{0.700pt}{0.400pt}}
\multiput(696.00,687.17)(1.547,-2.000){2}{\rule{0.350pt}{0.400pt}}
\multiput(699.60,683.51)(0.468,-0.627){5}{\rule{0.113pt}{0.600pt}}
\multiput(698.17,684.75)(4.000,-3.755){2}{\rule{0.400pt}{0.300pt}}
\put(703,680.67){\rule{0.723pt}{0.400pt}}
\multiput(703.00,680.17)(1.500,1.000){2}{\rule{0.361pt}{0.400pt}}
\put(706,680.67){\rule{0.723pt}{0.400pt}}
\multiput(706.00,681.17)(1.500,-1.000){2}{\rule{0.361pt}{0.400pt}}
\put(709,681.17){\rule{0.900pt}{0.400pt}}
\multiput(709.00,680.17)(2.132,2.000){2}{\rule{0.450pt}{0.400pt}}
\multiput(713.61,679.82)(0.447,-0.909){3}{\rule{0.108pt}{0.767pt}}
\multiput(712.17,681.41)(3.000,-3.409){2}{\rule{0.400pt}{0.383pt}}
\multiput(716.61,678.00)(0.447,0.685){3}{\rule{0.108pt}{0.633pt}}
\multiput(715.17,678.00)(3.000,2.685){2}{\rule{0.400pt}{0.317pt}}
\multiput(719.00,680.95)(0.462,-0.447){3}{\rule{0.500pt}{0.108pt}}
\multiput(719.00,681.17)(1.962,-3.000){2}{\rule{0.250pt}{0.400pt}}
\put(722,677.67){\rule{0.964pt}{0.400pt}}
\multiput(722.00,678.17)(2.000,-1.000){2}{\rule{0.482pt}{0.400pt}}
\put(726,678.17){\rule{0.700pt}{0.400pt}}
\multiput(726.00,677.17)(1.547,2.000){2}{\rule{0.350pt}{0.400pt}}
\multiput(729.61,676.82)(0.447,-0.909){3}{\rule{0.108pt}{0.767pt}}
\multiput(728.17,678.41)(3.000,-3.409){2}{\rule{0.400pt}{0.383pt}}
\multiput(732.00,675.61)(0.462,0.447){3}{\rule{0.500pt}{0.108pt}}
\multiput(732.00,674.17)(1.962,3.000){2}{\rule{0.250pt}{0.400pt}}
\put(735,678.17){\rule{0.900pt}{0.400pt}}
\multiput(735.00,677.17)(2.132,2.000){2}{\rule{0.450pt}{0.400pt}}
\put(686.0,694.0){\rule[-0.200pt]{0.964pt}{0.400pt}}
\multiput(742.61,680.00)(0.447,0.909){3}{\rule{0.108pt}{0.767pt}}
\multiput(741.17,680.00)(3.000,3.409){2}{\rule{0.400pt}{0.383pt}}
\multiput(745.60,685.00)(0.468,0.627){5}{\rule{0.113pt}{0.600pt}}
\multiput(744.17,685.00)(4.000,3.755){2}{\rule{0.400pt}{0.300pt}}
\put(749,688.17){\rule{0.700pt}{0.400pt}}
\multiput(749.00,689.17)(1.547,-2.000){2}{\rule{0.350pt}{0.400pt}}
\put(752,686.67){\rule{0.723pt}{0.400pt}}
\multiput(752.00,687.17)(1.500,-1.000){2}{\rule{0.361pt}{0.400pt}}
\multiput(755.61,683.26)(0.447,-1.132){3}{\rule{0.108pt}{0.900pt}}
\multiput(754.17,685.13)(3.000,-4.132){2}{\rule{0.400pt}{0.450pt}}
\put(758,680.67){\rule{0.964pt}{0.400pt}}
\multiput(758.00,680.17)(2.000,1.000){2}{\rule{0.482pt}{0.400pt}}
\put(762,680.67){\rule{0.723pt}{0.400pt}}
\multiput(762.00,681.17)(1.500,-1.000){2}{\rule{0.361pt}{0.400pt}}
\put(739.0,680.0){\rule[-0.200pt]{0.723pt}{0.400pt}}
\put(768,679.67){\rule{0.723pt}{0.400pt}}
\multiput(768.00,680.17)(1.500,-1.000){2}{\rule{0.361pt}{0.400pt}}
\multiput(771.00,680.61)(0.685,0.447){3}{\rule{0.633pt}{0.108pt}}
\multiput(771.00,679.17)(2.685,3.000){2}{\rule{0.317pt}{0.400pt}}
\multiput(775.00,683.61)(0.462,0.447){3}{\rule{0.500pt}{0.108pt}}
\multiput(775.00,682.17)(1.962,3.000){2}{\rule{0.250pt}{0.400pt}}
\put(765.0,681.0){\rule[-0.200pt]{0.723pt}{0.400pt}}
\put(781,686.17){\rule{0.700pt}{0.400pt}}
\multiput(781.00,685.17)(1.547,2.000){2}{\rule{0.350pt}{0.400pt}}
\multiput(784.00,688.61)(0.685,0.447){3}{\rule{0.633pt}{0.108pt}}
\multiput(784.00,687.17)(2.685,3.000){2}{\rule{0.317pt}{0.400pt}}
\put(788,690.67){\rule{0.723pt}{0.400pt}}
\multiput(788.00,690.17)(1.500,1.000){2}{\rule{0.361pt}{0.400pt}}
\put(791,690.17){\rule{0.700pt}{0.400pt}}
\multiput(791.00,691.17)(1.547,-2.000){2}{\rule{0.350pt}{0.400pt}}
\multiput(794.00,688.94)(0.481,-0.468){5}{\rule{0.500pt}{0.113pt}}
\multiput(794.00,689.17)(2.962,-4.000){2}{\rule{0.250pt}{0.400pt}}
\put(798,686.17){\rule{0.700pt}{0.400pt}}
\multiput(798.00,685.17)(1.547,2.000){2}{\rule{0.350pt}{0.400pt}}
\put(801,686.17){\rule{0.700pt}{0.400pt}}
\multiput(801.00,687.17)(1.547,-2.000){2}{\rule{0.350pt}{0.400pt}}
\put(804,684.17){\rule{0.700pt}{0.400pt}}
\multiput(804.00,685.17)(1.547,-2.000){2}{\rule{0.350pt}{0.400pt}}
\put(807,682.67){\rule{0.964pt}{0.400pt}}
\multiput(807.00,683.17)(2.000,-1.000){2}{\rule{0.482pt}{0.400pt}}
\put(811,681.67){\rule{0.723pt}{0.400pt}}
\multiput(811.00,682.17)(1.500,-1.000){2}{\rule{0.361pt}{0.400pt}}
\multiput(814.00,682.61)(0.462,0.447){3}{\rule{0.500pt}{0.108pt}}
\multiput(814.00,681.17)(1.962,3.000){2}{\rule{0.250pt}{0.400pt}}
\multiput(817.61,685.00)(0.447,0.685){3}{\rule{0.108pt}{0.633pt}}
\multiput(816.17,685.00)(3.000,2.685){2}{\rule{0.400pt}{0.317pt}}
\multiput(820.00,687.95)(0.685,-0.447){3}{\rule{0.633pt}{0.108pt}}
\multiput(820.00,688.17)(2.685,-3.000){2}{\rule{0.317pt}{0.400pt}}
\multiput(824.61,682.26)(0.447,-1.132){3}{\rule{0.108pt}{0.900pt}}
\multiput(823.17,684.13)(3.000,-4.132){2}{\rule{0.400pt}{0.450pt}}
\put(827,678.67){\rule{0.723pt}{0.400pt}}
\multiput(827.00,679.17)(1.500,-1.000){2}{\rule{0.361pt}{0.400pt}}
\put(830,679.17){\rule{0.900pt}{0.400pt}}
\multiput(830.00,678.17)(2.132,2.000){2}{\rule{0.450pt}{0.400pt}}
\put(834,679.17){\rule{0.700pt}{0.400pt}}
\multiput(834.00,680.17)(1.547,-2.000){2}{\rule{0.350pt}{0.400pt}}
\put(837,677.17){\rule{0.700pt}{0.400pt}}
\multiput(837.00,678.17)(1.547,-2.000){2}{\rule{0.350pt}{0.400pt}}
\put(840,676.67){\rule{0.723pt}{0.400pt}}
\multiput(840.00,676.17)(1.500,1.000){2}{\rule{0.361pt}{0.400pt}}
\put(843,676.17){\rule{0.900pt}{0.400pt}}
\multiput(843.00,677.17)(2.132,-2.000){2}{\rule{0.450pt}{0.400pt}}
\put(847,674.67){\rule{0.723pt}{0.400pt}}
\multiput(847.00,675.17)(1.500,-1.000){2}{\rule{0.361pt}{0.400pt}}
\multiput(850.00,675.61)(0.462,0.447){3}{\rule{0.500pt}{0.108pt}}
\multiput(850.00,674.17)(1.962,3.000){2}{\rule{0.250pt}{0.400pt}}
\put(853,678.17){\rule{0.700pt}{0.400pt}}
\multiput(853.00,677.17)(1.547,2.000){2}{\rule{0.350pt}{0.400pt}}
\put(856,679.67){\rule{0.964pt}{0.400pt}}
\multiput(856.00,679.17)(2.000,1.000){2}{\rule{0.482pt}{0.400pt}}
\put(860,680.67){\rule{0.723pt}{0.400pt}}
\multiput(860.00,680.17)(1.500,1.000){2}{\rule{0.361pt}{0.400pt}}
\put(863,680.67){\rule{0.723pt}{0.400pt}}
\multiput(863.00,681.17)(1.500,-1.000){2}{\rule{0.361pt}{0.400pt}}
\multiput(866.00,681.61)(0.685,0.447){3}{\rule{0.633pt}{0.108pt}}
\multiput(866.00,680.17)(2.685,3.000){2}{\rule{0.317pt}{0.400pt}}
\multiput(870.00,684.61)(0.462,0.447){3}{\rule{0.500pt}{0.108pt}}
\multiput(870.00,683.17)(1.962,3.000){2}{\rule{0.250pt}{0.400pt}}
\multiput(873.61,684.37)(0.447,-0.685){3}{\rule{0.108pt}{0.633pt}}
\multiput(872.17,685.69)(3.000,-2.685){2}{\rule{0.400pt}{0.317pt}}
\put(876,682.67){\rule{0.723pt}{0.400pt}}
\multiput(876.00,682.17)(1.500,1.000){2}{\rule{0.361pt}{0.400pt}}
\multiput(879.60,680.68)(0.468,-0.920){5}{\rule{0.113pt}{0.800pt}}
\multiput(878.17,682.34)(4.000,-5.340){2}{\rule{0.400pt}{0.400pt}}
\multiput(883.61,677.00)(0.447,0.685){3}{\rule{0.108pt}{0.633pt}}
\multiput(882.17,677.00)(3.000,2.685){2}{\rule{0.400pt}{0.317pt}}
\multiput(886.61,676.16)(0.447,-1.579){3}{\rule{0.108pt}{1.167pt}}
\multiput(885.17,678.58)(3.000,-5.579){2}{\rule{0.400pt}{0.583pt}}
\put(889,672.67){\rule{0.723pt}{0.400pt}}
\multiput(889.00,672.17)(1.500,1.000){2}{\rule{0.361pt}{0.400pt}}
\put(892,674.17){\rule{0.900pt}{0.400pt}}
\multiput(892.00,673.17)(2.132,2.000){2}{\rule{0.450pt}{0.400pt}}
\multiput(896.61,676.00)(0.447,0.685){3}{\rule{0.108pt}{0.633pt}}
\multiput(895.17,676.00)(3.000,2.685){2}{\rule{0.400pt}{0.317pt}}
\put(899,678.17){\rule{0.700pt}{0.400pt}}
\multiput(899.00,679.17)(1.547,-2.000){2}{\rule{0.350pt}{0.400pt}}
\put(902,677.67){\rule{0.964pt}{0.400pt}}
\multiput(902.00,677.17)(2.000,1.000){2}{\rule{0.482pt}{0.400pt}}
\put(906,677.67){\rule{0.723pt}{0.400pt}}
\multiput(906.00,678.17)(1.500,-1.000){2}{\rule{0.361pt}{0.400pt}}
\multiput(909.61,674.82)(0.447,-0.909){3}{\rule{0.108pt}{0.767pt}}
\multiput(908.17,676.41)(3.000,-3.409){2}{\rule{0.400pt}{0.383pt}}
\put(912,671.67){\rule{0.723pt}{0.400pt}}
\multiput(912.00,672.17)(1.500,-1.000){2}{\rule{0.361pt}{0.400pt}}
\multiput(915.60,672.00)(0.468,0.627){5}{\rule{0.113pt}{0.600pt}}
\multiput(914.17,672.00)(4.000,3.755){2}{\rule{0.400pt}{0.300pt}}
\put(919,675.17){\rule{0.700pt}{0.400pt}}
\multiput(919.00,676.17)(1.547,-2.000){2}{\rule{0.350pt}{0.400pt}}
\multiput(922.61,675.00)(0.447,0.685){3}{\rule{0.108pt}{0.633pt}}
\multiput(921.17,675.00)(3.000,2.685){2}{\rule{0.400pt}{0.317pt}}
\multiput(925.00,679.61)(0.462,0.447){3}{\rule{0.500pt}{0.108pt}}
\multiput(925.00,678.17)(1.962,3.000){2}{\rule{0.250pt}{0.400pt}}
\put(928,680.17){\rule{0.900pt}{0.400pt}}
\multiput(928.00,681.17)(2.132,-2.000){2}{\rule{0.450pt}{0.400pt}}
\multiput(932.00,680.61)(0.462,0.447){3}{\rule{0.500pt}{0.108pt}}
\multiput(932.00,679.17)(1.962,3.000){2}{\rule{0.250pt}{0.400pt}}
\put(778.0,686.0){\rule[-0.200pt]{0.723pt}{0.400pt}}
\put(938,682.67){\rule{0.964pt}{0.400pt}}
\multiput(938.00,682.17)(2.000,1.000){2}{\rule{0.482pt}{0.400pt}}
\multiput(942.00,682.95)(0.462,-0.447){3}{\rule{0.500pt}{0.108pt}}
\multiput(942.00,683.17)(1.962,-3.000){2}{\rule{0.250pt}{0.400pt}}
\put(945,679.17){\rule{0.700pt}{0.400pt}}
\multiput(945.00,680.17)(1.547,-2.000){2}{\rule{0.350pt}{0.400pt}}
\multiput(948.61,675.82)(0.447,-0.909){3}{\rule{0.108pt}{0.767pt}}
\multiput(947.17,677.41)(3.000,-3.409){2}{\rule{0.400pt}{0.383pt}}
\put(951,674.17){\rule{0.900pt}{0.400pt}}
\multiput(951.00,673.17)(2.132,2.000){2}{\rule{0.450pt}{0.400pt}}
\put(935.0,683.0){\rule[-0.200pt]{0.723pt}{0.400pt}}
\put(964,675.67){\rule{0.964pt}{0.400pt}}
\multiput(964.00,675.17)(2.000,1.000){2}{\rule{0.482pt}{0.400pt}}
\put(968,677.17){\rule{0.700pt}{0.400pt}}
\multiput(968.00,676.17)(1.547,2.000){2}{\rule{0.350pt}{0.400pt}}
\multiput(971.00,677.95)(0.462,-0.447){3}{\rule{0.500pt}{0.108pt}}
\multiput(971.00,678.17)(1.962,-3.000){2}{\rule{0.250pt}{0.400pt}}
\put(974,675.67){\rule{0.964pt}{0.400pt}}
\multiput(974.00,675.17)(2.000,1.000){2}{\rule{0.482pt}{0.400pt}}
\put(978,675.17){\rule{0.700pt}{0.400pt}}
\multiput(978.00,676.17)(1.547,-2.000){2}{\rule{0.350pt}{0.400pt}}
\multiput(981.61,672.37)(0.447,-0.685){3}{\rule{0.108pt}{0.633pt}}
\multiput(980.17,673.69)(3.000,-2.685){2}{\rule{0.400pt}{0.317pt}}
\put(984,670.67){\rule{0.723pt}{0.400pt}}
\multiput(984.00,670.17)(1.500,1.000){2}{\rule{0.361pt}{0.400pt}}
\put(987,672.17){\rule{0.900pt}{0.400pt}}
\multiput(987.00,671.17)(2.132,2.000){2}{\rule{0.450pt}{0.400pt}}
\put(955.0,676.0){\rule[-0.200pt]{2.168pt}{0.400pt}}
\put(994,673.67){\rule{0.723pt}{0.400pt}}
\multiput(994.00,673.17)(1.500,1.000){2}{\rule{0.361pt}{0.400pt}}
\put(991.0,674.0){\rule[-0.200pt]{0.723pt}{0.400pt}}
\put(1000,674.67){\rule{0.964pt}{0.400pt}}
\multiput(1000.00,674.17)(2.000,1.000){2}{\rule{0.482pt}{0.400pt}}
\put(1004,675.67){\rule{0.723pt}{0.400pt}}
\multiput(1004.00,675.17)(1.500,1.000){2}{\rule{0.361pt}{0.400pt}}
\put(1007,675.67){\rule{0.723pt}{0.400pt}}
\multiput(1007.00,676.17)(1.500,-1.000){2}{\rule{0.361pt}{0.400pt}}
\put(1010,676.17){\rule{0.900pt}{0.400pt}}
\multiput(1010.00,675.17)(2.132,2.000){2}{\rule{0.450pt}{0.400pt}}
\put(1014,677.67){\rule{0.723pt}{0.400pt}}
\multiput(1014.00,677.17)(1.500,1.000){2}{\rule{0.361pt}{0.400pt}}
\put(1017,677.67){\rule{0.723pt}{0.400pt}}
\multiput(1017.00,678.17)(1.500,-1.000){2}{\rule{0.361pt}{0.400pt}}
\put(1020,676.67){\rule{0.723pt}{0.400pt}}
\multiput(1020.00,677.17)(1.500,-1.000){2}{\rule{0.361pt}{0.400pt}}
\put(997.0,675.0){\rule[-0.200pt]{0.723pt}{0.400pt}}
\put(1027,675.67){\rule{0.723pt}{0.400pt}}
\multiput(1027.00,676.17)(1.500,-1.000){2}{\rule{0.361pt}{0.400pt}}
\multiput(1030.61,676.00)(0.447,0.685){3}{\rule{0.108pt}{0.633pt}}
\multiput(1029.17,676.00)(3.000,2.685){2}{\rule{0.400pt}{0.317pt}}
\multiput(1033.61,677.37)(0.447,-0.685){3}{\rule{0.108pt}{0.633pt}}
\multiput(1032.17,678.69)(3.000,-2.685){2}{\rule{0.400pt}{0.317pt}}
\put(1036,675.67){\rule{0.964pt}{0.400pt}}
\multiput(1036.00,675.17)(2.000,1.000){2}{\rule{0.482pt}{0.400pt}}
\put(1040,676.67){\rule{0.723pt}{0.400pt}}
\multiput(1040.00,676.17)(1.500,1.000){2}{\rule{0.361pt}{0.400pt}}
\multiput(1043.00,676.95)(0.462,-0.447){3}{\rule{0.500pt}{0.108pt}}
\multiput(1043.00,677.17)(1.962,-3.000){2}{\rule{0.250pt}{0.400pt}}
\put(1046,673.67){\rule{0.964pt}{0.400pt}}
\multiput(1046.00,674.17)(2.000,-1.000){2}{\rule{0.482pt}{0.400pt}}
\multiput(1050.61,671.37)(0.447,-0.685){3}{\rule{0.108pt}{0.633pt}}
\multiput(1049.17,672.69)(3.000,-2.685){2}{\rule{0.400pt}{0.317pt}}
\put(1053,669.67){\rule{0.723pt}{0.400pt}}
\multiput(1053.00,669.17)(1.500,1.000){2}{\rule{0.361pt}{0.400pt}}
\put(1056,670.67){\rule{0.723pt}{0.400pt}}
\multiput(1056.00,670.17)(1.500,1.000){2}{\rule{0.361pt}{0.400pt}}
\put(1023.0,677.0){\rule[-0.200pt]{0.964pt}{0.400pt}}
\multiput(1063.00,672.61)(0.462,0.447){3}{\rule{0.500pt}{0.108pt}}
\multiput(1063.00,671.17)(1.962,3.000){2}{\rule{0.250pt}{0.400pt}}
\put(1066,673.17){\rule{0.700pt}{0.400pt}}
\multiput(1066.00,674.17)(1.547,-2.000){2}{\rule{0.350pt}{0.400pt}}
\put(1069,671.67){\rule{0.723pt}{0.400pt}}
\multiput(1069.00,672.17)(1.500,-1.000){2}{\rule{0.361pt}{0.400pt}}
\put(1072,670.67){\rule{0.964pt}{0.400pt}}
\multiput(1072.00,671.17)(2.000,-1.000){2}{\rule{0.482pt}{0.400pt}}
\put(1076,669.67){\rule{0.723pt}{0.400pt}}
\multiput(1076.00,670.17)(1.500,-1.000){2}{\rule{0.361pt}{0.400pt}}
\put(1079,668.17){\rule{0.700pt}{0.400pt}}
\multiput(1079.00,669.17)(1.547,-2.000){2}{\rule{0.350pt}{0.400pt}}
\put(1082,668.17){\rule{0.900pt}{0.400pt}}
\multiput(1082.00,667.17)(2.132,2.000){2}{\rule{0.450pt}{0.400pt}}
\put(1086,668.67){\rule{0.723pt}{0.400pt}}
\multiput(1086.00,669.17)(1.500,-1.000){2}{\rule{0.361pt}{0.400pt}}
\put(1089,667.67){\rule{0.723pt}{0.400pt}}
\multiput(1089.00,668.17)(1.500,-1.000){2}{\rule{0.361pt}{0.400pt}}
\multiput(1092.61,668.00)(0.447,0.685){3}{\rule{0.108pt}{0.633pt}}
\multiput(1091.17,668.00)(3.000,2.685){2}{\rule{0.400pt}{0.317pt}}
\put(1095,671.67){\rule{0.964pt}{0.400pt}}
\multiput(1095.00,671.17)(2.000,1.000){2}{\rule{0.482pt}{0.400pt}}
\multiput(1099.00,671.95)(0.462,-0.447){3}{\rule{0.500pt}{0.108pt}}
\multiput(1099.00,672.17)(1.962,-3.000){2}{\rule{0.250pt}{0.400pt}}
\multiput(1102.00,668.95)(0.462,-0.447){3}{\rule{0.500pt}{0.108pt}}
\multiput(1102.00,669.17)(1.962,-3.000){2}{\rule{0.250pt}{0.400pt}}
\multiput(1105.00,667.61)(0.462,0.447){3}{\rule{0.500pt}{0.108pt}}
\multiput(1105.00,666.17)(1.962,3.000){2}{\rule{0.250pt}{0.400pt}}
\put(1108,670.17){\rule{0.900pt}{0.400pt}}
\multiput(1108.00,669.17)(2.132,2.000){2}{\rule{0.450pt}{0.400pt}}
\put(1059.0,672.0){\rule[-0.200pt]{0.964pt}{0.400pt}}
\put(1115,670.17){\rule{0.700pt}{0.400pt}}
\multiput(1115.00,671.17)(1.547,-2.000){2}{\rule{0.350pt}{0.400pt}}
\put(1112.0,672.0){\rule[-0.200pt]{0.723pt}{0.400pt}}
\put(1122,670.17){\rule{0.700pt}{0.400pt}}
\multiput(1122.00,669.17)(1.547,2.000){2}{\rule{0.350pt}{0.400pt}}
\put(1118.0,670.0){\rule[-0.200pt]{0.964pt}{0.400pt}}
\put(1131,671.67){\rule{0.964pt}{0.400pt}}
\multiput(1131.00,671.17)(2.000,1.000){2}{\rule{0.482pt}{0.400pt}}
\put(1135,671.17){\rule{0.700pt}{0.400pt}}
\multiput(1135.00,672.17)(1.547,-2.000){2}{\rule{0.350pt}{0.400pt}}
\put(1138,670.67){\rule{0.723pt}{0.400pt}}
\multiput(1138.00,670.17)(1.500,1.000){2}{\rule{0.361pt}{0.400pt}}
\put(1141,671.67){\rule{0.723pt}{0.400pt}}
\multiput(1141.00,671.17)(1.500,1.000){2}{\rule{0.361pt}{0.400pt}}
\put(1144,672.67){\rule{0.964pt}{0.400pt}}
\multiput(1144.00,672.17)(2.000,1.000){2}{\rule{0.482pt}{0.400pt}}
\put(1125.0,672.0){\rule[-0.200pt]{1.445pt}{0.400pt}}
\put(1154,672.67){\rule{0.964pt}{0.400pt}}
\multiput(1154.00,673.17)(2.000,-1.000){2}{\rule{0.482pt}{0.400pt}}
\multiput(1158.00,671.95)(0.462,-0.447){3}{\rule{0.500pt}{0.108pt}}
\multiput(1158.00,672.17)(1.962,-3.000){2}{\rule{0.250pt}{0.400pt}}
\put(1161,670.17){\rule{0.700pt}{0.400pt}}
\multiput(1161.00,669.17)(1.547,2.000){2}{\rule{0.350pt}{0.400pt}}
\put(1164,670.67){\rule{0.723pt}{0.400pt}}
\multiput(1164.00,671.17)(1.500,-1.000){2}{\rule{0.361pt}{0.400pt}}
\put(1167,671.17){\rule{0.900pt}{0.400pt}}
\multiput(1167.00,670.17)(2.132,2.000){2}{\rule{0.450pt}{0.400pt}}
\put(1171,672.67){\rule{0.723pt}{0.400pt}}
\multiput(1171.00,672.17)(1.500,1.000){2}{\rule{0.361pt}{0.400pt}}
\put(1148.0,674.0){\rule[-0.200pt]{1.445pt}{0.400pt}}
\put(1190,674.17){\rule{0.900pt}{0.400pt}}
\multiput(1190.00,673.17)(2.132,2.000){2}{\rule{0.450pt}{0.400pt}}
\put(1194,676.17){\rule{0.700pt}{0.400pt}}
\multiput(1194.00,675.17)(1.547,2.000){2}{\rule{0.350pt}{0.400pt}}
\put(1174.0,674.0){\rule[-0.200pt]{3.854pt}{0.400pt}}
\put(1200,678.17){\rule{0.700pt}{0.400pt}}
\multiput(1200.00,677.17)(1.547,2.000){2}{\rule{0.350pt}{0.400pt}}
\put(1197.0,678.0){\rule[-0.200pt]{0.723pt}{0.400pt}}
\put(1207,679.67){\rule{0.723pt}{0.400pt}}
\multiput(1207.00,679.17)(1.500,1.000){2}{\rule{0.361pt}{0.400pt}}
\put(1210,679.67){\rule{0.723pt}{0.400pt}}
\multiput(1210.00,680.17)(1.500,-1.000){2}{\rule{0.361pt}{0.400pt}}
\put(1213,680.17){\rule{0.700pt}{0.400pt}}
\multiput(1213.00,679.17)(1.547,2.000){2}{\rule{0.350pt}{0.400pt}}
\put(1216,680.67){\rule{0.964pt}{0.400pt}}
\multiput(1216.00,681.17)(2.000,-1.000){2}{\rule{0.482pt}{0.400pt}}
\multiput(1220.00,681.61)(0.462,0.447){3}{\rule{0.500pt}{0.108pt}}
\multiput(1220.00,680.17)(1.962,3.000){2}{\rule{0.250pt}{0.400pt}}
\put(1223,684.17){\rule{0.700pt}{0.400pt}}
\multiput(1223.00,683.17)(1.547,2.000){2}{\rule{0.350pt}{0.400pt}}
\put(1203.0,680.0){\rule[-0.200pt]{0.964pt}{0.400pt}}
\put(1230,686.17){\rule{0.700pt}{0.400pt}}
\multiput(1230.00,685.17)(1.547,2.000){2}{\rule{0.350pt}{0.400pt}}
\put(1233,686.67){\rule{0.723pt}{0.400pt}}
\multiput(1233.00,687.17)(1.500,-1.000){2}{\rule{0.361pt}{0.400pt}}
\put(1236,685.17){\rule{0.700pt}{0.400pt}}
\multiput(1236.00,686.17)(1.547,-2.000){2}{\rule{0.350pt}{0.400pt}}
\put(1226.0,686.0){\rule[-0.200pt]{0.964pt}{0.400pt}}
\put(1243,684.67){\rule{0.723pt}{0.400pt}}
\multiput(1243.00,684.17)(1.500,1.000){2}{\rule{0.361pt}{0.400pt}}
\put(1239.0,685.0){\rule[-0.200pt]{0.964pt}{0.400pt}}
\put(1252,685.67){\rule{0.964pt}{0.400pt}}
\multiput(1252.00,685.17)(2.000,1.000){2}{\rule{0.482pt}{0.400pt}}
\put(1256,686.67){\rule{0.723pt}{0.400pt}}
\multiput(1256.00,686.17)(1.500,1.000){2}{\rule{0.361pt}{0.400pt}}
\put(1259,686.67){\rule{0.723pt}{0.400pt}}
\multiput(1259.00,687.17)(1.500,-1.000){2}{\rule{0.361pt}{0.400pt}}
\put(1262,687.17){\rule{0.900pt}{0.400pt}}
\multiput(1262.00,686.17)(2.132,2.000){2}{\rule{0.450pt}{0.400pt}}
\multiput(1266.00,687.95)(0.462,-0.447){3}{\rule{0.500pt}{0.108pt}}
\multiput(1266.00,688.17)(1.962,-3.000){2}{\rule{0.250pt}{0.400pt}}
\put(1269,685.67){\rule{0.723pt}{0.400pt}}
\multiput(1269.00,685.17)(1.500,1.000){2}{\rule{0.361pt}{0.400pt}}
\put(1272,687.17){\rule{0.700pt}{0.400pt}}
\multiput(1272.00,686.17)(1.547,2.000){2}{\rule{0.350pt}{0.400pt}}
\put(1246.0,686.0){\rule[-0.200pt]{1.445pt}{0.400pt}}
\put(1279,687.67){\rule{0.723pt}{0.400pt}}
\multiput(1279.00,688.17)(1.500,-1.000){2}{\rule{0.361pt}{0.400pt}}
\put(1282,686.17){\rule{0.700pt}{0.400pt}}
\multiput(1282.00,687.17)(1.547,-2.000){2}{\rule{0.350pt}{0.400pt}}
\put(1285,685.67){\rule{0.723pt}{0.400pt}}
\multiput(1285.00,685.17)(1.500,1.000){2}{\rule{0.361pt}{0.400pt}}
\put(1275.0,689.0){\rule[-0.200pt]{0.964pt}{0.400pt}}
\put(1292,685.67){\rule{0.723pt}{0.400pt}}
\multiput(1292.00,686.17)(1.500,-1.000){2}{\rule{0.361pt}{0.400pt}}
\put(1295,685.67){\rule{0.723pt}{0.400pt}}
\multiput(1295.00,685.17)(1.500,1.000){2}{\rule{0.361pt}{0.400pt}}
\put(1288.0,687.0){\rule[-0.200pt]{0.964pt}{0.400pt}}
\put(1308,685.67){\rule{0.723pt}{0.400pt}}
\multiput(1308.00,686.17)(1.500,-1.000){2}{\rule{0.361pt}{0.400pt}}
\put(1298.0,687.0){\rule[-0.200pt]{2.409pt}{0.400pt}}
\put(1315,684.67){\rule{0.723pt}{0.400pt}}
\multiput(1315.00,685.17)(1.500,-1.000){2}{\rule{0.361pt}{0.400pt}}
\put(1318,685.17){\rule{0.700pt}{0.400pt}}
\multiput(1318.00,684.17)(1.547,2.000){2}{\rule{0.350pt}{0.400pt}}
\put(1311.0,686.0){\rule[-0.200pt]{0.964pt}{0.400pt}}
\put(1347,686.67){\rule{0.964pt}{0.400pt}}
\multiput(1347.00,686.17)(2.000,1.000){2}{\rule{0.482pt}{0.400pt}}
\put(1351,687.67){\rule{0.723pt}{0.400pt}}
\multiput(1351.00,687.17)(1.500,1.000){2}{\rule{0.361pt}{0.400pt}}
\put(1354,689.17){\rule{0.700pt}{0.400pt}}
\multiput(1354.00,688.17)(1.547,2.000){2}{\rule{0.350pt}{0.400pt}}
\put(1357,690.67){\rule{0.723pt}{0.400pt}}
\multiput(1357.00,690.17)(1.500,1.000){2}{\rule{0.361pt}{0.400pt}}
\put(1360,691.67){\rule{0.964pt}{0.400pt}}
\multiput(1360.00,691.17)(2.000,1.000){2}{\rule{0.482pt}{0.400pt}}
\put(1364,693.17){\rule{0.700pt}{0.400pt}}
\multiput(1364.00,692.17)(1.547,2.000){2}{\rule{0.350pt}{0.400pt}}
\put(1367,694.67){\rule{0.723pt}{0.400pt}}
\multiput(1367.00,694.17)(1.500,1.000){2}{\rule{0.361pt}{0.400pt}}
\put(1321.0,687.0){\rule[-0.200pt]{6.263pt}{0.400pt}}
\put(1374,694.67){\rule{0.723pt}{0.400pt}}
\multiput(1374.00,695.17)(1.500,-1.000){2}{\rule{0.361pt}{0.400pt}}
\put(1370.0,696.0){\rule[-0.200pt]{0.964pt}{0.400pt}}
\multiput(1380.00,695.61)(0.462,0.447){3}{\rule{0.500pt}{0.108pt}}
\multiput(1380.00,694.17)(1.962,3.000){2}{\rule{0.250pt}{0.400pt}}
\put(1383,696.17){\rule{0.900pt}{0.400pt}}
\multiput(1383.00,697.17)(2.132,-2.000){2}{\rule{0.450pt}{0.400pt}}
\put(1387,694.17){\rule{0.700pt}{0.400pt}}
\multiput(1387.00,695.17)(1.547,-2.000){2}{\rule{0.350pt}{0.400pt}}
\put(1390,693.67){\rule{0.723pt}{0.400pt}}
\multiput(1390.00,693.17)(1.500,1.000){2}{\rule{0.361pt}{0.400pt}}
\put(1377.0,695.0){\rule[-0.200pt]{0.723pt}{0.400pt}}
\put(1406,694.67){\rule{0.964pt}{0.400pt}}
\multiput(1406.00,694.17)(2.000,1.000){2}{\rule{0.482pt}{0.400pt}}
\put(1410,695.67){\rule{0.723pt}{0.400pt}}
\multiput(1410.00,695.17)(1.500,1.000){2}{\rule{0.361pt}{0.400pt}}
\put(1413,696.67){\rule{0.723pt}{0.400pt}}
\multiput(1413.00,696.17)(1.500,1.000){2}{\rule{0.361pt}{0.400pt}}
\put(1416,697.67){\rule{0.723pt}{0.400pt}}
\multiput(1416.00,697.17)(1.500,1.000){2}{\rule{0.361pt}{0.400pt}}
\put(1419,698.67){\rule{0.964pt}{0.400pt}}
\multiput(1419.00,698.17)(2.000,1.000){2}{\rule{0.482pt}{0.400pt}}
\put(1393.0,695.0){\rule[-0.200pt]{3.132pt}{0.400pt}}
\put(1432,699.67){\rule{0.964pt}{0.400pt}}
\multiput(1432.00,699.17)(2.000,1.000){2}{\rule{0.482pt}{0.400pt}}
\put(1423.0,700.0){\rule[-0.200pt]{2.168pt}{0.400pt}}
\put(1436.0,701.0){\rule[-0.200pt]{0.723pt}{0.400pt}}
\put(130.0,131.0){\rule[-0.200pt]{0.400pt}{175.375pt}}
\put(130.0,131.0){\rule[-0.200pt]{315.338pt}{0.400pt}}
\put(1439.0,131.0){\rule[-0.200pt]{0.400pt}{175.375pt}}
\put(130.0,859.0){\rule[-0.200pt]{315.338pt}{0.400pt}}
\end{picture}

    \caption{Espectro de Lyapunov en el primer punto fijo}
\end{figure}

\begin{figure}[H]
    \centering
    % GNUPLOT: LaTeX picture
\setlength{\unitlength}{0.240900pt}
\ifx\plotpoint\undefined\newsavebox{\plotpoint}\fi
\sbox{\plotpoint}{\rule[-0.200pt]{0.400pt}{0.400pt}}%
\begin{picture}(1500,900)(0,0)
\sbox{\plotpoint}{\rule[-0.200pt]{0.400pt}{0.400pt}}%
\put(130.0,131.0){\rule[-0.200pt]{4.818pt}{0.400pt}}
\put(110,131){\makebox(0,0)[r]{$5.41$}}
\put(1419.0,131.0){\rule[-0.200pt]{4.818pt}{0.400pt}}
\put(130.0,204.0){\rule[-0.200pt]{4.818pt}{0.400pt}}
\put(110,204){\makebox(0,0)[r]{$5.42$}}
\put(1419.0,204.0){\rule[-0.200pt]{4.818pt}{0.400pt}}
\put(130.0,277.0){\rule[-0.200pt]{4.818pt}{0.400pt}}
\put(110,277){\makebox(0,0)[r]{$5.43$}}
\put(1419.0,277.0){\rule[-0.200pt]{4.818pt}{0.400pt}}
\put(130.0,349.0){\rule[-0.200pt]{4.818pt}{0.400pt}}
\put(110,349){\makebox(0,0)[r]{$5.44$}}
\put(1419.0,349.0){\rule[-0.200pt]{4.818pt}{0.400pt}}
\put(130.0,422.0){\rule[-0.200pt]{4.818pt}{0.400pt}}
\put(110,422){\makebox(0,0)[r]{$5.45$}}
\put(1419.0,422.0){\rule[-0.200pt]{4.818pt}{0.400pt}}
\put(130.0,495.0){\rule[-0.200pt]{4.818pt}{0.400pt}}
\put(110,495){\makebox(0,0)[r]{$5.46$}}
\put(1419.0,495.0){\rule[-0.200pt]{4.818pt}{0.400pt}}
\put(130.0,568.0){\rule[-0.200pt]{4.818pt}{0.400pt}}
\put(110,568){\makebox(0,0)[r]{$5.47$}}
\put(1419.0,568.0){\rule[-0.200pt]{4.818pt}{0.400pt}}
\put(130.0,641.0){\rule[-0.200pt]{4.818pt}{0.400pt}}
\put(110,641){\makebox(0,0)[r]{$5.48$}}
\put(1419.0,641.0){\rule[-0.200pt]{4.818pt}{0.400pt}}
\put(130.0,713.0){\rule[-0.200pt]{4.818pt}{0.400pt}}
\put(110,713){\makebox(0,0)[r]{$5.49$}}
\put(1419.0,713.0){\rule[-0.200pt]{4.818pt}{0.400pt}}
\put(130.0,786.0){\rule[-0.200pt]{4.818pt}{0.400pt}}
\put(110,786){\makebox(0,0)[r]{$5.5$}}
\put(1419.0,786.0){\rule[-0.200pt]{4.818pt}{0.400pt}}
\put(130.0,859.0){\rule[-0.200pt]{4.818pt}{0.400pt}}
\put(110,859){\makebox(0,0)[r]{$5.51$}}
\put(1419.0,859.0){\rule[-0.200pt]{4.818pt}{0.400pt}}
\put(130.0,131.0){\rule[-0.200pt]{0.400pt}{4.818pt}}
\put(130,90){\makebox(0,0){$0$}}
\put(130.0,839.0){\rule[-0.200pt]{0.400pt}{4.818pt}}
\put(261.0,131.0){\rule[-0.200pt]{0.400pt}{4.818pt}}
\put(261,90){\makebox(0,0){$1000$}}
\put(261.0,839.0){\rule[-0.200pt]{0.400pt}{4.818pt}}
\put(392.0,131.0){\rule[-0.200pt]{0.400pt}{4.818pt}}
\put(392,90){\makebox(0,0){$2000$}}
\put(392.0,839.0){\rule[-0.200pt]{0.400pt}{4.818pt}}
\put(523.0,131.0){\rule[-0.200pt]{0.400pt}{4.818pt}}
\put(523,90){\makebox(0,0){$3000$}}
\put(523.0,839.0){\rule[-0.200pt]{0.400pt}{4.818pt}}
\put(654.0,131.0){\rule[-0.200pt]{0.400pt}{4.818pt}}
\put(654,90){\makebox(0,0){$4000$}}
\put(654.0,839.0){\rule[-0.200pt]{0.400pt}{4.818pt}}
\put(784.0,131.0){\rule[-0.200pt]{0.400pt}{4.818pt}}
\put(784,90){\makebox(0,0){$5000$}}
\put(784.0,839.0){\rule[-0.200pt]{0.400pt}{4.818pt}}
\put(915.0,131.0){\rule[-0.200pt]{0.400pt}{4.818pt}}
\put(915,90){\makebox(0,0){$6000$}}
\put(915.0,839.0){\rule[-0.200pt]{0.400pt}{4.818pt}}
\put(1046.0,131.0){\rule[-0.200pt]{0.400pt}{4.818pt}}
\put(1046,90){\makebox(0,0){$7000$}}
\put(1046.0,839.0){\rule[-0.200pt]{0.400pt}{4.818pt}}
\put(1177.0,131.0){\rule[-0.200pt]{0.400pt}{4.818pt}}
\put(1177,90){\makebox(0,0){$8000$}}
\put(1177.0,839.0){\rule[-0.200pt]{0.400pt}{4.818pt}}
\put(1308.0,131.0){\rule[-0.200pt]{0.400pt}{4.818pt}}
\put(1308,90){\makebox(0,0){$9000$}}
\put(1308.0,839.0){\rule[-0.200pt]{0.400pt}{4.818pt}}
\put(1439.0,131.0){\rule[-0.200pt]{0.400pt}{4.818pt}}
\put(1439,90){\makebox(0,0){$10000$}}
\put(1439.0,839.0){\rule[-0.200pt]{0.400pt}{4.818pt}}
\put(130.0,131.0){\rule[-0.200pt]{0.400pt}{175.375pt}}
\put(130.0,131.0){\rule[-0.200pt]{315.338pt}{0.400pt}}
\put(1439.0,131.0){\rule[-0.200pt]{0.400pt}{175.375pt}}
\put(130.0,859.0){\rule[-0.200pt]{315.338pt}{0.400pt}}
\put(784,29){\makebox(0,0){$t$}}
\put(1339,819){\makebox(0,0){exponentes}}
\put(133,139){\usebox{\plotpoint}}
\multiput(133.60,139.00)(0.468,49.319){5}{\rule{0.113pt}{33.900pt}}
\multiput(132.17,139.00)(4.000,267.639){2}{\rule{0.400pt}{16.950pt}}
\multiput(137.61,477.00)(0.447,25.021){3}{\rule{0.108pt}{15.167pt}}
\multiput(136.17,477.00)(3.000,81.521){2}{\rule{0.400pt}{7.583pt}}
\multiput(140.61,590.00)(0.447,12.295){3}{\rule{0.108pt}{7.567pt}}
\multiput(139.17,590.00)(3.000,40.295){2}{\rule{0.400pt}{3.783pt}}
\multiput(143.61,646.00)(0.447,7.383){3}{\rule{0.108pt}{4.633pt}}
\multiput(142.17,646.00)(3.000,24.383){2}{\rule{0.400pt}{2.317pt}}
\multiput(146.60,680.00)(0.468,3.259){5}{\rule{0.113pt}{2.400pt}}
\multiput(145.17,680.00)(4.000,18.019){2}{\rule{0.400pt}{1.200pt}}
\multiput(150.61,703.00)(0.447,3.365){3}{\rule{0.108pt}{2.233pt}}
\multiput(149.17,703.00)(3.000,11.365){2}{\rule{0.400pt}{1.117pt}}
\multiput(153.61,719.00)(0.447,2.472){3}{\rule{0.108pt}{1.700pt}}
\multiput(152.17,719.00)(3.000,8.472){2}{\rule{0.400pt}{0.850pt}}
\multiput(156.61,731.00)(0.447,1.802){3}{\rule{0.108pt}{1.300pt}}
\multiput(155.17,731.00)(3.000,6.302){2}{\rule{0.400pt}{0.650pt}}
\multiput(159.60,740.00)(0.468,1.066){5}{\rule{0.113pt}{0.900pt}}
\multiput(158.17,740.00)(4.000,6.132){2}{\rule{0.400pt}{0.450pt}}
\multiput(163.61,748.00)(0.447,1.132){3}{\rule{0.108pt}{0.900pt}}
\multiput(162.17,748.00)(3.000,4.132){2}{\rule{0.400pt}{0.450pt}}
\multiput(166.61,754.00)(0.447,0.909){3}{\rule{0.108pt}{0.767pt}}
\multiput(165.17,754.00)(3.000,3.409){2}{\rule{0.400pt}{0.383pt}}
\multiput(169.60,759.00)(0.468,0.627){5}{\rule{0.113pt}{0.600pt}}
\multiput(168.17,759.00)(4.000,3.755){2}{\rule{0.400pt}{0.300pt}}
\multiput(173.00,764.61)(0.462,0.447){3}{\rule{0.500pt}{0.108pt}}
\multiput(173.00,763.17)(1.962,3.000){2}{\rule{0.250pt}{0.400pt}}
\multiput(176.00,767.61)(0.462,0.447){3}{\rule{0.500pt}{0.108pt}}
\multiput(176.00,766.17)(1.962,3.000){2}{\rule{0.250pt}{0.400pt}}
\multiput(179.00,770.61)(0.462,0.447){3}{\rule{0.500pt}{0.108pt}}
\multiput(179.00,769.17)(1.962,3.000){2}{\rule{0.250pt}{0.400pt}}
\multiput(182.00,773.61)(0.685,0.447){3}{\rule{0.633pt}{0.108pt}}
\multiput(182.00,772.17)(2.685,3.000){2}{\rule{0.317pt}{0.400pt}}
\put(186,776.17){\rule{0.700pt}{0.400pt}}
\multiput(186.00,775.17)(1.547,2.000){2}{\rule{0.350pt}{0.400pt}}
\put(189,778.17){\rule{0.700pt}{0.400pt}}
\multiput(189.00,777.17)(1.547,2.000){2}{\rule{0.350pt}{0.400pt}}
\put(192,780.17){\rule{0.700pt}{0.400pt}}
\multiput(192.00,779.17)(1.547,2.000){2}{\rule{0.350pt}{0.400pt}}
\put(195,781.67){\rule{0.964pt}{0.400pt}}
\multiput(195.00,781.17)(2.000,1.000){2}{\rule{0.482pt}{0.400pt}}
\put(199,783.17){\rule{0.700pt}{0.400pt}}
\multiput(199.00,782.17)(1.547,2.000){2}{\rule{0.350pt}{0.400pt}}
\put(202,784.67){\rule{0.723pt}{0.400pt}}
\multiput(202.00,784.17)(1.500,1.000){2}{\rule{0.361pt}{0.400pt}}
\put(205,785.67){\rule{0.964pt}{0.400pt}}
\multiput(205.00,785.17)(2.000,1.000){2}{\rule{0.482pt}{0.400pt}}
\put(209,787.17){\rule{0.700pt}{0.400pt}}
\multiput(209.00,786.17)(1.547,2.000){2}{\rule{0.350pt}{0.400pt}}
\put(212,788.67){\rule{0.723pt}{0.400pt}}
\multiput(212.00,788.17)(1.500,1.000){2}{\rule{0.361pt}{0.400pt}}
\put(215,789.67){\rule{0.723pt}{0.400pt}}
\multiput(215.00,789.17)(1.500,1.000){2}{\rule{0.361pt}{0.400pt}}
\put(222,790.67){\rule{0.723pt}{0.400pt}}
\multiput(222.00,790.17)(1.500,1.000){2}{\rule{0.361pt}{0.400pt}}
\put(225,791.67){\rule{0.723pt}{0.400pt}}
\multiput(225.00,791.17)(1.500,1.000){2}{\rule{0.361pt}{0.400pt}}
\put(228,792.67){\rule{0.723pt}{0.400pt}}
\multiput(228.00,792.17)(1.500,1.000){2}{\rule{0.361pt}{0.400pt}}
\put(218.0,791.0){\rule[-0.200pt]{0.964pt}{0.400pt}}
\put(235,793.67){\rule{0.723pt}{0.400pt}}
\multiput(235.00,793.17)(1.500,1.000){2}{\rule{0.361pt}{0.400pt}}
\put(238,794.67){\rule{0.723pt}{0.400pt}}
\multiput(238.00,794.17)(1.500,1.000){2}{\rule{0.361pt}{0.400pt}}
\put(231.0,794.0){\rule[-0.200pt]{0.964pt}{0.400pt}}
\put(245,795.67){\rule{0.723pt}{0.400pt}}
\multiput(245.00,795.17)(1.500,1.000){2}{\rule{0.361pt}{0.400pt}}
\put(241.0,796.0){\rule[-0.200pt]{0.964pt}{0.400pt}}
\put(251,796.67){\rule{0.723pt}{0.400pt}}
\multiput(251.00,796.17)(1.500,1.000){2}{\rule{0.361pt}{0.400pt}}
\put(248.0,797.0){\rule[-0.200pt]{0.723pt}{0.400pt}}
\put(258,797.67){\rule{0.723pt}{0.400pt}}
\multiput(258.00,797.17)(1.500,1.000){2}{\rule{0.361pt}{0.400pt}}
\put(254.0,798.0){\rule[-0.200pt]{0.964pt}{0.400pt}}
\put(267,798.67){\rule{0.964pt}{0.400pt}}
\multiput(267.00,798.17)(2.000,1.000){2}{\rule{0.482pt}{0.400pt}}
\put(261.0,799.0){\rule[-0.200pt]{1.445pt}{0.400pt}}
\put(274,799.67){\rule{0.723pt}{0.400pt}}
\multiput(274.00,799.17)(1.500,1.000){2}{\rule{0.361pt}{0.400pt}}
\put(271.0,800.0){\rule[-0.200pt]{0.723pt}{0.400pt}}
\put(284,800.67){\rule{0.723pt}{0.400pt}}
\multiput(284.00,800.17)(1.500,1.000){2}{\rule{0.361pt}{0.400pt}}
\put(277.0,801.0){\rule[-0.200pt]{1.686pt}{0.400pt}}
\put(297,801.67){\rule{0.723pt}{0.400pt}}
\multiput(297.00,801.17)(1.500,1.000){2}{\rule{0.361pt}{0.400pt}}
\put(287.0,802.0){\rule[-0.200pt]{2.409pt}{0.400pt}}
\put(310,802.67){\rule{0.723pt}{0.400pt}}
\multiput(310.00,802.17)(1.500,1.000){2}{\rule{0.361pt}{0.400pt}}
\put(300.0,803.0){\rule[-0.200pt]{2.409pt}{0.400pt}}
\put(326,803.67){\rule{0.964pt}{0.400pt}}
\multiput(326.00,803.17)(2.000,1.000){2}{\rule{0.482pt}{0.400pt}}
\put(313.0,804.0){\rule[-0.200pt]{3.132pt}{0.400pt}}
\put(346,804.67){\rule{0.723pt}{0.400pt}}
\multiput(346.00,804.17)(1.500,1.000){2}{\rule{0.361pt}{0.400pt}}
\put(330.0,805.0){\rule[-0.200pt]{3.854pt}{0.400pt}}
\put(372,805.67){\rule{0.723pt}{0.400pt}}
\multiput(372.00,805.17)(1.500,1.000){2}{\rule{0.361pt}{0.400pt}}
\put(349.0,806.0){\rule[-0.200pt]{5.541pt}{0.400pt}}
\put(402,806.67){\rule{0.723pt}{0.400pt}}
\multiput(402.00,806.17)(1.500,1.000){2}{\rule{0.361pt}{0.400pt}}
\put(375.0,807.0){\rule[-0.200pt]{6.504pt}{0.400pt}}
\put(441,807.67){\rule{0.723pt}{0.400pt}}
\multiput(441.00,807.17)(1.500,1.000){2}{\rule{0.361pt}{0.400pt}}
\put(405.0,808.0){\rule[-0.200pt]{8.672pt}{0.400pt}}
\put(490,808.67){\rule{0.723pt}{0.400pt}}
\multiput(490.00,808.17)(1.500,1.000){2}{\rule{0.361pt}{0.400pt}}
\put(444.0,809.0){\rule[-0.200pt]{11.081pt}{0.400pt}}
\put(562,809.67){\rule{0.723pt}{0.400pt}}
\multiput(562.00,809.17)(1.500,1.000){2}{\rule{0.361pt}{0.400pt}}
\put(493.0,810.0){\rule[-0.200pt]{16.622pt}{0.400pt}}
\put(667,810.67){\rule{0.723pt}{0.400pt}}
\multiput(667.00,810.17)(1.500,1.000){2}{\rule{0.361pt}{0.400pt}}
\put(565.0,811.0){\rule[-0.200pt]{24.572pt}{0.400pt}}
\put(840,811.67){\rule{0.723pt}{0.400pt}}
\multiput(840.00,811.17)(1.500,1.000){2}{\rule{0.361pt}{0.400pt}}
\put(670.0,812.0){\rule[-0.200pt]{40.953pt}{0.400pt}}
\put(1177,812.67){\rule{0.723pt}{0.400pt}}
\multiput(1177.00,812.17)(1.500,1.000){2}{\rule{0.361pt}{0.400pt}}
\put(843.0,813.0){\rule[-0.200pt]{80.461pt}{0.400pt}}
\put(1180.0,814.0){\rule[-0.200pt]{62.393pt}{0.400pt}}
\put(130.0,131.0){\rule[-0.200pt]{0.400pt}{175.375pt}}
\put(130.0,131.0){\rule[-0.200pt]{315.338pt}{0.400pt}}
\put(1439.0,131.0){\rule[-0.200pt]{0.400pt}{175.375pt}}
\put(130.0,859.0){\rule[-0.200pt]{315.338pt}{0.400pt}}
\end{picture}

    \caption{Espectro de Lyapunov en el segundo punto fijo}
\end{figure}

\section{Conclusiones}
El Modelo hyperca\'otico de Rossler tiene un comportamiento bastante raro, haciendo varias pruebas 
con puntos que distaban una unidad del primer punto fijo se obtienen resultados de nan para 
los 
exponentes y valores muy diferentes para la evoluci\'on de las variables mientras que para el 
segundo punto punto fijo casi cualquier valor diferente del punto fijo arroja resultados que 
afirman que el sistema diverge r\'apidamente cuando se inicia lejos del punto fijo. 

 
\begin{thebibliography}{9}
\bibitem{tesis}
K. Ramasubramanian, M.S. Sriram
	\emph{A comparative study of computation of Lyapunov spectra with different algorithms}
Department of Theoretical Physics, University of Madras, Guindy Campus, Chennai 600 025, India
\end{thebibliography}

\end{document}
