\documentclass[9pt]{article}
\usepackage[spanish]{babel}
\usepackage{amsmath}
\usepackage{epsfig}
\usepackage{graphicx,wrapfig,lipsum}
\begin{document}
\title{Mapeo log\'istico con raz\'on de crecimiento parab\'olica}
\author{Mauricio Yamil Tame Soria}	
\maketitle
	
El mapeo log\'istico es un modelo matem\'atico que describe la din\'amica de poblaciones. $N_n$ es el n\'umero de individuos en la 
generaci\'on $n$, $r$ es la raz\'on de crecimiento de la poblaci\'on. El n\'umero de individuos en la generaci\'on $n+1$ ser\'a 

\begin{equation}
N_{n+1}=rN_n
\label{name}
\end{equation}
	
Si consideramos que el par\'ametro $r$ se comporta como una par\'abola que depende de la poblaci\'on 
de la siguiente forma

\begin{figure}[h!]
\begin{center}
	% GNUPLOT: LaTeX picture
\setlength{\unitlength}{0.240900pt}
\ifx\plotpoint\undefined\newsavebox{\plotpoint}\fi
\sbox{\plotpoint}{\rule[-0.200pt]{0.400pt}{0.400pt}}%
\begin{picture}(900,450)(0,0)
\sbox{\plotpoint}{\rule[-0.200pt]{0.400pt}{0.400pt}}%
\put(91.0,90.0){\rule[-0.200pt]{4.818pt}{0.400pt}}
\put(819.0,90.0){\rule[-0.200pt]{4.818pt}{0.400pt}}
\put(91.0,154.0){\rule[-0.200pt]{4.818pt}{0.400pt}}
\put(819.0,154.0){\rule[-0.200pt]{4.818pt}{0.400pt}}
\put(91.0,218.0){\rule[-0.200pt]{4.818pt}{0.400pt}}
\put(819.0,218.0){\rule[-0.200pt]{4.818pt}{0.400pt}}
\put(91.0,282.0){\rule[-0.200pt]{4.818pt}{0.400pt}}
\put(819.0,282.0){\rule[-0.200pt]{4.818pt}{0.400pt}}
\put(91.0,346.0){\rule[-0.200pt]{4.818pt}{0.400pt}}
\put(819.0,346.0){\rule[-0.200pt]{4.818pt}{0.400pt}}
\put(91.0,410.0){\rule[-0.200pt]{4.818pt}{0.400pt}}
\put(819.0,410.0){\rule[-0.200pt]{4.818pt}{0.400pt}}
\put(91.0,90.0){\rule[-0.200pt]{0.400pt}{4.818pt}}
\put(91.0,390.0){\rule[-0.200pt]{0.400pt}{4.818pt}}
\put(241.0,90.0){\rule[-0.200pt]{0.400pt}{4.818pt}}
\put(241.0,390.0){\rule[-0.200pt]{0.400pt}{4.818pt}}
\put(390.0,90.0){\rule[-0.200pt]{0.400pt}{4.818pt}}
\put(390.0,390.0){\rule[-0.200pt]{0.400pt}{4.818pt}}
\put(540.0,90.0){\rule[-0.200pt]{0.400pt}{4.818pt}}
\put(540.0,390.0){\rule[-0.200pt]{0.400pt}{4.818pt}}
\put(689.0,90.0){\rule[-0.200pt]{0.400pt}{4.818pt}}
\put(689.0,390.0){\rule[-0.200pt]{0.400pt}{4.818pt}}
\put(839.0,90.0){\rule[-0.200pt]{0.400pt}{4.818pt}}
\put(839.0,390.0){\rule[-0.200pt]{0.400pt}{4.818pt}}
\put(91.0,90.0){\rule[-0.200pt]{0.400pt}{77.088pt}}
\put(91.0,90.0){\rule[-0.200pt]{180.193pt}{0.400pt}}
\put(839.0,90.0){\rule[-0.200pt]{0.400pt}{77.088pt}}
\put(91.0,410.0){\rule[-0.200pt]{180.193pt}{0.400pt}}
\put(30,250){\makebox(0,0){$r$}}
\put(465,29){\makebox(0,0){Numero de individuos}}
\put(749,370){\makebox(0,0){$\frac{N_{n+1}}{N_n}$}}
\put(91,410){\usebox{\plotpoint}}
\put(114,408.67){\rule{1.686pt}{0.400pt}}
\multiput(114.00,409.17)(3.500,-1.000){2}{\rule{0.843pt}{0.400pt}}
\put(91.0,410.0){\rule[-0.200pt]{5.541pt}{0.400pt}}
\put(136,407.67){\rule{1.927pt}{0.400pt}}
\multiput(136.00,408.17)(4.000,-1.000){2}{\rule{0.964pt}{0.400pt}}
\put(121.0,409.0){\rule[-0.200pt]{3.613pt}{0.400pt}}
\put(151,406.67){\rule{1.927pt}{0.400pt}}
\multiput(151.00,407.17)(4.000,-1.000){2}{\rule{0.964pt}{0.400pt}}
\put(144.0,408.0){\rule[-0.200pt]{1.686pt}{0.400pt}}
\put(167,405.67){\rule{1.686pt}{0.400pt}}
\multiput(167.00,406.17)(3.500,-1.000){2}{\rule{0.843pt}{0.400pt}}
\put(174,404.67){\rule{1.927pt}{0.400pt}}
\multiput(174.00,405.17)(4.000,-1.000){2}{\rule{0.964pt}{0.400pt}}
\put(182,403.67){\rule{1.686pt}{0.400pt}}
\multiput(182.00,404.17)(3.500,-1.000){2}{\rule{0.843pt}{0.400pt}}
\put(159.0,407.0){\rule[-0.200pt]{1.927pt}{0.400pt}}
\put(197,402.67){\rule{1.686pt}{0.400pt}}
\multiput(197.00,403.17)(3.500,-1.000){2}{\rule{0.843pt}{0.400pt}}
\put(204,401.67){\rule{1.927pt}{0.400pt}}
\multiput(204.00,402.17)(4.000,-1.000){2}{\rule{0.964pt}{0.400pt}}
\put(212,400.67){\rule{1.686pt}{0.400pt}}
\multiput(212.00,401.17)(3.500,-1.000){2}{\rule{0.843pt}{0.400pt}}
\put(219,399.17){\rule{1.700pt}{0.400pt}}
\multiput(219.00,400.17)(4.472,-2.000){2}{\rule{0.850pt}{0.400pt}}
\put(227,397.67){\rule{1.927pt}{0.400pt}}
\multiput(227.00,398.17)(4.000,-1.000){2}{\rule{0.964pt}{0.400pt}}
\put(235,396.67){\rule{1.686pt}{0.400pt}}
\multiput(235.00,397.17)(3.500,-1.000){2}{\rule{0.843pt}{0.400pt}}
\put(242,395.67){\rule{1.927pt}{0.400pt}}
\multiput(242.00,396.17)(4.000,-1.000){2}{\rule{0.964pt}{0.400pt}}
\put(250,394.17){\rule{1.500pt}{0.400pt}}
\multiput(250.00,395.17)(3.887,-2.000){2}{\rule{0.750pt}{0.400pt}}
\put(257,392.67){\rule{1.927pt}{0.400pt}}
\multiput(257.00,393.17)(4.000,-1.000){2}{\rule{0.964pt}{0.400pt}}
\put(265,391.17){\rule{1.500pt}{0.400pt}}
\multiput(265.00,392.17)(3.887,-2.000){2}{\rule{0.750pt}{0.400pt}}
\put(272,389.67){\rule{1.927pt}{0.400pt}}
\multiput(272.00,390.17)(4.000,-1.000){2}{\rule{0.964pt}{0.400pt}}
\put(280,388.17){\rule{1.500pt}{0.400pt}}
\multiput(280.00,389.17)(3.887,-2.000){2}{\rule{0.750pt}{0.400pt}}
\put(287,386.17){\rule{1.700pt}{0.400pt}}
\multiput(287.00,387.17)(4.472,-2.000){2}{\rule{0.850pt}{0.400pt}}
\put(295,384.17){\rule{1.700pt}{0.400pt}}
\multiput(295.00,385.17)(4.472,-2.000){2}{\rule{0.850pt}{0.400pt}}
\put(303,382.67){\rule{1.686pt}{0.400pt}}
\multiput(303.00,383.17)(3.500,-1.000){2}{\rule{0.843pt}{0.400pt}}
\put(310,381.17){\rule{1.700pt}{0.400pt}}
\multiput(310.00,382.17)(4.472,-2.000){2}{\rule{0.850pt}{0.400pt}}
\put(318,379.17){\rule{1.500pt}{0.400pt}}
\multiput(318.00,380.17)(3.887,-2.000){2}{\rule{0.750pt}{0.400pt}}
\put(325,377.17){\rule{1.700pt}{0.400pt}}
\multiput(325.00,378.17)(4.472,-2.000){2}{\rule{0.850pt}{0.400pt}}
\multiput(333.00,375.95)(1.355,-0.447){3}{\rule{1.033pt}{0.108pt}}
\multiput(333.00,376.17)(4.855,-3.000){2}{\rule{0.517pt}{0.400pt}}
\put(340,372.17){\rule{1.700pt}{0.400pt}}
\multiput(340.00,373.17)(4.472,-2.000){2}{\rule{0.850pt}{0.400pt}}
\put(348,370.17){\rule{1.500pt}{0.400pt}}
\multiput(348.00,371.17)(3.887,-2.000){2}{\rule{0.750pt}{0.400pt}}
\put(355,368.17){\rule{1.700pt}{0.400pt}}
\multiput(355.00,369.17)(4.472,-2.000){2}{\rule{0.850pt}{0.400pt}}
\multiput(363.00,366.95)(1.579,-0.447){3}{\rule{1.167pt}{0.108pt}}
\multiput(363.00,367.17)(5.579,-3.000){2}{\rule{0.583pt}{0.400pt}}
\put(371,363.17){\rule{1.500pt}{0.400pt}}
\multiput(371.00,364.17)(3.887,-2.000){2}{\rule{0.750pt}{0.400pt}}
\multiput(378.00,361.95)(1.579,-0.447){3}{\rule{1.167pt}{0.108pt}}
\multiput(378.00,362.17)(5.579,-3.000){2}{\rule{0.583pt}{0.400pt}}
\put(386,358.17){\rule{1.500pt}{0.400pt}}
\multiput(386.00,359.17)(3.887,-2.000){2}{\rule{0.750pt}{0.400pt}}
\multiput(393.00,356.95)(1.579,-0.447){3}{\rule{1.167pt}{0.108pt}}
\multiput(393.00,357.17)(5.579,-3.000){2}{\rule{0.583pt}{0.400pt}}
\multiput(401.00,353.95)(1.355,-0.447){3}{\rule{1.033pt}{0.108pt}}
\multiput(401.00,354.17)(4.855,-3.000){2}{\rule{0.517pt}{0.400pt}}
\put(408,350.17){\rule{1.700pt}{0.400pt}}
\multiput(408.00,351.17)(4.472,-2.000){2}{\rule{0.850pt}{0.400pt}}
\multiput(416.00,348.95)(1.355,-0.447){3}{\rule{1.033pt}{0.108pt}}
\multiput(416.00,349.17)(4.855,-3.000){2}{\rule{0.517pt}{0.400pt}}
\multiput(423.00,345.95)(1.579,-0.447){3}{\rule{1.167pt}{0.108pt}}
\multiput(423.00,346.17)(5.579,-3.000){2}{\rule{0.583pt}{0.400pt}}
\multiput(431.00,342.95)(1.579,-0.447){3}{\rule{1.167pt}{0.108pt}}
\multiput(431.00,343.17)(5.579,-3.000){2}{\rule{0.583pt}{0.400pt}}
\multiput(439.00,339.95)(1.355,-0.447){3}{\rule{1.033pt}{0.108pt}}
\multiput(439.00,340.17)(4.855,-3.000){2}{\rule{0.517pt}{0.400pt}}
\multiput(446.00,336.95)(1.579,-0.447){3}{\rule{1.167pt}{0.108pt}}
\multiput(446.00,337.17)(5.579,-3.000){2}{\rule{0.583pt}{0.400pt}}
\multiput(454.00,333.95)(1.355,-0.447){3}{\rule{1.033pt}{0.108pt}}
\multiput(454.00,334.17)(4.855,-3.000){2}{\rule{0.517pt}{0.400pt}}
\multiput(461.00,330.94)(1.066,-0.468){5}{\rule{0.900pt}{0.113pt}}
\multiput(461.00,331.17)(6.132,-4.000){2}{\rule{0.450pt}{0.400pt}}
\multiput(469.00,326.95)(1.355,-0.447){3}{\rule{1.033pt}{0.108pt}}
\multiput(469.00,327.17)(4.855,-3.000){2}{\rule{0.517pt}{0.400pt}}
\multiput(476.00,323.95)(1.579,-0.447){3}{\rule{1.167pt}{0.108pt}}
\multiput(476.00,324.17)(5.579,-3.000){2}{\rule{0.583pt}{0.400pt}}
\multiput(484.00,320.94)(0.920,-0.468){5}{\rule{0.800pt}{0.113pt}}
\multiput(484.00,321.17)(5.340,-4.000){2}{\rule{0.400pt}{0.400pt}}
\multiput(491.00,316.95)(1.579,-0.447){3}{\rule{1.167pt}{0.108pt}}
\multiput(491.00,317.17)(5.579,-3.000){2}{\rule{0.583pt}{0.400pt}}
\multiput(499.00,313.94)(1.066,-0.468){5}{\rule{0.900pt}{0.113pt}}
\multiput(499.00,314.17)(6.132,-4.000){2}{\rule{0.450pt}{0.400pt}}
\multiput(507.00,309.95)(1.355,-0.447){3}{\rule{1.033pt}{0.108pt}}
\multiput(507.00,310.17)(4.855,-3.000){2}{\rule{0.517pt}{0.400pt}}
\multiput(514.00,306.94)(1.066,-0.468){5}{\rule{0.900pt}{0.113pt}}
\multiput(514.00,307.17)(6.132,-4.000){2}{\rule{0.450pt}{0.400pt}}
\multiput(522.00,302.94)(0.920,-0.468){5}{\rule{0.800pt}{0.113pt}}
\multiput(522.00,303.17)(5.340,-4.000){2}{\rule{0.400pt}{0.400pt}}
\multiput(529.00,298.94)(1.066,-0.468){5}{\rule{0.900pt}{0.113pt}}
\multiput(529.00,299.17)(6.132,-4.000){2}{\rule{0.450pt}{0.400pt}}
\multiput(537.00,294.94)(0.920,-0.468){5}{\rule{0.800pt}{0.113pt}}
\multiput(537.00,295.17)(5.340,-4.000){2}{\rule{0.400pt}{0.400pt}}
\multiput(544.00,290.95)(1.579,-0.447){3}{\rule{1.167pt}{0.108pt}}
\multiput(544.00,291.17)(5.579,-3.000){2}{\rule{0.583pt}{0.400pt}}
\multiput(552.00,287.93)(0.710,-0.477){7}{\rule{0.660pt}{0.115pt}}
\multiput(552.00,288.17)(5.630,-5.000){2}{\rule{0.330pt}{0.400pt}}
\multiput(559.00,282.94)(1.066,-0.468){5}{\rule{0.900pt}{0.113pt}}
\multiput(559.00,283.17)(6.132,-4.000){2}{\rule{0.450pt}{0.400pt}}
\multiput(567.00,278.94)(1.066,-0.468){5}{\rule{0.900pt}{0.113pt}}
\multiput(567.00,279.17)(6.132,-4.000){2}{\rule{0.450pt}{0.400pt}}
\multiput(575.00,274.94)(0.920,-0.468){5}{\rule{0.800pt}{0.113pt}}
\multiput(575.00,275.17)(5.340,-4.000){2}{\rule{0.400pt}{0.400pt}}
\multiput(582.00,270.94)(1.066,-0.468){5}{\rule{0.900pt}{0.113pt}}
\multiput(582.00,271.17)(6.132,-4.000){2}{\rule{0.450pt}{0.400pt}}
\multiput(590.00,266.93)(0.710,-0.477){7}{\rule{0.660pt}{0.115pt}}
\multiput(590.00,267.17)(5.630,-5.000){2}{\rule{0.330pt}{0.400pt}}
\multiput(597.00,261.94)(1.066,-0.468){5}{\rule{0.900pt}{0.113pt}}
\multiput(597.00,262.17)(6.132,-4.000){2}{\rule{0.450pt}{0.400pt}}
\multiput(605.00,257.94)(0.920,-0.468){5}{\rule{0.800pt}{0.113pt}}
\multiput(605.00,258.17)(5.340,-4.000){2}{\rule{0.400pt}{0.400pt}}
\multiput(612.00,253.93)(0.821,-0.477){7}{\rule{0.740pt}{0.115pt}}
\multiput(612.00,254.17)(6.464,-5.000){2}{\rule{0.370pt}{0.400pt}}
\multiput(620.00,248.93)(0.710,-0.477){7}{\rule{0.660pt}{0.115pt}}
\multiput(620.00,249.17)(5.630,-5.000){2}{\rule{0.330pt}{0.400pt}}
\multiput(627.00,243.94)(1.066,-0.468){5}{\rule{0.900pt}{0.113pt}}
\multiput(627.00,244.17)(6.132,-4.000){2}{\rule{0.450pt}{0.400pt}}
\multiput(635.00,239.93)(0.821,-0.477){7}{\rule{0.740pt}{0.115pt}}
\multiput(635.00,240.17)(6.464,-5.000){2}{\rule{0.370pt}{0.400pt}}
\multiput(643.00,234.93)(0.710,-0.477){7}{\rule{0.660pt}{0.115pt}}
\multiput(643.00,235.17)(5.630,-5.000){2}{\rule{0.330pt}{0.400pt}}
\multiput(650.00,229.93)(0.821,-0.477){7}{\rule{0.740pt}{0.115pt}}
\multiput(650.00,230.17)(6.464,-5.000){2}{\rule{0.370pt}{0.400pt}}
\multiput(658.00,224.93)(0.710,-0.477){7}{\rule{0.660pt}{0.115pt}}
\multiput(658.00,225.17)(5.630,-5.000){2}{\rule{0.330pt}{0.400pt}}
\multiput(665.00,219.93)(0.821,-0.477){7}{\rule{0.740pt}{0.115pt}}
\multiput(665.00,220.17)(6.464,-5.000){2}{\rule{0.370pt}{0.400pt}}
\multiput(673.00,214.93)(0.710,-0.477){7}{\rule{0.660pt}{0.115pt}}
\multiput(673.00,215.17)(5.630,-5.000){2}{\rule{0.330pt}{0.400pt}}
\multiput(680.00,209.93)(0.821,-0.477){7}{\rule{0.740pt}{0.115pt}}
\multiput(680.00,210.17)(6.464,-5.000){2}{\rule{0.370pt}{0.400pt}}
\multiput(688.00,204.93)(0.710,-0.477){7}{\rule{0.660pt}{0.115pt}}
\multiput(688.00,205.17)(5.630,-5.000){2}{\rule{0.330pt}{0.400pt}}
\multiput(695.00,199.93)(0.821,-0.477){7}{\rule{0.740pt}{0.115pt}}
\multiput(695.00,200.17)(6.464,-5.000){2}{\rule{0.370pt}{0.400pt}}
\multiput(703.00,194.93)(0.671,-0.482){9}{\rule{0.633pt}{0.116pt}}
\multiput(703.00,195.17)(6.685,-6.000){2}{\rule{0.317pt}{0.400pt}}
\multiput(711.00,188.93)(0.710,-0.477){7}{\rule{0.660pt}{0.115pt}}
\multiput(711.00,189.17)(5.630,-5.000){2}{\rule{0.330pt}{0.400pt}}
\multiput(718.00,183.93)(0.821,-0.477){7}{\rule{0.740pt}{0.115pt}}
\multiput(718.00,184.17)(6.464,-5.000){2}{\rule{0.370pt}{0.400pt}}
\multiput(726.00,178.93)(0.581,-0.482){9}{\rule{0.567pt}{0.116pt}}
\multiput(726.00,179.17)(5.824,-6.000){2}{\rule{0.283pt}{0.400pt}}
\multiput(733.00,172.93)(0.821,-0.477){7}{\rule{0.740pt}{0.115pt}}
\multiput(733.00,173.17)(6.464,-5.000){2}{\rule{0.370pt}{0.400pt}}
\multiput(741.00,167.93)(0.581,-0.482){9}{\rule{0.567pt}{0.116pt}}
\multiput(741.00,168.17)(5.824,-6.000){2}{\rule{0.283pt}{0.400pt}}
\multiput(748.00,161.93)(0.671,-0.482){9}{\rule{0.633pt}{0.116pt}}
\multiput(748.00,162.17)(6.685,-6.000){2}{\rule{0.317pt}{0.400pt}}
\multiput(756.00,155.93)(0.581,-0.482){9}{\rule{0.567pt}{0.116pt}}
\multiput(756.00,156.17)(5.824,-6.000){2}{\rule{0.283pt}{0.400pt}}
\multiput(763.00,149.93)(0.821,-0.477){7}{\rule{0.740pt}{0.115pt}}
\multiput(763.00,150.17)(6.464,-5.000){2}{\rule{0.370pt}{0.400pt}}
\multiput(771.00,144.93)(0.671,-0.482){9}{\rule{0.633pt}{0.116pt}}
\multiput(771.00,145.17)(6.685,-6.000){2}{\rule{0.317pt}{0.400pt}}
\multiput(779.00,138.93)(0.581,-0.482){9}{\rule{0.567pt}{0.116pt}}
\multiput(779.00,139.17)(5.824,-6.000){2}{\rule{0.283pt}{0.400pt}}
\multiput(786.00,132.93)(0.671,-0.482){9}{\rule{0.633pt}{0.116pt}}
\multiput(786.00,133.17)(6.685,-6.000){2}{\rule{0.317pt}{0.400pt}}
\multiput(794.00,126.93)(0.581,-0.482){9}{\rule{0.567pt}{0.116pt}}
\multiput(794.00,127.17)(5.824,-6.000){2}{\rule{0.283pt}{0.400pt}}
\multiput(801.00,120.93)(0.569,-0.485){11}{\rule{0.557pt}{0.117pt}}
\multiput(801.00,121.17)(6.844,-7.000){2}{\rule{0.279pt}{0.400pt}}
\multiput(809.00,113.93)(0.581,-0.482){9}{\rule{0.567pt}{0.116pt}}
\multiput(809.00,114.17)(5.824,-6.000){2}{\rule{0.283pt}{0.400pt}}
\multiput(816.00,107.93)(0.671,-0.482){9}{\rule{0.633pt}{0.116pt}}
\multiput(816.00,108.17)(6.685,-6.000){2}{\rule{0.317pt}{0.400pt}}
\multiput(824.00,101.93)(0.492,-0.485){11}{\rule{0.500pt}{0.117pt}}
\multiput(824.00,102.17)(5.962,-7.000){2}{\rule{0.250pt}{0.400pt}}
\multiput(831.00,94.93)(0.671,-0.482){9}{\rule{0.633pt}{0.116pt}}
\multiput(831.00,95.17)(6.685,-6.000){2}{\rule{0.317pt}{0.400pt}}
\put(189.0,404.0){\rule[-0.200pt]{1.927pt}{0.400pt}}
\put(91.0,90.0){\rule[-0.200pt]{0.400pt}{77.088pt}}
\put(91.0,90.0){\rule[-0.200pt]{180.193pt}{0.400pt}}
\put(839.0,90.0){\rule[-0.200pt]{0.400pt}{77.088pt}}
\put(91.0,410.0){\rule[-0.200pt]{180.193pt}{0.400pt}}
\end{picture}

\end{center}
\end {figure}	

\begin{equation}
r(N_n)=R-\frac{N_n^2}{N_m^2}R=R(1-\frac{N_n^2}{N_m^2})
\label{name2}
\end{equation} 

donde $N_m$ es la poblaci\'on m\'axima que alcanza la especie que se est\'a modelando 
y $R$ es una constante, sustituyendo (\ref{name2}) en (\ref{name}) se obtiene
	
\begin{equation}
N_{n+1}=R(1-\frac{N_n^2}{N_m^2})N_n
\label{name3}
\end{equation}
	
La versi\'on normalizada del modelo se obtiene introduciendo la variable $x_n=\frac{N_n}{N_m}$ en la
ecuaci\'on (\ref{name3}).
	
\begin{equation}
x_{n+1}=Rx_n(1-x_n^2) \hspace{1cm} 0\leq x_n\leq1
\label{name4}
\end{equation}
	
\end{document}
