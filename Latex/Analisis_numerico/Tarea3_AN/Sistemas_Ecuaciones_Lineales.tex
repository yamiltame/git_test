\documentclass[10pt,twocolumn]{article}
\usepackage{graphicx}
\usepackage[spanish]{babel}
\usepackage{amsmath}
\usepackage{amssymb}

\begin{document}
\title{M\'etodos num\'ericos para resolver sistemas de ecuaciones lineales}
\author{Mauricio Yamil Tame Soria}
\maketitle

\begin{abstract}
En el texto se exponen dos m\'etodos para resolver sistemas de ecuaciones lineales, la eliminaci\'on de inc\'ognitas de
Gauss-Jordan y la descomposici\'on en factores LU. Se analizan algunas limitaciones y conflictos de los m\'etodos.
\end{abstract}

Palabras clave: eliminaci\'on Gauss-Jordan, Descomposici\'on LU

\section{Introducci\'on}
Eliminar inc\'ognitas consiste en sumar y restar las ecuaciones del sistema para reducirlo. La 
forma en la que se 
consigue es multiplicando las ecuaciones por 
constantes de tal manera que al combinarlas se eliminen las inc\'ognitas y quede al menos una ecuaci\'on de una sola 
inc\'ognita para 
hacer sustituci\'on con ese valor y conocer las dem\'as inc\'ognitas. 

\section{Eliminaci\'on de Gauss}
Este m\'etodo se utiliza para resolver un sistema general de $n$ ecuaciones con $n$ inc\'ognitas. El m\'etodo consiste en hacer 
una eliminaci\'on de inc\'ognitas sistem\'aticamente para obtener una matriz triangular superior y partiendo de ella calcular 
las inc\'ognitas con una sustituci\'on apoyada del valor obtenido de la ecuaci\'on de una sola inc\'ognita. Para ello primero 
hacemos la matriz aumentada con los coeficientes de las variables y los valores independientes de las ecuaciones, una vez que 
tenemos la matriz aumentada lo que se hace es transformar la parte de los coeficientes de la matriz aumentada en una matriz 
triangular superior, para poder transformar la matriz aumentada necesitamos evitar las divisiones por cero o las divisiones por 
n\'umeros chicos al momento de normalizar la ecuaci\'on, es decir, convertir el coeficiente de la variable en 1, para 
evitar estas situaciones lo que se utiliza es un 
pivoteo parcial. El pivoteo parcial consiste en reacomodar las filas de la matriz aumentada para obtener el mayor coeficiente 
posible que sirva como factor para la eliminaci\'on. Entonces primero se acomoda el sistema con un pivoteo parcial para obtener 
el mayor n\'umero como factor divisor despu\'es a la fila siguiente se le suma la ecuaci\'on anterior multiplicada por un factor 
tal que se elimine la inc\'ognita en la ecuaci\'on siguiente, por ejemplo en el sistema visto en clase

\[ \left( \begin{array}{cccc}
3 & -0.1 & -0.2 & 7.85 \\
0.1 & 7 & -0.3 & -19.3 \\
0.3 & -0.2 & 10 & 71.4 \end{array} \right) \]

el primer paso de la eliminaci\'on ya est\'a, es el pivoteo, es decir el mayor n\'umero de la primera columna es el 3 y est\'a 
arriba que es lo que necesitamos para calcular el factor $ m=\frac{A(2,1)}{A(1,1)}= \frac{-0.1}{3} $, es el factor por el cual 
se multiplica la ecuaci\'on correspondiente a la fila 1 para sumarla a la ecuaci\'on correspondiente a la fila 2 para eliminar 
la inc\'ognita 1 de la ecuaci\'on 2, de esta manera 
iterando sobre las filas y columnas se obtiene la matriz triangular, en el c\'odigo se puede observar la sitematizaci\'on del 
proceso. Una vez obtenida la matriz triangular se realiza la sustituci\'on hacia atr\'as para obtener los valores de las 
inc\'ognitas. 
\subsection{Gauss-Jordan}
La eliminaci\'on de Gauss-Jordan consiste en hacer la elminicai\'on de Gauss y despu\'es seguir el proceso hasta obtener una 
matriz diagonal, es decir, continuar eliminando inc\'ognitas hasta reducir el sistema a tres ecuaciones de una sola inc\'ognita, 
por ejemplo, tomamos la matriz aumentada del sistema de ecuaciones lineales visto en clase despu\'es de haber realizado la 
eliminaci\'on gaussiana

\[ \left( \begin{array}{cccc}
1 & -3.33E-2 & -6.66E-2 & 2.616 \\
0.0 & 1 & -4.188E-2 & -2.793 \\
0.0 & 0.0 & 1 & 7.0 \end{array} \right) \]

si quisieramos eliminar la variable 2 de la primera ecuaci\'on podr\'iamos sumarle a la ecuaci\'on 1 la ecuaci\'on 2 
multiplicada por el factor $m=3.333$ y de esta manera eliminar el valor de la segunda columna en la ecuaci\'on 1. Repitiendo el 
proceso para la fila 1 y 2 con la ecuaci\'on 3 obtenemos la matriz 
\[ \left( \begin{array}{cccc}
1 & 0 & 0 & 3.000 \\
0.0 & 1 & 0 & -2.500 \\
0.0 & 0.0 & 1 & 7.000 \end{array} \right) \]
de la cual obtenemos los valores de las inc\'ognitas.

\section{Descomposici\'on LU}
La descomposici\'on LU consiste en factorizar la matriz de coeficientes en una matriz triangular superior $U$ y una matriz 
triangular inferior $L$ de tal manera que $LU=A$, donde A es la matriz de coeficientes, cumpliendo esta condici\'on se deduce a 
partir de la ecuaci\'on 
\begin{equation}
	Ax=B
\end{equation}
 que 
\begin{equation}
	L \left( Ux - D \right) = Ax - B =0 
\end {equation}
por lo tanto $LU=A$ y $LD=B$ donde $x$ es el vector de soluciones y $B$ es el vector de valores independientes. Con esas 
igualdades podemos obtener $D$ haciendo sustituci\'on hacia adelante y despues usar $D$ sobre la igualdad $Ux=D$ para obtener 
$x$ con una sustituci\'on hacia atr\'as. Para hacer 
la 
descomposici\'on LU necesitamos hacer eliminaci\'on gaussiana sobre la matriz de coeficientes para obtener la matriz triangular 
superior $U$ entonces U es la matriz que resulta de eliminaci\'on gaussiana sobre la matriz de coeficientes mientras que la 
matriz triangular inferior $L$ se construye primero con una diagonal de unos y luego las entadas son los factores utilizados 
para hacer las eliminaciones de inc\'ognitas, 
en el ejemplo anterior 
consideramos el factor $m=\frac{A(2,1)}{A(1,1)}= \frac{-0.1}{3} $ para eliminar la primera inc\'ognita de la ecuaci\'on 2, dicho 
factor ser\'ia la entrada $l_{2,1}$ de la matriz triangular inferior $L$ y as\'i sistem\'aticamente, lo que nos garantiza esto 
es que al hacer la multiplicaci\'on $LU$ se obtiene la matriz original pues la multiplicaci\'on de dichas matrices representa 
las operaciones inversas de la triangulaci\'on de la matriz de coeficientes $A$.

\begin{thebibliography}
	\bibitem{chapra}
	Stephen C. Chapra
	\emph{M\'etodos num\'ericos para ingenieros}
	Mc. Graw Hill, 5th edition, 2007.
\end{thebibliography}

\end{document}
