\documentclass[10pt,twocolumn]{article}
\usepackage{graphicx}
\usepackage[spanish]{babel}
\usepackage{amsmath}
\usepackage{amssymb}

\begin{document}
\title{Errores de an\'alisis en los m\'etodos num\'ericos}
\author{Mauricio Yamil Tame Soria}
\maketitle

\begin{abstract}
Se exponen los errores que se presentan al momento de hacer an\'alisis 
num\'ericos. El error que se obtiene almacenando la 
informaci\'on en bits, error iterativo, error relativo y error absoluto.
\end{abstract}

Palabras clave: Error, an\'alisis num\'erico, punto flotante.

\section{Introducci\'on}
El error es la diferencia entre el valor obtenido por alg\'un m\'etodo o modelo y el valor real o esperado,es decir, qu\'e tanto se acerca 
un valor predecido u obtenido al valor que esperamos o conocemos. Es un indicador de la precisi\'on que tiene el m\'etodo o modelo. Conocer 
el comportaminto del error que tienen los m\'etodos y el almacenamiento de datos nos sirve para dar validez a los valores que se utilizan 
como argumentos.

\section{Error de almacenamiento}
La informaci\'on que guardamos en la computadora no es como la abstraemos cotidianamente, se encuentra representada en cadenas de bits, un 
bit es un valor de 0 \'o 1 que se almacena como un estado de apagado o encendido. Entonces dicho conjunto de estados nos puede representar 
un n\'umero en un sistema binario o una letra en clave morse.
\subsection{Punto flotante}
Un n\'umero $N$ en punto flotante est\'a representado de la siguiente forma:

\begin{equation}�
	N=m \times b^e \hspace{1cm} \frac{1}{b} \leq m \leq 1
	\label{flotante}
\end{equation}

\end{document}
