\documentclass[10pt,twocolumn]{article}
\usepackage{graphicx}
\usepackage[spanish]{babel}
\usepackage{amsmath}
\usepackage{amssymb}

\begin{document}
\title{Interpolaci\'on}
\author{Mauricio Yamil Tame Soria}
\maketitle

\begin{abstract}
Se describen m\'etodos de interpolaci\'on para un conjunto de puntos y se utilizan para hacer una 
interpolaci\'on con 8 valores que corresponden a una funci\'on conocida. Los m\'etodos para una 
variable son:   
Interpolaci\'on polinomial de Newton en diferencias divididas, polinomio 
de Lagrange y la interpolaci\'on con trazadores c\'ubicos.
\end{abstract}

\emph{Palabras clave: Interpolaci\'on, Newton, diferencias divididas, 
Lagrange, splines, interpolaci\'on 2D.}

\section{Introducci\'on}
Se explicar\'a brevemente en lo que consisten los m\'etodos para ponerlos en pr\'actica con un ejemplo.
Los datos que utilizaremos para probar los m\'etodos son 8 puntos que corresponden a la funci\'on 
logaritmo natural. Con los 8 puntos obtendremos el polinomio de Interpolaci\'on de $7$ grado. Mediante 
un programa en Fortran se calculan los coeficientes para el polinomio con ambos m\'etodos, el de Lagrange 
y el 
de 
Newton; despu\'es de obtener el polinomio estimamos el valor en 2, es decir, evaluamos $f(2)$ y lo 
comparamos con el valor conocido de $ln(2)$ para analizar el error en la estimaci\'on del polinomio.  

\section{Interpolaci\'on polinomial}
La interpolaci\'on polinomial es un m\'etodo que nos ayuda a estimar un valor partiendo de unos que ya 
conocemos. Dicha estimaci\'on se calcula con un polinomio $f(x)= \sum_{i=0}^{n-1} a_ix^{i}$ donde $n$ es 
el 
n\'umero de puntos que tenemos, los cuales son de la forma $(x,y)$. El m\'etodo de Newton calcula los 
coeficientes de $f(x)$, el m\'etodo de Lagrange construye el polinomio de tal 
manera que satisfaga $f(x_i)=y_i$, con los trazadores c\'ubicos se hace una interpolaci\'on en trozos(con 
pares de puntos).
Por ejemplo 
para dos puntos usamos un 
polinomio $f(x)=a_0 + a_1x$ que tiene la forma de una recta, el coeficiente $a_0$ es el 
primer valor 
$y_1$ y el coeficiente $a_1$ es la pendiente de la recta entonces $a_1=\frac{y_2-y_1}{x_2-x_1}$. Para un 
conjunto de $n$ puntos obtenemos un polinomio de grando $n-1$.

\subsection{Polinomio de Lagrange} 
El polinomio de Lagrange se construye con la intuici\'on de que el polinomio evaluado en $x_1$ debe ser 
$y_1$. Para un conjunto de $n+1$ puntos numerados $(x_0,y_0),...,(x_n,y_n)$ se formula el polinomio 
$f_n(x)=\sum_{i=0}^{n}L_i(x)y_i$ de tal manera que $f_n(x_i)=y_i$ eso 
implica que $L_i(x_j)=\delta_{ij}$, En este polinomio los valores $y_i$ actuan como coeficientes mientras 
que los $L_i(x)$ son polinomios de grado i, entonces $f_n$ es una suma de polinomios. para cumplir estas 
condiciones se construye 
el polinomio 
de la 
siguiente forma

\begin{equation}
	f_n(x)=\sum_{i=0}^{n}L_i(x)y_1
	\label{polf}
\end{equation}

\begin{equation}
	L_i(x)=\prod_{j=0}^{n}\frac{x-x_j}{x_i-x_j} \hspace{1cm} j\neq i
	\label{Lpol}
\end{equation}

Esta construcci\'on nos asegura que $f_n(x_i)=y_i$	

\subsection{Diferencias Divididas}
Para un conjunto de $n+1$ puntos numerados de 0 a $n$ el m\'etodo de Newton ofrece una sistematizaci\'on 
para obtener los coeficientes $a_i$ mediante un 
planteamiento de que $f$ tiene la forma de $f(x)=\sum_{i=0}^{n}b_i\prod_{j=0}^{i-1}(x-x_j)$, 
desarrollando 
los productos y reagrupando los t\'erminos reescribimos $f$ de la forma $f(x)=\sum_{i=0}^{n}a_ix^{i} $ lo 
que nos da un conjunto de ecuaciones $a_i=a_i(b_0,...,b_n)$, es decir, los coeficientes en t\'erminos 
de las $b_i$ las cuales se calculan partiendo de la condici\'on 
$f(x_i)=\sum_{j=0}^{i}b_i\prod_{k=0}^{j-1}(x-x_k)=y_i$. El m\'etodo de Newton proporciona una forma 
recursiva para obtener los $b_i$ de la siguiente forma, $b_0=y_0$, $b_i=d[x_i,...,x_0]$ y se define 
$d[x_i,...,x_0]$ como la diferencia dividia entre $x_i$ y $x_0$ y se calcula de la manera siguiente.

\begin{equation}
	d[x_i,x_{i-1}]=\frac{y_i-y_{i-1}}{x_i-x_{i-1}} \hspace{1cm}
	\label{dif}
\end{equation}

y la regla recursiva

\begin{equation}
	d[x_k,...,x_0]=\frac{d[x_k,...,x_1]-d[x_{k-1},...,x_0]}{x_k-x_0}
	\label{dif2}
\end{equation}

de esta forma se obtiene el polinomio de interpolaci\'on para los $n+1$ sustituyendo los $b_i$ lo que 
resulta en
\begin{equation} 	
	f_n(x)=y_0+\sum_{i=1}^{n}d[x_i,...,x_0]\prod_{j=0}^{i-1}(x-x_i)
	\label{newton}
\end{equation}

\subsection{Error de interpolaci\'on}
Para calcular el error de la estimaci\'on con el polinomio de Newton se observa que la ecuaci\'on 
(\ref{newton}) se parece a la serie de Taylor. Para calcular el error de truncamiento de Taylor se 
necesita la derivada de orden $(n+1)$ para calcular el error del polinomio se utiliza una diferencia 
dividida para aproximar dicha derivada, pero para eso ocupamos un punto extra, es decir, para calcular el 
error con $n$ puntos necesitamos tener $n+1$ puntos. Usando esa aproximaci\'on resulta que el error para 
la estimaci\'on con el valor $x$ es 

\begin{equation}
	E_n \approx d[x_{n+1},...,x_0]\prod_{i=0}^{n}(x-x_i)
	\label{mistake}
\end{equation}

Esta forma de calcular el error tambi\'es es v\'alida para el polinomio de Newton ya que los dos 
m\'etodos regresan el mismo polinomio, el \'unico polinomio de grado $n-1$ que pasa por los $n$ puntos. 
La diferencia que existe en los valores que devuelve un m\'etodo y el otro se debe al error de 
truncamiento que se comete al momento de almacenar y operar los datos en forma de bytes.

\subsection{Metodolog\'ia y resultados}
Se utilizan 8 puntos para hacer la interpolaci\'on, la tabla de valores es:

\begin{tabular}[h]{|l|l|}
	\hline
	x & y=$ln(x)$ \\ \hline
	1 & 0 \\
	4 & 1.3862944 \\
	6 & 1.7917595 \\
	5 & 1.6094379 \\
	3 & 1.0986123 \\
	1.5 & 0.4054641 \\
	2.5 & 0.9162907 \\
	3.5 & 1.2527630 \\ \hline
\end{tabular}

para el c\'alculo de las diferencias divididas se utiliza una matriz de $2n-1 \times n+1$ que se llena as\'i
\begin{tabular}[b]{|l|l|l|l}
	\hline
	$x_0$ & $y_0$ & 00 & $...$ \\
	00 & 00 & $d[x_0,x_1]$ & $...$ \\
	$x_1$ & $y_1$ & 00 & $d[x_2,x_1,x_0]$ \\
	00 & 00 & $d[x_2,x_1]$ & $...$ \\
	$x_2$ & $y_2$ & 00 & $d[x_3,x_2,x_1]$ \\
	$...$ & $...$ & $...$ & $...$ \\ \hline
\end{tabular}

entonces para los puntos de la tabla de valores calculamos el valor estimado para 2 usando el 
programa en Fortran codificado en clase para obtener el Cuadro 1 que representa los resultados obtenidos, los resultados fueron truncados hasta 7 
deimales para facilitar la visualizaci\'on, los valores con la presici\'on completa se obtienen ejecutando el c\'odigo. Analizando 
el cuadro 1 podemos observar que en 
general el error estimado disminuye a medida que se incrementa el n\'umero de puntos, para los errores absolutos podemos observar lo mismo pero esto se 
debe a que los puntos que interpolamos corresponden a la funci\'on que conocemos.

 

\begin{table*}[t]
	\centering
	\begin{tabular}{|l|l|l|l|l|l|}
		\hline
		puntos & Newton & Lagrange & Error Newton & Error Lagrange & Error estimado \\
		\hline
		1 & 0 & 0 & 0 & 0 & 0.4620981 \\ \hline
		2 & 0.4620981  & 0.4620981 & 0.231049 & 0.2310490 & 0.10374623 \\ \hline
		3 &0.56584436 &0.56584436  &0.12730281 &0.12730281 & 6.29243333E-002 \\ \hline
		4 & 0.62876867 & 0.6287687 & 6.43784805E-002& 6.43784805E-002 & 4.69531E-002 \\ \hline
		5 & 0.6757218 & 0.6757218& 1.74253805E-002 & 1.74253805E-002 & 2.17914927E-002 \\ \hline
		6 & 0.69751329 &0.69751329 &-4.36611213E-003 &-4.36611213E-003 &-3.61604254E-003 \\ \hline
		7 & 0.69389725 & 0.69389725 &-7.50069598E-004 &-7.50069598E-004 &-4.58899682E-004 \\ \hline
		8 & 0.69343835 & 0.69343835 &-2.91169912E-004 &-2.91169916E-004 & 000 \\ \hline
	\end{tabular}
	\caption{Resultados}
\end{table*}

\section{Trazadores c\'ubicos}
Interpolar los $n+1$ puntos nos devuelve un polinomio de grado $n$, sin embargo existen casos donde los polinomios devuelven resultados err\'oneos 
devido a errores de redondeo y la lejan\'ia entre los puntos. El m\'etodo de los trazadores c\'ubicos consiste en hacer polinomios de grado menor en 
subconjuntos de los datos, los polinomios son lo que llamamos trazadores. Los trazadores se vuelven m\'as precisos cuando la funci\'on es suave pero 
de repente cambia muy r\'apido, por lo tanto si la funci\'on tiene cambios locales e intensos ser\'a conveniente usar el m\'etodo de los trazadores. En 
este texto se expone el m\'etodo usando polinomios de grado 3 para la interpolaci\'on de los subconjuntos de puntos. Necesitamos un conjunto de 
polinomios de la forma $f_i(x)=a_ix^3 + b_ix^2 + c_ix + d_i$ donde el \'indice $i$ corresponde al n\'umero de intervalo, por ejemplo el polinomio 
$f_1(x)$ es el trazador asignado al intervalo del punto $(x_0,y_0)$ al punto $(x_1,y_1)$ entonces para $n+1$ puntos numerados $(x_0,...,x_n)$ existen 
$n$ 
intervalos y 
para 
cada 
intervalo un polinomio con 4 inc\'ognitas $a_i,b_i,c_i,d_i$ lo que nos resulta en $4n$ inc\'ognitas por lo que necesitamos $4n$ ecuaciones. Dichas 
ecuaciones se obtienen fijando condiciones las cuales son:

\begin{equation}
	f_i(x_i)=f_{i+1}(x_i)=y_i \hspace{1cm} i=1,...,n-1
\label{cond1}
\end{equation}

\begin{equation}
	f_1(x_0)=y_0 , f_n(x_n)=y_n
\label{cond2}
\end{equation}

\begin{equation}
	f_i'(x_i)=f_{i+1}'(x_i) \hspace{1cm} i=1,...,n-1
\label{cond3}
\end{equation}

\begin{equation}
	f_i''(x_i)=f_{i+1}''(x_i) \hspace{1cm} i=1,...,n-1
\label{cond4}
\end{equation}

\begin{equation}
	f_1''(x_1)=0 , f_n''(x_n)=0
\label{cond5}
\end{equation}

la condici\'on (\ref{cond1}) determina $2(n-1)$ ecuaciones, la condicion (\ref{cond2}) igual que la (\ref{cond5}) determinan 2 ecuaciones, la 
condici\'on
(\ref{cond3}) igual que la 
(\ref{cond4}) determinan $n-1$ ecuaciones, por lo tanto tenemos $2n-2 + 2 + n-1 + n-1 + 2 = 4n$ ecuaciones. Resolviendo el sistema de ecuaciones para 
obtener los coeficientes $a_i,b_i,c_i,d_i$ podemos construir los trazadores $f_i(x)$. Pero existe una alternativa m\'as eficiente que s\'olo requiere 
resolver $n-1$ ecuaciones. Para este m\'etodo observamos que la segunda derivada de cada $f_i(x)$ es una l\'inea recta entonces al derivar el polinomio
$f_i(x)=a_ix^3 + b_ix^2 + c_ix + d_i$ obtenemos 

\begin{equation}
	f_i''(x)=f_i''(x_{i-1})\frac{x-x_i}{x_{i-1}-x_i} + f_i''(x_i)\frac{x-x_{i-1}}{x_i-x_{i-1}}
	\label{spline}
\end{equation}

esta ecuaci\'on calcula la segunda derivada en el intervalo $i$ partiendo del conocimiento que es una recta que interpola $(x_{i-1},f_i''(x_{i-1}))$ y 
$(x_i,f_i''(x_i))$. Al integrar (\ref{spline}) dos veces obtenemos 

\begin{equation}
	f_i(x)=\frac{(x-x_1)^3f_i''(x_{i-1})}{6(x_{i-1}-x_i)}+\frac{(x-x_i)^3f_i''(x_i)}{6(x_i-x_{i-1})}+bx+c
	\label{int}
\end{equation}
 donde b y c son constantes de integraci\'on que obtenemos de las condiciones de igualdad de las funciones, es decir, $f_i(x_i)=y_i$ y 
$f_i(x_{i-1})=y_{i-1}$ entonces con esas condiciones despejamos por $b$ y por $c$ obteniendo la expresi\'on 

\begin{multline}
	f_i(x)=\frac{(x-x_1)^3f_i''(x_{i-1})}{6(x_{i-1}-x_i)}+\frac{(x-x_i)^3f_i''(x_i)}{6(x_i-x_{i-1})} \\
	+[\frac{y_{i-1}}{x_i-x_{i-1}} - \frac{f_i''(x_{i-1})(x_i-x_{i-1})}{6}](x_i-x) \\
	+[\frac{y_i}{x_i-x_{i-1}}-\frac{f_i''(x_i)(x_i-x_{i-1})}{6}](x-x_{i-1})
\end{multline}

Para esta ecuaci\'on s\'olo quedan dos valores desconocidos, $f_i''(x_{i-1})$ y $f_i''(x_i)$; las segundas derivadas se eval\'uan con la condici\'on de 
que las primeras derivadas son continuas, es decir, $f_i'(x_i)=f_{i-1}'(x_i)$, entonces derivamos la ecuaci\'on de $f_i(x)$ para meter la condici\'on y 
de 
esta manera 
obtenemos para todas las $f_i$ , $n-1$ ecuaciones con $n+1$ segundas derivadas desconocidas pero sabemos que en los extremos son cero, entonces sólo 
quedan $n-1$ segundas derivadas por encontrar que se hace resolviendo el sistema de $n-1$ ecuaciones con $n-1$ inc\'ognitas.

\begin{thebibliography}{9}
\bibitem{chapra}
	Stephen C. Chapra,
	\emph{M\'etodos num\'ericos para ingenieros}
	Mc. Graw Hill, 5th edition, 2007.
\end{thebibliography}

\end{document}
