\documentclass[10pt]{article}
\usepackage{graphicx}
\usepackage[spanish]{babel}
\usepackage{amsmath}
\usepackage{amssymb}

\begin{document}
\title{Puntos fijos de un sistema din\'amico de primer orden}
\author{Mauricio Yamil Tame Soria}
\maketitle

\begin{abstract}
Se determinan los puntos fijos y sus estabilidades para un sistema din\'amico particular. Tambi\'en se determina un valor donde se presenta 
una bifurcaci\'on y se hace un diagrama.
\end{abstract}

\emph{Palabras clave: sistema din\'amico, puntos fijos, estabilidad, bifurcaci\'on}

\section{Introducci\'on}
Analizaremos un sistema $f(x)=\dot{x}$. La ecuaci\'on representa la relaci\'on que hay entre una variable (en este caso $x$) con su cambio en 
el tiempo ($\dot{x}$). El an\'alisis ser\'a determinar los puntos fijos $x_f$ y 
verificar el valor de $\frac{df}{dx}(x_f)$ para determinar su 
estabilidad. La obtenci\'on de los puntos fijos se har\'a anal\'iticamente primero y despu\'es num\'ericamente utilizando el m\'etodo de 
Newton Raphson. La ecuaci\'on del sistema es:

\begin{equation}
	\dot{x}=x+\frac{rx}{1+x^2}
\label{sistema}
\end{equation}

\section{Puntos fijos}
Los puntos fijos $x_f$ cumplen $f(x_f)=0$, entonces para (\ref{sistema}) tenemos 
$x_f+\frac{rx_f}{1+x_f^2}=0$. Al despejar $x_f$ obtenemos

\begin{equation}
	x_f(1+\frac{r}{1+x_f^2})=0	
\label{puntosfijos}
\end{equation}

esta ecuaci\'on se cumple si $x_f=0$ \'o si $(1-\frac{r}{1+x_f^2})=0$ por lo 
tanto 
$x_f=0$ siempre ser\'a punto fijo mientras que para el otro 
factor debemos analizar la forma de los puntos fijos. Para el segundo factor si despejamos $x_f$ se obtiene

\begin{equation}
	x_f=\pm\sqrt{-r-1}
\label{factor}
\end{equation}

por lo tanto los puntos fijos son $x_f=0$ y (\ref{factor}), analizando el segundo observamos que $x_f$ tendr\'a valores reales s\'olo si 
$-r-1\geq0$ lo que implica que $r\leq-1$ si $r=-1$ entonces $x_f=0$ es el \'unico punto fijo, si $r<-1$ los puntos fijos ser\'ian dos y 
estar\'ian determinados por (\ref{factor}) por lo tanto podemos intuir que el punto $r_c$ donde ocurre la bifurcaci\'on es $r_c=-1$.

\subsection{Estabilidad}
Para determinar si los puntos fijos son estables o no, se construye un criterio de estabilidad mediante un analisis lineal. Si un punto fijo 
es 
estable 
entonces al hacer 
una leve perturbaci\'on sobre el punto fijo $x_f+\eta$ el sistema deber\'ia regresar al punto fijo $x_f$ conforme pasa el tiempo. Considerando 
el punto $x=x_f + \eta$ que se encuentra cerca del punto fijo tenemos que su evoluci\'on en el tiempo ser\'a $\frac{dx}{dt}= \frac{dx_f}{dt} + 
\frac{d\eta}{dt}$ que depende realmente de c\'omo evoluciona la perturbaci\'on $\eta$, por lo tanto $\dot{x}=\dot{\eta}=f(x)=f(x_f+\eta)$ 
donde al hacer una expansi\'on en serie de Taylor obtenemos

\begin{equation}
	\dot{\eta}=f(x_f) + \eta f'(x_f) + O(\eta^2)
\label{cond3}
\end{equation}

despreciando los t\'erminos de orden $\eta^2$ (suponiendo una perturbaci\'on lo suficientemente leve) y tomando en cuenta que $f(x_f)=0$ nos 
queda una ecuaci\'on 
diferencial que nos revela c\'omo se 
comporta la perturbaci\'on $\eta$.

\begin{equation}
	\dot{\eta} \approx \eta f'(x_f)
\label{cond4}
\end{equation}

de la cual la soluci\'on es 

\begin{equation}
	\eta=ke^{tf'(x_f)}
\label{cond5}
\end{equation}

De la soluci\'on podemos observar que si $f'(x_f)<0$ entonces la perturbaci\'on $\eta$ se har\'a cero al pasar el tiempo, lo que significa que 
$x$ regresa a $x_f$ por lo tanto $x_f$ ser\'ia estable. Si $f'(x_f)=0$ entonces los t\'erminos $O(\eta^2)$ no son despreciables y se requiere 
otro tipo de an\'alisis que en este caso ser\'a num\'erico y si $f'(x_f)>0$ la 
perturbaci\'on crece por lo tanto $x_f$ ser\'ia inestable. Este es el criterio de estabilidad. 

Para nuestro sistema tenemos lo siguiente

\begin{equation}
	f'(x)=1+\frac{r(1-x^2)}{(1+x^2)^2}
\label{spline}
\end{equation}

entonces para el punto fijo $x_f=0$ tenemos que $f'(0)=1+r$ lo que implica que $x_f=0$ ser\'a estable si $r<-1$ e inestable si $r>-1$, para 
cuando $r=-1$ basta observar la gr\'afica de $f(x)$ (figura 1). 

\begin{figure}[h]
\begin{center}
	% GNUPLOT: LaTeX picture
\setlength{\unitlength}{0.240900pt}
\ifx\plotpoint\undefined\newsavebox{\plotpoint}\fi
\sbox{\plotpoint}{\rule[-0.200pt]{0.400pt}{0.400pt}}%
\begin{picture}(1500,900)(0,0)
\sbox{\plotpoint}{\rule[-0.200pt]{0.400pt}{0.400pt}}%
\put(171.0,131.0){\rule[-0.200pt]{4.818pt}{0.400pt}}
\put(151,131){\makebox(0,0)[r]{$-1$}}
\put(1419.0,131.0){\rule[-0.200pt]{4.818pt}{0.400pt}}
\put(171.0,313.0){\rule[-0.200pt]{4.818pt}{0.400pt}}
\put(151,313){\makebox(0,0)[r]{$-0.5$}}
\put(1419.0,313.0){\rule[-0.200pt]{4.818pt}{0.400pt}}
\put(171.0,495.0){\rule[-0.200pt]{4.818pt}{0.400pt}}
\put(151,495){\makebox(0,0)[r]{$0$}}
\put(1419.0,495.0){\rule[-0.200pt]{4.818pt}{0.400pt}}
\put(171.0,677.0){\rule[-0.200pt]{4.818pt}{0.400pt}}
\put(151,677){\makebox(0,0)[r]{$0.5$}}
\put(1419.0,677.0){\rule[-0.200pt]{4.818pt}{0.400pt}}
\put(171.0,859.0){\rule[-0.200pt]{4.818pt}{0.400pt}}
\put(151,859){\makebox(0,0)[r]{$1$}}
\put(1419.0,859.0){\rule[-0.200pt]{4.818pt}{0.400pt}}
\put(171.0,131.0){\rule[-0.200pt]{0.400pt}{4.818pt}}
\put(171,90){\makebox(0,0){$-1$}}
\put(171.0,839.0){\rule[-0.200pt]{0.400pt}{4.818pt}}
\put(488.0,131.0){\rule[-0.200pt]{0.400pt}{4.818pt}}
\put(488,90){\makebox(0,0){$-0.5$}}
\put(488.0,839.0){\rule[-0.200pt]{0.400pt}{4.818pt}}
\put(805.0,131.0){\rule[-0.200pt]{0.400pt}{4.818pt}}
\put(805,90){\makebox(0,0){$0$}}
\put(805.0,839.0){\rule[-0.200pt]{0.400pt}{4.818pt}}
\put(1122.0,131.0){\rule[-0.200pt]{0.400pt}{4.818pt}}
\put(1122,90){\makebox(0,0){$0.5$}}
\put(1122.0,839.0){\rule[-0.200pt]{0.400pt}{4.818pt}}
\put(1439.0,131.0){\rule[-0.200pt]{0.400pt}{4.818pt}}
\put(1439,90){\makebox(0,0){$1$}}
\put(1439.0,839.0){\rule[-0.200pt]{0.400pt}{4.818pt}}
\put(171.0,495.0){\rule[-0.200pt]{305.461pt}{0.400pt}}
\put(805.0,131.0){\rule[-0.200pt]{0.400pt}{175.375pt}}
\put(171.0,131.0){\rule[-0.200pt]{0.400pt}{175.375pt}}
\put(171.0,131.0){\rule[-0.200pt]{305.461pt}{0.400pt}}
\put(1439.0,131.0){\rule[-0.200pt]{0.400pt}{175.375pt}}
\put(171.0,859.0){\rule[-0.200pt]{305.461pt}{0.400pt}}
\put(30,495){\makebox(0,0){$f(x)=\dot{x}$}}
\put(805,29){\makebox(0,0){$x$}}
\put(171,313){\usebox{\plotpoint}}
\multiput(171.00,313.59)(0.950,0.485){11}{\rule{0.843pt}{0.117pt}}
\multiput(171.00,312.17)(11.251,7.000){2}{\rule{0.421pt}{0.400pt}}
\multiput(184.00,320.59)(0.824,0.488){13}{\rule{0.750pt}{0.117pt}}
\multiput(184.00,319.17)(11.443,8.000){2}{\rule{0.375pt}{0.400pt}}
\multiput(197.00,328.59)(0.874,0.485){11}{\rule{0.786pt}{0.117pt}}
\multiput(197.00,327.17)(10.369,7.000){2}{\rule{0.393pt}{0.400pt}}
\multiput(209.00,335.59)(0.950,0.485){11}{\rule{0.843pt}{0.117pt}}
\multiput(209.00,334.17)(11.251,7.000){2}{\rule{0.421pt}{0.400pt}}
\multiput(222.00,342.59)(0.950,0.485){11}{\rule{0.843pt}{0.117pt}}
\multiput(222.00,341.17)(11.251,7.000){2}{\rule{0.421pt}{0.400pt}}
\multiput(235.00,349.59)(0.950,0.485){11}{\rule{0.843pt}{0.117pt}}
\multiput(235.00,348.17)(11.251,7.000){2}{\rule{0.421pt}{0.400pt}}
\multiput(248.00,356.59)(1.123,0.482){9}{\rule{0.967pt}{0.116pt}}
\multiput(248.00,355.17)(10.994,6.000){2}{\rule{0.483pt}{0.400pt}}
\multiput(261.00,362.59)(0.874,0.485){11}{\rule{0.786pt}{0.117pt}}
\multiput(261.00,361.17)(10.369,7.000){2}{\rule{0.393pt}{0.400pt}}
\multiput(273.00,369.59)(0.950,0.485){11}{\rule{0.843pt}{0.117pt}}
\multiput(273.00,368.17)(11.251,7.000){2}{\rule{0.421pt}{0.400pt}}
\multiput(286.00,376.59)(1.123,0.482){9}{\rule{0.967pt}{0.116pt}}
\multiput(286.00,375.17)(10.994,6.000){2}{\rule{0.483pt}{0.400pt}}
\multiput(299.00,382.59)(1.123,0.482){9}{\rule{0.967pt}{0.116pt}}
\multiput(299.00,381.17)(10.994,6.000){2}{\rule{0.483pt}{0.400pt}}
\multiput(312.00,388.59)(1.123,0.482){9}{\rule{0.967pt}{0.116pt}}
\multiput(312.00,387.17)(10.994,6.000){2}{\rule{0.483pt}{0.400pt}}
\multiput(325.00,394.59)(1.123,0.482){9}{\rule{0.967pt}{0.116pt}}
\multiput(325.00,393.17)(10.994,6.000){2}{\rule{0.483pt}{0.400pt}}
\multiput(338.00,400.59)(1.033,0.482){9}{\rule{0.900pt}{0.116pt}}
\multiput(338.00,399.17)(10.132,6.000){2}{\rule{0.450pt}{0.400pt}}
\multiput(350.00,406.59)(1.123,0.482){9}{\rule{0.967pt}{0.116pt}}
\multiput(350.00,405.17)(10.994,6.000){2}{\rule{0.483pt}{0.400pt}}
\multiput(363.00,412.59)(1.123,0.482){9}{\rule{0.967pt}{0.116pt}}
\multiput(363.00,411.17)(10.994,6.000){2}{\rule{0.483pt}{0.400pt}}
\multiput(376.00,418.59)(1.378,0.477){7}{\rule{1.140pt}{0.115pt}}
\multiput(376.00,417.17)(10.634,5.000){2}{\rule{0.570pt}{0.400pt}}
\multiput(389.00,423.59)(1.378,0.477){7}{\rule{1.140pt}{0.115pt}}
\multiput(389.00,422.17)(10.634,5.000){2}{\rule{0.570pt}{0.400pt}}
\multiput(402.00,428.59)(1.267,0.477){7}{\rule{1.060pt}{0.115pt}}
\multiput(402.00,427.17)(9.800,5.000){2}{\rule{0.530pt}{0.400pt}}
\multiput(414.00,433.59)(1.378,0.477){7}{\rule{1.140pt}{0.115pt}}
\multiput(414.00,432.17)(10.634,5.000){2}{\rule{0.570pt}{0.400pt}}
\multiput(427.00,438.59)(1.378,0.477){7}{\rule{1.140pt}{0.115pt}}
\multiput(427.00,437.17)(10.634,5.000){2}{\rule{0.570pt}{0.400pt}}
\multiput(440.00,443.60)(1.797,0.468){5}{\rule{1.400pt}{0.113pt}}
\multiput(440.00,442.17)(10.094,4.000){2}{\rule{0.700pt}{0.400pt}}
\multiput(453.00,447.59)(1.378,0.477){7}{\rule{1.140pt}{0.115pt}}
\multiput(453.00,446.17)(10.634,5.000){2}{\rule{0.570pt}{0.400pt}}
\multiput(466.00,452.60)(1.651,0.468){5}{\rule{1.300pt}{0.113pt}}
\multiput(466.00,451.17)(9.302,4.000){2}{\rule{0.650pt}{0.400pt}}
\multiput(478.00,456.60)(1.797,0.468){5}{\rule{1.400pt}{0.113pt}}
\multiput(478.00,455.17)(10.094,4.000){2}{\rule{0.700pt}{0.400pt}}
\multiput(491.00,460.61)(2.695,0.447){3}{\rule{1.833pt}{0.108pt}}
\multiput(491.00,459.17)(9.195,3.000){2}{\rule{0.917pt}{0.400pt}}
\multiput(504.00,463.60)(1.797,0.468){5}{\rule{1.400pt}{0.113pt}}
\multiput(504.00,462.17)(10.094,4.000){2}{\rule{0.700pt}{0.400pt}}
\multiput(517.00,467.61)(2.695,0.447){3}{\rule{1.833pt}{0.108pt}}
\multiput(517.00,466.17)(9.195,3.000){2}{\rule{0.917pt}{0.400pt}}
\multiput(530.00,470.61)(2.472,0.447){3}{\rule{1.700pt}{0.108pt}}
\multiput(530.00,469.17)(8.472,3.000){2}{\rule{0.850pt}{0.400pt}}
\multiput(542.00,473.61)(2.695,0.447){3}{\rule{1.833pt}{0.108pt}}
\multiput(542.00,472.17)(9.195,3.000){2}{\rule{0.917pt}{0.400pt}}
\put(555,476.17){\rule{2.700pt}{0.400pt}}
\multiput(555.00,475.17)(7.396,2.000){2}{\rule{1.350pt}{0.400pt}}
\multiput(568.00,478.61)(2.695,0.447){3}{\rule{1.833pt}{0.108pt}}
\multiput(568.00,477.17)(9.195,3.000){2}{\rule{0.917pt}{0.400pt}}
\put(581,481.17){\rule{2.700pt}{0.400pt}}
\multiput(581.00,480.17)(7.396,2.000){2}{\rule{1.350pt}{0.400pt}}
\put(594,483.17){\rule{2.500pt}{0.400pt}}
\multiput(594.00,482.17)(6.811,2.000){2}{\rule{1.250pt}{0.400pt}}
\put(606,485.17){\rule{2.700pt}{0.400pt}}
\multiput(606.00,484.17)(7.396,2.000){2}{\rule{1.350pt}{0.400pt}}
\put(619,486.67){\rule{3.132pt}{0.400pt}}
\multiput(619.00,486.17)(6.500,1.000){2}{\rule{1.566pt}{0.400pt}}
\put(632,487.67){\rule{3.132pt}{0.400pt}}
\multiput(632.00,487.17)(6.500,1.000){2}{\rule{1.566pt}{0.400pt}}
\put(645,489.17){\rule{2.700pt}{0.400pt}}
\multiput(645.00,488.17)(7.396,2.000){2}{\rule{1.350pt}{0.400pt}}
\put(658,490.67){\rule{3.132pt}{0.400pt}}
\multiput(658.00,490.17)(6.500,1.000){2}{\rule{1.566pt}{0.400pt}}
\put(671,491.67){\rule{2.891pt}{0.400pt}}
\multiput(671.00,491.17)(6.000,1.000){2}{\rule{1.445pt}{0.400pt}}
\put(696,492.67){\rule{3.132pt}{0.400pt}}
\multiput(696.00,492.17)(6.500,1.000){2}{\rule{1.566pt}{0.400pt}}
\put(683.0,493.0){\rule[-0.200pt]{3.132pt}{0.400pt}}
\put(722,493.67){\rule{3.132pt}{0.400pt}}
\multiput(722.00,493.17)(6.500,1.000){2}{\rule{1.566pt}{0.400pt}}
\put(709.0,494.0){\rule[-0.200pt]{3.132pt}{0.400pt}}
\put(875,494.67){\rule{3.132pt}{0.400pt}}
\multiput(875.00,494.17)(6.500,1.000){2}{\rule{1.566pt}{0.400pt}}
\put(735.0,495.0){\rule[-0.200pt]{33.726pt}{0.400pt}}
\put(901,495.67){\rule{3.132pt}{0.400pt}}
\multiput(901.00,495.17)(6.500,1.000){2}{\rule{1.566pt}{0.400pt}}
\put(888.0,496.0){\rule[-0.200pt]{3.132pt}{0.400pt}}
\put(927,496.67){\rule{2.891pt}{0.400pt}}
\multiput(927.00,496.17)(6.000,1.000){2}{\rule{1.445pt}{0.400pt}}
\put(939,497.67){\rule{3.132pt}{0.400pt}}
\multiput(939.00,497.17)(6.500,1.000){2}{\rule{1.566pt}{0.400pt}}
\put(952,499.17){\rule{2.700pt}{0.400pt}}
\multiput(952.00,498.17)(7.396,2.000){2}{\rule{1.350pt}{0.400pt}}
\put(965,500.67){\rule{3.132pt}{0.400pt}}
\multiput(965.00,500.17)(6.500,1.000){2}{\rule{1.566pt}{0.400pt}}
\put(978,501.67){\rule{3.132pt}{0.400pt}}
\multiput(978.00,501.17)(6.500,1.000){2}{\rule{1.566pt}{0.400pt}}
\put(991,503.17){\rule{2.700pt}{0.400pt}}
\multiput(991.00,502.17)(7.396,2.000){2}{\rule{1.350pt}{0.400pt}}
\put(1004,505.17){\rule{2.500pt}{0.400pt}}
\multiput(1004.00,504.17)(6.811,2.000){2}{\rule{1.250pt}{0.400pt}}
\put(1016,507.17){\rule{2.700pt}{0.400pt}}
\multiput(1016.00,506.17)(7.396,2.000){2}{\rule{1.350pt}{0.400pt}}
\multiput(1029.00,509.61)(2.695,0.447){3}{\rule{1.833pt}{0.108pt}}
\multiput(1029.00,508.17)(9.195,3.000){2}{\rule{0.917pt}{0.400pt}}
\put(1042,512.17){\rule{2.700pt}{0.400pt}}
\multiput(1042.00,511.17)(7.396,2.000){2}{\rule{1.350pt}{0.400pt}}
\multiput(1055.00,514.61)(2.695,0.447){3}{\rule{1.833pt}{0.108pt}}
\multiput(1055.00,513.17)(9.195,3.000){2}{\rule{0.917pt}{0.400pt}}
\multiput(1068.00,517.61)(2.472,0.447){3}{\rule{1.700pt}{0.108pt}}
\multiput(1068.00,516.17)(8.472,3.000){2}{\rule{0.850pt}{0.400pt}}
\multiput(1080.00,520.61)(2.695,0.447){3}{\rule{1.833pt}{0.108pt}}
\multiput(1080.00,519.17)(9.195,3.000){2}{\rule{0.917pt}{0.400pt}}
\multiput(1093.00,523.60)(1.797,0.468){5}{\rule{1.400pt}{0.113pt}}
\multiput(1093.00,522.17)(10.094,4.000){2}{\rule{0.700pt}{0.400pt}}
\multiput(1106.00,527.61)(2.695,0.447){3}{\rule{1.833pt}{0.108pt}}
\multiput(1106.00,526.17)(9.195,3.000){2}{\rule{0.917pt}{0.400pt}}
\multiput(1119.00,530.60)(1.797,0.468){5}{\rule{1.400pt}{0.113pt}}
\multiput(1119.00,529.17)(10.094,4.000){2}{\rule{0.700pt}{0.400pt}}
\multiput(1132.00,534.60)(1.651,0.468){5}{\rule{1.300pt}{0.113pt}}
\multiput(1132.00,533.17)(9.302,4.000){2}{\rule{0.650pt}{0.400pt}}
\multiput(1144.00,538.59)(1.378,0.477){7}{\rule{1.140pt}{0.115pt}}
\multiput(1144.00,537.17)(10.634,5.000){2}{\rule{0.570pt}{0.400pt}}
\multiput(1157.00,543.60)(1.797,0.468){5}{\rule{1.400pt}{0.113pt}}
\multiput(1157.00,542.17)(10.094,4.000){2}{\rule{0.700pt}{0.400pt}}
\multiput(1170.00,547.59)(1.378,0.477){7}{\rule{1.140pt}{0.115pt}}
\multiput(1170.00,546.17)(10.634,5.000){2}{\rule{0.570pt}{0.400pt}}
\multiput(1183.00,552.59)(1.378,0.477){7}{\rule{1.140pt}{0.115pt}}
\multiput(1183.00,551.17)(10.634,5.000){2}{\rule{0.570pt}{0.400pt}}
\multiput(1196.00,557.59)(1.267,0.477){7}{\rule{1.060pt}{0.115pt}}
\multiput(1196.00,556.17)(9.800,5.000){2}{\rule{0.530pt}{0.400pt}}
\multiput(1208.00,562.59)(1.378,0.477){7}{\rule{1.140pt}{0.115pt}}
\multiput(1208.00,561.17)(10.634,5.000){2}{\rule{0.570pt}{0.400pt}}
\multiput(1221.00,567.59)(1.378,0.477){7}{\rule{1.140pt}{0.115pt}}
\multiput(1221.00,566.17)(10.634,5.000){2}{\rule{0.570pt}{0.400pt}}
\multiput(1234.00,572.59)(1.123,0.482){9}{\rule{0.967pt}{0.116pt}}
\multiput(1234.00,571.17)(10.994,6.000){2}{\rule{0.483pt}{0.400pt}}
\multiput(1247.00,578.59)(1.123,0.482){9}{\rule{0.967pt}{0.116pt}}
\multiput(1247.00,577.17)(10.994,6.000){2}{\rule{0.483pt}{0.400pt}}
\multiput(1260.00,584.59)(1.033,0.482){9}{\rule{0.900pt}{0.116pt}}
\multiput(1260.00,583.17)(10.132,6.000){2}{\rule{0.450pt}{0.400pt}}
\multiput(1272.00,590.59)(1.123,0.482){9}{\rule{0.967pt}{0.116pt}}
\multiput(1272.00,589.17)(10.994,6.000){2}{\rule{0.483pt}{0.400pt}}
\multiput(1285.00,596.59)(1.123,0.482){9}{\rule{0.967pt}{0.116pt}}
\multiput(1285.00,595.17)(10.994,6.000){2}{\rule{0.483pt}{0.400pt}}
\multiput(1298.00,602.59)(1.123,0.482){9}{\rule{0.967pt}{0.116pt}}
\multiput(1298.00,601.17)(10.994,6.000){2}{\rule{0.483pt}{0.400pt}}
\multiput(1311.00,608.59)(1.123,0.482){9}{\rule{0.967pt}{0.116pt}}
\multiput(1311.00,607.17)(10.994,6.000){2}{\rule{0.483pt}{0.400pt}}
\multiput(1324.00,614.59)(0.950,0.485){11}{\rule{0.843pt}{0.117pt}}
\multiput(1324.00,613.17)(11.251,7.000){2}{\rule{0.421pt}{0.400pt}}
\multiput(1337.00,621.59)(0.874,0.485){11}{\rule{0.786pt}{0.117pt}}
\multiput(1337.00,620.17)(10.369,7.000){2}{\rule{0.393pt}{0.400pt}}
\multiput(1349.00,628.59)(1.123,0.482){9}{\rule{0.967pt}{0.116pt}}
\multiput(1349.00,627.17)(10.994,6.000){2}{\rule{0.483pt}{0.400pt}}
\multiput(1362.00,634.59)(0.950,0.485){11}{\rule{0.843pt}{0.117pt}}
\multiput(1362.00,633.17)(11.251,7.000){2}{\rule{0.421pt}{0.400pt}}
\multiput(1375.00,641.59)(0.950,0.485){11}{\rule{0.843pt}{0.117pt}}
\multiput(1375.00,640.17)(11.251,7.000){2}{\rule{0.421pt}{0.400pt}}
\multiput(1388.00,648.59)(0.950,0.485){11}{\rule{0.843pt}{0.117pt}}
\multiput(1388.00,647.17)(11.251,7.000){2}{\rule{0.421pt}{0.400pt}}
\multiput(1401.00,655.59)(0.874,0.485){11}{\rule{0.786pt}{0.117pt}}
\multiput(1401.00,654.17)(10.369,7.000){2}{\rule{0.393pt}{0.400pt}}
\multiput(1413.00,662.59)(0.824,0.488){13}{\rule{0.750pt}{0.117pt}}
\multiput(1413.00,661.17)(11.443,8.000){2}{\rule{0.375pt}{0.400pt}}
\multiput(1426.00,670.59)(0.950,0.485){11}{\rule{0.843pt}{0.117pt}}
\multiput(1426.00,669.17)(11.251,7.000){2}{\rule{0.421pt}{0.400pt}}
\put(914.0,497.0){\rule[-0.200pt]{3.132pt}{0.400pt}}
\put(171.0,131.0){\rule[-0.200pt]{0.400pt}{175.375pt}}
\put(171.0,131.0){\rule[-0.200pt]{305.461pt}{0.400pt}}
\put(1439.0,131.0){\rule[-0.200pt]{0.400pt}{175.375pt}}
\put(171.0,859.0){\rule[-0.200pt]{305.461pt}{0.400pt}}
\end{picture}

\caption{gr\'afica de $f(x)$ con $r=-1$}
\end{center}
\end{figure}

analizando la gr\'afica la direcci\'on del flujo nos indica que $x_f=0$ es 
inestable cuando $r=-1$ y ah\'i es cuando comienza la inestabilidad de este 
punto. Para los puntos fijos 
determinados por la ecuaci\'on (\ref{factor}) evaluamos $f'(x_f)$ 
obteniendo

\begin{equation}
	f'(x_f)=1+\frac{r(1+r+1)}{(1-r-1)^2}=1+\frac{r+2}{r}=2(1+\frac{1}{r})	
\label{est}
\end{equation}

como estos puntos fijos s\'olo existen si $r<-1$ entonces obtenemos que los puntos fijos obtenidos por (\ref{factor}) ser\'an 
inestables si $\frac{1}{r}>-1$ lo que implica que $0<r<-1$ por lo tanto todos son inestables. Si $r=-1$ tenemos el punto $x_f=0$.

\section{Diagrama $x_f$ vs $r$}
El diagrama se obtiene haciendo una iteraci\'on sobre el par\'ametro $r$ donde en 
cada paso se resuelve por los $x_f$ tales que $f(x_f)=0$, para resolver se 
utiliza el m\'etodo para encontrar raices de Newton. Para la aproximaci\'on 
inicial en el m\'etodo de Newton usamos el an\'alisis de los 
puntos fijos, es decir, sabemos que son dos cuando $r<-1$ y sabemos que tienen la 
forma de la ecuaci\'on (\ref{factor}) por lo tanto escogemos una 
aproximaci\'on inicial $aprox=\sqrt{r}$, se escogi\'o esta porque sabemos que 
est\'a cerca del punto fijo diferente de cero y est\'a tambi\'en lo suficiente 
lejos de $x_f=0$ para no caer en ese punto. La ecuaci\'on (\ref{factor}) nos 
indica que las raices son sim\'etricas por lo tanto tambi\'en buscamos la ra\'iz 
cerca de $aprox=-\sqrt{r}$ para obtener el par de puntos fijos. Entonces una 
iteraci\'on sobre $r$ desde $-5$ hasta $0$ con un paso de $0.1$ nos genera la 
gr\'afica siguiente.
 
\begin{figure}[h]
\begin{center}
	% GNUPLOT: LaTeX picture
\setlength{\unitlength}{0.240900pt}
\ifx\plotpoint\undefined\newsavebox{\plotpoint}\fi
\sbox{\plotpoint}{\rule[-0.200pt]{0.400pt}{0.400pt}}%
\begin{picture}(1500,900)(0,0)
\sbox{\plotpoint}{\rule[-0.200pt]{0.400pt}{0.400pt}}%
\put(131.0,131.0){\rule[-0.200pt]{4.818pt}{0.400pt}}
\put(111,131){\makebox(0,0)[r]{$-3$}}
\put(1419.0,131.0){\rule[-0.200pt]{4.818pt}{0.400pt}}
\put(131.0,252.0){\rule[-0.200pt]{4.818pt}{0.400pt}}
\put(111,252){\makebox(0,0)[r]{$-2$}}
\put(1419.0,252.0){\rule[-0.200pt]{4.818pt}{0.400pt}}
\put(131.0,374.0){\rule[-0.200pt]{4.818pt}{0.400pt}}
\put(111,374){\makebox(0,0)[r]{$-1$}}
\put(1419.0,374.0){\rule[-0.200pt]{4.818pt}{0.400pt}}
\put(131.0,495.0){\rule[-0.200pt]{4.818pt}{0.400pt}}
\put(111,495){\makebox(0,0)[r]{$0$}}
\put(1419.0,495.0){\rule[-0.200pt]{4.818pt}{0.400pt}}
\put(131.0,616.0){\rule[-0.200pt]{4.818pt}{0.400pt}}
\put(111,616){\makebox(0,0)[r]{$1$}}
\put(1419.0,616.0){\rule[-0.200pt]{4.818pt}{0.400pt}}
\put(131.0,738.0){\rule[-0.200pt]{4.818pt}{0.400pt}}
\put(111,738){\makebox(0,0)[r]{$2$}}
\put(1419.0,738.0){\rule[-0.200pt]{4.818pt}{0.400pt}}
\put(131.0,859.0){\rule[-0.200pt]{4.818pt}{0.400pt}}
\put(111,859){\makebox(0,0)[r]{$3$}}
\put(1419.0,859.0){\rule[-0.200pt]{4.818pt}{0.400pt}}
\put(131.0,131.0){\rule[-0.200pt]{0.400pt}{4.818pt}}
\put(131,90){\makebox(0,0){$-5$}}
\put(131.0,839.0){\rule[-0.200pt]{0.400pt}{4.818pt}}
\put(393.0,131.0){\rule[-0.200pt]{0.400pt}{4.818pt}}
\put(393,90){\makebox(0,0){$-4$}}
\put(393.0,839.0){\rule[-0.200pt]{0.400pt}{4.818pt}}
\put(654.0,131.0){\rule[-0.200pt]{0.400pt}{4.818pt}}
\put(654,90){\makebox(0,0){$-3$}}
\put(654.0,839.0){\rule[-0.200pt]{0.400pt}{4.818pt}}
\put(916.0,131.0){\rule[-0.200pt]{0.400pt}{4.818pt}}
\put(916,90){\makebox(0,0){$-2$}}
\put(916.0,839.0){\rule[-0.200pt]{0.400pt}{4.818pt}}
\put(1177.0,131.0){\rule[-0.200pt]{0.400pt}{4.818pt}}
\put(1177,90){\makebox(0,0){$-1$}}
\put(1177.0,839.0){\rule[-0.200pt]{0.400pt}{4.818pt}}
\put(1439.0,131.0){\rule[-0.200pt]{0.400pt}{4.818pt}}
\put(1439,90){\makebox(0,0){$0$}}
\put(1439.0,839.0){\rule[-0.200pt]{0.400pt}{4.818pt}}
\put(131.0,131.0){\rule[-0.200pt]{0.400pt}{175.375pt}}
\put(131.0,131.0){\rule[-0.200pt]{315.097pt}{0.400pt}}
\put(1439.0,131.0){\rule[-0.200pt]{0.400pt}{175.375pt}}
\put(131.0,859.0){\rule[-0.200pt]{315.097pt}{0.400pt}}
\put(30,495){\makebox(0,0){$x_f$}}
\put(785,29){\makebox(0,0){$r$}}
\put(131,738){\makebox(0,0){$\bullet$}}
\put(131,252){\makebox(0,0){$\bullet$}}
\put(157,735){\makebox(0,0){$\bullet$}}
\put(157,255){\makebox(0,0){$\bullet$}}
\put(183,732){\makebox(0,0){$\bullet$}}
\put(183,258){\makebox(0,0){$\bullet$}}
\put(209,728){\makebox(0,0){$\bullet$}}
\put(209,262){\makebox(0,0){$\bullet$}}
\put(236,725){\makebox(0,0){$\bullet$}}
\put(236,265){\makebox(0,0){$\bullet$}}
\put(262,722){\makebox(0,0){$\bullet$}}
\put(262,268){\makebox(0,0){$\bullet$}}
\put(288,719){\makebox(0,0){$\bullet$}}
\put(288,271){\makebox(0,0){$\bullet$}}
\put(314,715){\makebox(0,0){$\bullet$}}
\put(314,275){\makebox(0,0){$\bullet$}}
\put(340,712){\makebox(0,0){$\bullet$}}
\put(340,278){\makebox(0,0){$\bullet$}}
\put(366,709){\makebox(0,0){$\bullet$}}
\put(366,281){\makebox(0,0){$\bullet$}}
\put(393,705){\makebox(0,0){$\bullet$}}
\put(393,285){\makebox(0,0){$\bullet$}}
\put(419,702){\makebox(0,0){$\bullet$}}
\put(419,288){\makebox(0,0){$\bullet$}}
\put(445,698){\makebox(0,0){$\bullet$}}
\put(445,292){\makebox(0,0){$\bullet$}}
\put(471,694){\makebox(0,0){$\bullet$}}
\put(471,296){\makebox(0,0){$\bullet$}}
\put(497,691){\makebox(0,0){$\bullet$}}
\put(497,299){\makebox(0,0){$\bullet$}}
\put(523,687){\makebox(0,0){$\bullet$}}
\put(523,303){\makebox(0,0){$\bullet$}}
\put(550,683){\makebox(0,0){$\bullet$}}
\put(550,307){\makebox(0,0){$\bullet$}}
\put(576,679){\makebox(0,0){$\bullet$}}
\put(576,311){\makebox(0,0){$\bullet$}}
\put(602,675){\makebox(0,0){$\bullet$}}
\put(602,315){\makebox(0,0){$\bullet$}}
\put(628,671){\makebox(0,0){$\bullet$}}
\put(628,319){\makebox(0,0){$\bullet$}}
\put(654,667){\makebox(0,0){$\bullet$}}
\put(654,323){\makebox(0,0){$\bullet$}}
\put(680,662){\makebox(0,0){$\bullet$}}
\put(680,328){\makebox(0,0){$\bullet$}}
\put(707,658){\makebox(0,0){$\bullet$}}
\put(707,332){\makebox(0,0){$\bullet$}}
\put(733,653){\makebox(0,0){$\bullet$}}
\put(733,337){\makebox(0,0){$\bullet$}}
\put(759,648){\makebox(0,0){$\bullet$}}
\put(759,342){\makebox(0,0){$\bullet$}}
\put(785,644){\makebox(0,0){$\bullet$}}
\put(785,346){\makebox(0,0){$\bullet$}}
\put(811,639){\makebox(0,0){$\bullet$}}
\put(811,351){\makebox(0,0){$\bullet$}}
\put(837,633){\makebox(0,0){$\bullet$}}
\put(837,357){\makebox(0,0){$\bullet$}}
\put(863,628){\makebox(0,0){$\bullet$}}
\put(863,362){\makebox(0,0){$\bullet$}}
\put(890,622){\makebox(0,0){$\bullet$}}
\put(890,368){\makebox(0,0){$\bullet$}}
\put(916,616){\makebox(0,0){$\bullet$}}
\put(916,374){\makebox(0,0){$\bullet$}}
\put(942,610){\makebox(0,0){$\bullet$}}
\put(942,380){\makebox(0,0){$\bullet$}}
\put(968,604){\makebox(0,0){$\bullet$}}
\put(968,386){\makebox(0,0){$\bullet$}}
\put(994,597){\makebox(0,0){$\bullet$}}
\put(994,393){\makebox(0,0){$\bullet$}}
\put(1020,589){\makebox(0,0){$\bullet$}}
\put(1020,401){\makebox(0,0){$\bullet$}}
\put(1047,581){\makebox(0,0){$\bullet$}}
\put(1047,409){\makebox(0,0){$\bullet$}}
\put(1073,572){\makebox(0,0){$\bullet$}}
\put(1073,418){\makebox(0,0){$\bullet$}}
\put(1099,561){\makebox(0,0){$\bullet$}}
\put(1099,429){\makebox(0,0){$\bullet$}}
\put(1125,549){\makebox(0,0){$\bullet$}}
\put(1125,441){\makebox(0,0){$\bullet$}}
\put(1151,533){\makebox(0,0){$\bullet$}}
\put(1151,457){\makebox(0,0){$\bullet$}}
\put(1177,495){\makebox(0,0){$\bullet$}}
\put(1177,495){\makebox(0,0){$\bullet$}}
\put(1204,495){\makebox(0,0){$\bullet$}}
\put(1204,495){\makebox(0,0){$\bullet$}}
\put(1230,495){\makebox(0,0){$\bullet$}}
\put(1230,495){\makebox(0,0){$\bullet$}}
\put(1256,495){\makebox(0,0){$\bullet$}}
\put(1256,495){\makebox(0,0){$\bullet$}}
\put(1282,495){\makebox(0,0){$\bullet$}}
\put(1282,495){\makebox(0,0){$\bullet$}}
\put(1308,495){\makebox(0,0){$\bullet$}}
\put(1308,495){\makebox(0,0){$\bullet$}}
\put(1334,495){\makebox(0,0){$\bullet$}}
\put(1334,495){\makebox(0,0){$\bullet$}}
\put(1361,495){\makebox(0,0){$\bullet$}}
\put(1361,495){\makebox(0,0){$\bullet$}}
\put(1387,495){\makebox(0,0){$\bullet$}}
\put(1387,495){\makebox(0,0){$\bullet$}}
\put(1413,495){\makebox(0,0){$\bullet$}}
\put(1413,495){\makebox(0,0){$\bullet$}}
\put(1439,495){\makebox(0,0){$\bullet$}}
\put(1439,495){\makebox(0,0){$\bullet$}}
\put(131.0,131.0){\rule[-0.200pt]{0.400pt}{175.375pt}}
\put(131.0,131.0){\rule[-0.200pt]{315.097pt}{0.400pt}}
\put(1439.0,131.0){\rule[-0.200pt]{0.400pt}{175.375pt}}
\put(131.0,859.0){\rule[-0.200pt]{315.097pt}{0.400pt}}
\end{picture}

\end{center}
\end{figure}


\begin{thebibliography}{9}
\bibitem{chapra}
	Stephen H. Strogatz,
	\emph{Nonlinear dynamics and chaos}
	Westview, 2th edition, 2015.
\end{thebibliography}

\end{document}
